\documentclass[svgnames]{watsonbook}


% ----- ASY SETUP ----------------------------
\begin{asydef} 
defaultpen(fontsize(10));
settings.outformat="pdf";
usepackage("mathpazo_modified");
texpreamble("\renewcommand{\vec}[1]{\mathbf{#1}}"); 
import x11colors;
pen softblue = rgb(0.92,0.95,0.99);
pen softred = rgb(0.99, 0.92, 0.91);
pen softyellow = rgb(0.98, 0.98, 0.9);
pen softgreen = rgb(0.96, 0.995, 0.98);
settings.render = 8; 
\end{asydef}

\newsavebox{\asybox}
\def\asydir{asy}

%---- PRINT AUTHOR INFO ------------------
\makeatletter                       
\def\printauthor{%                  
  {\large \@author}}              
\makeatother
\author{%
  {\Large \textit{Samuel S.\ Watson}} \\ \vspace{2mm}  sswatson@mit.edu
}

%----- SIDE ICONS ------------------------
\newcommand{\milink}[3][-7.5mm]{\sidenote{\href{http://mathinsight.org/#2}{\mi} on #3}[#1]}
\newcommand\cocalc[1][-2pt]{\raisebox{#1}{\includegraphics[width=12pt]{figures/cocalc_new}}}
\newcommand\tbob[1][-2pt]{\raisebox{#1}{\includegraphics[width=12pt]{figures/3b1b_new}}}
\newcommand\mi[1][-2pt]{\raisebox{#1}{\includegraphics[width=12pt]{figures/mathinsight}}}

%This cover art is taken from 
%http://tex.stackexchange.com/questions/85904/showcase-of-beautiful-title-page-done-in-tex

\definecolor{titlepagecolor}{cmyk}{1,.60,0,.40}

\newcommand\titlepagedecoration{%
\begin{tikzpicture}[remember picture,overlay,shorten >= -10pt]

\coordinate (aux1) at ([yshift=-15pt]current page.north east);
\coordinate (aux2) at ([yshift=-410pt]current page.north east);
\coordinate (aux3) at ([xshift=-4.5cm]current page.north east);
\coordinate (aux4) at ([yshift=-150pt]current page.north east);

\begin{scope}[titlepagecolor!40,line width=12pt,rounded corners=12pt]
\draw
  (aux1) -- coordinate (a)
  ++(225:5) --
  ++(-45:5.1) coordinate (b);
\draw[shorten <= -10pt]
  (aux3) --
  (a) --
  (aux1);
\draw[opacity=0.6,titlepagecolor,shorten <= -10pt]
  (b) --
  ++(225:2.2) --
  ++(-45:2.2);
\end{scope}
\draw[titlepagecolor,line width=8pt,rounded corners=8pt,shorten <= -10pt]
  (aux4) --
  ++(225:0.8) --
  ++(-45:0.8);
\begin{scope}[titlepagecolor!70,line width=6pt,rounded corners=8pt]
\draw[shorten <= -10pt]
  (aux2) --
  ++(225:3) coordinate[pos=0.45] (c) --
  ++(-45:3.1);
\draw
  (aux2) --
  (c) --
  ++(135:2.5) --
  ++(45:2.5) --
  ++(-45:2.5) coordinate[pos=0.3] (d);   
\draw 
  (d) -- +(45:1);
\end{scope}
\end{tikzpicture}%
}


\begin{document} 

\begin{titlepage} 
  \pagecolor{softblue}
  {\titlefont \color{MidnightBlue} MULTIVARIABLE CALCULUS} 
  
  \vspace*{5mm} 
  { \hspace*{9cm} \subtitlefont for MIT Interphase students} \par
  \null\vfill
  \vspace*{1cm}
  \noindent
  \hfill
  \begin{minipage}{0.35\linewidth}
    \begin{flushright}
      \printauthor
    \end{flushright}
  \end{minipage}
  % 
  \begin{minipage}{0.02\linewidth}
    \rule{1pt}{125pt}
  \end{minipage}
  \titlepagedecoration
\end{titlepage}


\chapter*{Preface} 

\pagecolor{white} 

This text was written to accompany the multivariable calculus course
taught during the Office of Minority Education's six-week summer
\textit{Interphase} program at the Massachusetts Institute of
Technology. Students participating in this program have the
opportunity to take a full-semester version of multivariable calculus
in the fall semester. 

For this reason, the material presented here is substantially
streamlined compared to a typical multivariable calculus
course. Rather than thoroughly covering the first half of the standard
curriculum and leaving students with an abrupt drop-off in content
knowledge midway through the fall, I have carved out a selection of
topics which approximately spans the standard curriculum but which 
nevertheless includes the most important ideas and is self-contained.

Another casualty of the abbreviated schedule is mathematical rigor. I
emphasize concept visualization and physical intuition, and I include
proofs only when they are sufficiently illuminating to be worth the
effort. I fully embrace the physicist's perspective that one should
derive formulas by reasoning about small positive quantities like
$\Delta x$, trusting that one may replace sums with integrals and
$\Delta x$ with $\d x$ to obtain correct results. Turning these
heuristics into proofs is left to later courses (although the matter
is discussed further in Appendix~\ref{sec:centralidea}). While there
are advantages to maintaining a higher standard of rigor, my top
priority is for students to gain problem-solving facility with the
methods covered in the course.

SW \\
2017

\chapter*{For the reader}

Textbooks customarily include an abundance of material and leave you
to figure out what to focus on. By contrast, I have attempted to pare
the multivariable calculus story down to its essence. My goal is that
you will be able to read each section carefully and work all the
examples and exercises, thereby wasting no effort on figuring out what
to skip. However, problem solving is the heart of mathematics, and you
are likely to want additional exercises. I recommend the following
sources for more exercises, discussion, or examples: 
\begin{enumerate}[itemsep = 3pt]
\item Appendix~\ref{sec:addexer}, which lists additional exercises by
  section. Some of these are original, and others are
  modified from exercises in 
  Stewart's \textit{Multivariable Calculus} or Colley's \textit{Vector
    Calculus} 
\item The community calculus textbook, available at
  \texttt{communitycalculus.org}. This is an open source textbook available (for
  free) online, and it has \textit{a lot} of exercises 
\item Frederick Tsz-Ho Fong's multivariable calculus notes for Math 0200 at
  Brown University, available at
  \href{http://www.math.brown.edu/~sswatson/pdfs/frederick_multinotes.pdf}{\url{http://www.math.brown.edu/~sswatson/pdfs/frederick\_multinotes.pdf}}. These
  notes are very nicely written and have quite a few examples. 
\item MathInsight (\texttt{mathinsight.org}) has webpages for many
  multivariable topics with nice exposition and some beautiful applets for exploring the
  ideas developed in this course
\item My Math 0200 webpage, with homework sets, practice tests, and
  solutions: \\
  \href{http://www.math.brown.edu/~sswatson/math0200.html}{\url{http://www.math.brown.edu/~sswatson/math0200.html}}
\item A standard multivariable calculus textbook, such as Stewart
  or Edwards and Penney
\item 3blue1brown, a math video creator with an excellent series on
  linear algebra. Unfortunately he hasn't done multivariable calculus
  yet, so these videos will only be available for the vector topics
 \end{enumerate}

The following margin note icons in the text are clickable:
\begin{enumerate}[itemsep=6pt, topsep = -6pt]
  \item \href{http://cocalc.com}{\cocalc}\,
links to a CoCalc worksheet with a relevant calculation (see Appendix~\ref{sec:sagemath} for more
discussion)
\item \href{http://3blue1brown.com}{\tbob} \, links to a relevant 3blue1brown YouTube video.
\item \href{http://mathinsight.org}{\mi} \, links to a relevant page at mathinsight.org
\end{enumerate}

\vspace{6pt}

Almost all of the \textbf{3D graphics in this PDF may be interactively manipulated},
but that feature requires that the PDF be viewed with Adobe's (free) Acrobat
Reader
(\href{https://get.adobe.com/reader/}{\url{https://get.adobe.com/reader/}}). The
references in this text are hyperlinked, which means for example that you can click on
Theorem~\ref{th:green} to navigate directly to Green's theorem. 

Please do not hesitate to contact me via email (sswatson@mit.edu)
about any mistakes you find in these notes, no matter how
minor. Enjoy!

\newpage

\null\vfill

\includegraphics[width=3cm]{figures/cc-by-nc-sa.pdf} \\
{\small
This work is licensed under the Creative Commons
Attribution-NonCommercial-ShareAlike License. To view a copy of this
license, visit 
\url{http://creativecommons.org/licenses/by-nc-sa/3.0/} 
or
send a letter to Creative Commons, 543 Howard Street, 5th Floor, San
Francisco, California, 94105, USA. If you distribute this work or a
derivative, include the history of the document.

These notes are an original work of the author. 
}

\vspace{5cm}


\newpage 

\tableofcontents

\newpage 

\chapter{Transformations of Euclidean space}

\section{\textit{n}-dimensional space} \label{sec:ndimspace} 

\reversemarginpar

\setlength{\wrapoverhang}{\marginparwidth}

\begin{wrapfigure}[20]{R}{5cm}
\begin{asy} 
size(5cm);
defaultpen(fontsize(8));
draw((-5,0)--(5,0),Arrows());
for(int k=-4;k<=4;++k){
  draw((k,-0.2)--(k,0.2));
  label("$" + string(k) + "$",(k - ( k >= 0 ? 0.0 : 0.18),0),align=2*S);
}
\end{asy} 
\caption{The real number line \label{fig:real}}

\vspace{12pt}

\begin{asy} 
size(5cm);
draw((-5,0)--(5,0),Arrows());
draw((0,5)--(0,-5),Arrows());
for(int k=-4;k<=4;++k){
  draw((k,-0.2)--(k,0.2));
  draw((-0.2,k)--(0.2,k));
}
dot((-2,3));
real ep = 0.2; 
label("$(-2,3)$",(-2,3),align=W);
draw("$-2$",(-ep,3)--(-2+ep,3),MidnightBlue,Arrows());
draw("$3$",(-2,ep)--(-2,3-ep),MidnightBlue,Arrows());
label("$x$",(5,0),align=E);
label("$y$",(0,5),align=N);
\end{asy} 
\caption{Coordinates in $\R^2$ \label{fig:plane}}

\vspace{12pt}

\begin{asy}
size(5cm);
import three;
draw(O--4*X,Arrow3());
draw(O--4*Y,Arrow3());
draw(O--4*Z,Arrow3());
dot((1,2,3)); 
label("$(1,2,3)$",(1,2,3),align=2*NNE);
real ep = 0.1;
draw((1,2,0)--(1,2,3),dashed);
draw((1,2-ep,0)--(1,2-ep,ep)--(1,2,ep));
draw((1,0,3)--(1,2,3),dashed);
draw((1,0,3-ep)--(1,ep,3-ep)--(1,ep,3));
draw((0,2,3)--(1,2,3),dashed);
draw((0,2,3-ep)--(ep,2,3-ep)--(ep,2,3));
draw(surface((O--4*X--4*(X+Y)--4*Y--cycle)),LightSeaGreen+opacity(0.4));
draw(surface((O--4*Z--4*(X+Z)--4*X--cycle)),LightSeaGreen+opacity(0.4));
draw(surface((O--4*Y--4*(Z+Y)--4*Z--cycle)),LightSeaGreen+opacity(0.4));
label("$x$",(4*X),align=W);
label("$y$",(4*Y),align=E);
label("$z$",(4*Z),align=N);
\end{asy}
\caption{Coordinates in $\R^3$ \label{fig:space}}
\end{wrapfigure} 

We visualize the set of \textbf{real numbers} (denoted by $\R$ or
$\R^1$) as a line, called the \textit{real number line}
(Figure~\ref{fig:real}).

The location of a number $x$ on this line is the point whose
\textit{signed} distance from 0 is $x$. The word \textit{signed} means
that distances measured from 0 to a point which is left of 0 count as
negative.

The set $\R^2$ of ordered pairs $(x,y)$ where $x$ and $y$ are real
numbers can be thought of as a \textit{plane}, since a point in a
plane can be specified by two real numbers: its signed distances from
two lines which meet at a right angle. These two lines are called the
$x$-axis and the $y$-axis.

The set $\R^3$ of ordered triples of real numbers $(x,y,z)$ can be
visualized as a point in space, since a point in space can be
specified by three real numbers: its signed distances from three
planes which meet one another at right angles. These planes are called
the $xy$-plane, the $yz$-plane, and the $xz$-plane, and their lines of
intersection are called the $x$-axis, the $y$-axis, and the $z$-axis. 

The set $\R^4$ is defined to be the set of ordered quadruples of real
numbers, and similarly for $\R^5, \R^6, \ldots$.  
\sidenote{* Since
  we live in three dimensions, spatial visualization of $\R^n$ only
  works for $n \leq 3$. We will use other visualization tools or
  reason by analogy to build intuition in $\R^n$ for $n \geq 4$.} If
$n$ is a positive integer*, we refer to $\R^n$ as a \textbf{Euclidean
  space}. The superscript $n$
is called the \textbf{dimension} of $\R^n$. 

\begin{exercise}{}{absxy}
  Write down a formula for the distance on the number line between two \\
  real numbers $x$ and $y$. Repeat for two points in the plane, and
  for two \\ points in $\R^3$. 
\end{exercise}

\section{Functions from $\R^n$ to $\R^n$} \label{sec:RntoRn}

\subsection{Visualizing functions}

A \textbf{function} from $\R^1$ to $\R^1$ takes a real number $x$ as
input and returns another real number, denoted $f(x)$, as output. We
can draw the \textit{graph}* of such a function in $1 + 1 = 2$
dimensions, by associating the horizontal axis with input values and
the vertical axis with output values. For example, see
Figure~\ref{fig:squaring} for a graph of the squaring
function. \sidenote{* A point $(x,y)$ is on the graph of $f$ if and
  only if $f(x) = y$.}[-1cm]

A function from $\R^2$ to $\R^1$ can be visualized in $2 + 1 = 3$
dimensions by using the $xy$-plane for the input values and the
$z$-axis for the output value. In other words, we plot every triple of
the form $(x,y,f(x,y))$ where $x$ and $y$ are real numbers. See
Figure~\ref{fig:exp} for a graph of $f(x) = e^{-x^2 - y^2}$ over a
square-shaped region.

The graph of function from $\R^2$ to $\R^2$ would require $2 + 2 = 4$
dimensions to visualize, so we are out of luck there. However, we can
visualize a function from $\R^2$ to $\R^2$ drawing a picture of where
all the grid lines go*---see
Figure~\ref{fig:gridlines}.\sidenote{* One way this method of visualization
  is different from graphing is that we separate the input values on
  the left side from the output values on the right side.}

\begin{figure}[t]
  \centering
\begin{minipage}{0.49\textwidth}
\centering
\begin{asy} 
size(0,5cm);
import graph;
real f(real x){ return x*x;}
draw(graph(f,-2,2),MidnightBlue,Arrows());
draw((-2,0)--(2,0),Arrows());
draw((0,0)--(0,4),Arrow());
label("$x$",(2,0),align=E,MidnightBlue);
label("$f(x)=x^2$",(0,4),align=N,MidnightBlue);   
\end{asy}
\captionof{figure}{The graph of a function from $\R$ to $\R$ \label{fig:squaring}}
\end{minipage}
\begin{minipage}{0.49\textwidth}
\begin{asy}
size(0,5cm);
import graph3;
settings.outformat="pdf";
real f(pair z) {return exp(-z.x^2-z.y^2);}
draw(surface(f,(0,0),(1.7,1.7),nx=32,Spline),
     LightSeaGreen+opacity(0.8),MidnightBlue);
draw(O--2.2*X,Arrow3());
draw(O--2.2*Y,Arrow3());
draw(O--1.3*Z,Arrow3());
label("$x$",(2.2*X),align=W);
label("$y$",(2.2*Y),align=E);
label("$f(x,y) = e^{-x^2-y^2}$",(1.3*Z),align=N);
\end{asy}
\captionof{figure}{A graph of a function from $\R^2$ to $\R$ \label{fig:exp}}
\end{minipage}
\end{figure} 

\begin{wrapfigure}[12]{R}{10cm} 
\begin{asy} 
size(10cm);
defaultpen(fontsize(8));
import graph;
real f(real x){
  return x*x;
}
int n = 4;
draw((-n-1,0)--(n+1,0),Arrows());
draw((0,-n-1)--(0,n+1),Arrows());
picture sidepic;
size(sidepic,5cm);
add(sidepic,currentpicture);
for(int i=-n;i<=n; ++i){
  draw((i,-n)--(i,n),MidnightBlue+opacity(0.5));
  draw((-n,i)--(n,i),MidnightBlue+opacity(0.5));
}
pair f(real s, real t){
  return (s+t+0.01*t^3+1,s-t);
}
typedef pair fun(real);
fun F(real t) {
  return new pair(real s) {return f(s,t);};
}
fun G(real s) {
  return new pair(real t) {return f(s,t);};
}
for(int i=-n;i<=n;++i){
  draw(sidepic,graph(F(i),-n,n),MidnightBlue+opacity(0.5));
  draw(sidepic,graph(G(i),-n,n),MidnightBlue+opacity(0.5));
}
add(shift((15,0))*sidepic);
label("$f(x,y) = (x+y+\frac{x^3}{100}+1,x-y)$",(5,7));
draw((3,5)..(6,5.5)..(9,5),Arrow());
\end{asy} 
\caption{A transformation from $\R^2$ to $\R^2$ \label{fig:gridlines}}
\end{wrapfigure}

We often refer to a function from $\R^2$ to $\R^2$ as a
\textit{transformation}, which is just a synonym of \textit{function}
but is meant to evoke this particular method of
visualization. Functions from $\R^3$ to $\R^3$ are also called
transformations and can be visualized in the same way, but for now we
will focus on transformations from $\R^2$ to
$\R^2$.

\subsection{Linear transformations}

\sidenote{\href{https://www.youtube.com/watch?v=kYB8IZa5AuE}{\tbob}
on linear transformations}[-7.5mm]

\label{subsec:lintrans} 

One of the key ideas of differential calculus is to use
\textit{linear} functions to approximate curvy ones. Although linear
functions are simple, they are very useful because all differentiable
functions look increasingly linear as you zoom in. We will apply the
same principle in higher dimensions: use linear transformations to
approximate more complex transformations. Thus, as you did before you
learned single-variable calculus, we will begin by learning about
linear functions. 

So what \textit{is} a linear function from $\R^2$ to $\R^2$? 
Functions of the form 
$f(x) = mx + b$, where $m$ and $b$ are constants, are often called
linear. However, we will take a slightly different view by requiring
$b = 0$, so only functions of the form $f(x) = mx$ are considered
linear. Our definition of linearity in higher dimensions will work
similarly: only terms of the form ``constant times variable'' are
allowed. 

\begin{defn}{Linearity}{def:linear2}
  A function $f$ from $\R^2$ to $\R^2$ is \textbf{linear} if there exist*
  $a,b,c,d \in \R$ so that 
  \[
    f(x,y) = (ax + by , cx + dy). 
  \]
\end{defn}
\sidenote{* This notation means that $a,b,c,d$ are in the set of real numbers.}[-1cm]

\begin{figure}[h!]
\input transformations.tex
\caption{Four linear transformations from $\R^2$ to $\R^2$ \label{fig:four_trans}}
\end{figure}

Figure~\ref{fig:four_trans} shows four examples* of linear
transformations from $\R^2$ to $\R^2$. These pictures might lead us to
conjecture that linear transformations map equally spaced lines to
equally spaced lines, where coincident lines count as equally spaced
(as in the last example). This is almost accurate: sometimes equally
spaced lines can map to equally spaced \textit{points}
(Exercise~\ref{exer:espoints}). \sidenote{* These transformations \textit{scale}, \textit{shear}, \textit{rotate/scale}, and
  \textit{project}, respectively.}[-1cm]

\begin{theo}{}{equalspacing}
  A function from $\R^2$ to $\R^2$ is linear if and only if it maps
  the origin to the origin and equally spaced lines* to equally spaced
  lines or points. \sidenote{* \textit{Equally spaced lines} are lines
  which are parallel and for which the distances between consecutive
  lines are the same.}
\end{theo}

\begin{exercise}{}{}
  Use Theorem~\ref{th:equalspacing} to explain why if $f : \R^2 \to
  \R^2$ rotates every point counterclockwise about the origin by $30^\circ$, there
  necessarily exist exist $a,b,c,d\in\R$ such that $f(x,y) = (ax+ by,
  cx+ dy)$ for all $(x,y) \in\R^2$. 
\end{exercise}

\begin{exercise}{}{espoints}
  Show that the linear function $f(x,y)=(2x,0)$ maps any collection of
  equally spaced vertical lines to a collection of equally spaced points.
\end{exercise}

\section{The determinant} \label{sec:det} 

\sidenote{\href{https://www.youtube.com/watch?v=Ip3X9LOh2dk}{\tbob}
   on the determinant}[-7.5mm]

The \textit{slope} of a linear function from $\R^1$ to $\R^1$ measures
how it distorts length. For example, the function $f(x) = 3x$ maps any
interval $[a,b]$ to the interval $[3a,3b]$ which is three times as
long. The function $g(x) = -\tfrac{1}{2}x$ maps any interval to an
interval which is half as long, and it also flips the interval
around. We can say that the absolute value of the slope of a linear
function is the \textit{factor by which lengths are transformed}, and
the sign of the slope tells us whether the function reverses the real
number line.

What is the corresponding idea for transformations from $\R^2$ to
$\R^2$? Can we look at a linear transformation and conveniently
calculate the factor by which that transformation multiplies
\textit{areas}? Yes!

In each linear transformation picture in
Section~\ref{subsec:lintrans}, the quadrilaterals on the image side of
the picture are all congruent. This suggests that the linear
transformation does indeed transform every area by the same
factor. Taking this fact as given, it suffices for us to consider the
image of a single square, which we will take to be $[0,1]^2$, the set
of points both of whose coordinates are between 0 and 1.

\begin{example}{}{det2}
  Find the area of the image of the unit square $[0,1]^2$ under the transformation 
\[
f(x,y) = (ax + by, cx + dy).
\] 
\begin{center} 
\begin{asy} 
size(10cm);
settings.outformat="pdf";
draw((0,2)--(0,0)--(2,0),Arrows());
filldraw(box((0,0),(1,1)),lightgray);
draw("$1$",(1,0),align=S);
draw("$1$",(0,1),align=W);
draw(shift((3,0))*((0,2)--(0,0)--(2,0)),Arrows());
filldraw(shift((3,0))*((0,0)--(2,1)--(3,2)--(1,1)--cycle),lightgray);
dot("$(a,c)$",shift((3,0))*(2,1),align=SE);
dot("$(b,d)$",shift((3,0))*(1,1),align=NW);
\end{asy}
\end{center}
\end{example} 

\begin{solution}
The area of the unit square can be calculated by filling in some triangles to get a complete rectangle, as follows: 
\begin{center} 
\begin{asy} 
size(6cm);
draw(((0,2.3)--(0,0)--(3.3,0)),Arrows());
filldraw(((0,0)--(2,1)--(3,2)--(1,1)--cycle),lightgray);
dot("$(a,c)$",(2,1),align=2*E);
dot("$(b,d)$",(1,1),align=2*W);
dot("$(a+b,c+d)$",(3,2),align=2*E);
filldraw((0,0)--(3,0)--(2,1)--cycle,LightSeaGreen+opacity(0.5));
filldraw((0,2)--(1,1)--(3,2)--cycle,LightSeaGreen+opacity(0.5));
filldraw((0,0)--(1,1)--(0,2)--cycle,LightBlue+opacity(0.5));
filldraw((3,0)--(2,1)--(3,2)--cycle,LightBlue+opacity(0.5));
\end{asy}
\end{center} 
The area of the larger rectangle is $(a+b)(c+d)$, and the total area
of the triangles we added is
$2\cdot\tfrac{1}{2} (a+b)(c) + 2\cdot\tfrac{1}{2}
(c+d)(b)$. Subtracting, we get that the area of the parallelogram is
$ad - bc$.

We are not quite finished, however. Note that we assumed in our
diagram that the line segment to from the origin to $(a,c)$ is
clockwise from the line segment from the origin to $(b,d)$. If we
switched these line segments around, the same reasoning would have
given us the formula $bc - ad$. We can put this all together by saying
that the factor by which areas are transformed is $\boxed{|ad - bc|}$.
\end{solution}

We can interpret the result of Example~\ref{exam:det2}: computing
$ad - bc$ tells us how $f(x,y) = (ax + by, cx + dy)$ transforms areas
(via its absolute value) and whether applying $f$ to the three points
$(0,0)$, $(1,0)$, and $(0,1)$ reverses their orientation* (via its
sign). This idea is important enough to deserve its own name. To
simplify the definition, we refer to length as $1$-dimensional volume
and area as 2-dimensional volume. \sidenote{* Reversing the
  orientation of three points $A$, $B$, and $C$ means that if these
  points are given in counterclockwise order around the triangle
  $ABC$, then their images are in clockwise order around the triangle
  they form.}[-4cm]

\begin{defn}{Determinant }{def:det} \bang{-5mm}
  The \textbf{determinant} of a linear transformation from $\R^n$ to
  $\R^n$ is the signed factor by which it transforms $n$-dimensional
  volumes. 
\end{defn}

We have already figured out that the determinant of $f(x) = mx$ from
$\R^1$ to $\R^1$ is the slope $m$, and the determinant of a function
$f(x,y) = (ax + by, cx + dy)$ from $\R^2$ to $\R^2$ is given by the
formula $ad - bc$.

For convenience, we sometimes represent a linear function by arranging
its coefficients into grid of numbers called a \textit{matrix}. By
convention, rows correspond to coordinates of the output of the
function, and columns correspond to the variables. So, for example,
$f(x,y) = (ax + by, cx + dy)$ is represented by the matrix
$\left[\begin{array}{cc} a & b \\ c & d \end{array}\right]$. So we
have
\[
\det [m] = m, \text{ and} \qquad \det \left[\begin{array}{cc} a & b \\ c & d \end{array}\right] = ad - bc. 
\]

\begin{exercise}{}{det}
  Find the determinant of each of the following matrices, and draw the image of the unit square under the corresponding linear transformations to see that value of the determinant you computed makes sense. 

\pairofprobs{$\left[\begin{array}{cc} 1 & 0 \\ 0 & -1 \end{array}\right]$}{
$\left[\begin{array}{cc} 2 & 1 \\ 0 & 2 \end{array}\right]$}

\pairofprobs{$\left[\begin{array}{cc} 0 & 1 \\ -1 & 0 \end{array}\right]$}{
$\left[\begin{array}{cc} 2 & 1  \\ 4 & 2 \end{array}\right]$}
\end{exercise}

The absolute value of the determinant of a linear transformation 
\[
f(x,y,z) = (ax + by + cz, \, dx + ey + fz,\, hx + iy  + jz)
\]
is the volume of the three-dimensional shape, called a
\textit{parallelepiped} whose vertices are the images under $f$ of the
vertices of the unit cube $[0,1]^3$:

\begin{center} 
\begin{asy} 
import three;
import tube;
unitsize(1.5cm);
settings.outformat="pdf";
draw(O--3*X,Arrow3());
draw(O--3*Y,Arrow3());
draw(O--3*Z,Arrow3());
triple[] A =
  {
    (1,0,0),
    (0,1,0),
    (0,0,1)
  };
currentprojection=orthographic(4,2,1);
guide3 p = O--A[0]--A[0]+A[1]--A[1]--cycle;
draw(p);
draw(surface(p),0.5*white+0.5*blue+opacity(0.7),nolight);
guide3 p = O--A[1]--A[1]+A[2]--A[2]--cycle;
draw(p);
draw(surface(p),0.7*yellow+opacity(0.7),nolight);
guide3 p = O--A[2]--A[2]+A[0]--A[0]--cycle;
draw(p);
draw(surface(p),0.5*green+opacity(0.7),nolight);
guide3 p = A[2]--A[2]+A[0]--A[2]+A[0]+A[1]--A[2]+A[1]--cycle; 
draw(p); 
draw(surface(p),0.5*white+0.5*blue+opacity(0.7),nolight);
guide3 p = A[0]--A[0]+A[1]--A[0]+A[1]+A[2]--A[0]+A[2]--cycle;
draw(p);
draw(surface(p),0.7*yellow+opacity(0.7),nolight);
guide3 p = A[1]--A[1]+A[2]--A[1]+A[2]+A[0]--A[1]+A[0]--cycle;
draw(p);
draw(surface(p),0.5*green+opacity(0.7),nolight);
draw(shift(A[2])*(A[0]/3+A[1]/2 .. A[0]/2+A[1]/4  .. 2*A[0]/3 + A[1]/2),linewidth(3.0),Arrow3(10));
label("$f(x,y,z) = (x , x + y + z, 2y + z)$",(0,2.5,2.5),fontsize(8));
draw((0,2,2)..(0,3,2.1)..(0,4,2),Arrow3());
\end{asy} 
\begin{asy} 
import three;
import tube;
unitsize(1.5cm);
settings.outformat="pdf";
settings.render=8;
draw(O--3*X,Arrow3());
draw(O--3*Y,Arrow3());
draw(O--3*Z,Arrow3());
triple[] A =
  {
    (1,1,0),
    (0,1,2),
    (0,1,1)
  };
currentprojection=orthographic(4,2,1);
guide3 p = O--A[0]--A[0]+A[1]--A[1]--cycle;
draw(p);
draw(surface(p),0.5*white+0.5*blue+opacity(0.7),nolight);
guide3 p = O--A[1]--A[1]+A[2]--A[2]--cycle;
draw(p);
draw(surface(p),0.7*yellow+opacity(0.7),nolight);
guide3 p = O--A[2]--A[2]+A[0]--A[0]--cycle;
draw(p);
draw(surface(p),0.5*green+opacity(0.7),nolight);
guide3 p = A[2]--A[2]+A[0]--A[2]+A[0]+A[1]--A[2]+A[1]--cycle; 
draw(p); 
draw(surface(p),0.5*white+0.5*blue+opacity(0.7),nolight);
guide3 p = A[0]--A[0]+A[1]--A[0]+A[1]+A[2]--A[0]+A[2]--cycle;
draw(p);
draw(surface(p),0.7*yellow+opacity(0.7),nolight);
guide3 p = A[1]--A[1]+A[2]--A[1]+A[2]+A[0]--A[1]+A[0]--cycle;
draw(p);
draw(surface(p),0.5*green+opacity(0.7),nolight);
draw(shift(A[2])*(A[0]/3+A[1]/2 .. A[0]/2+A[1]/4  .. 2*A[0]/3 + A[1]/2),linewidth(3.0),Arrow3(10));
\end{asy} 
\captionof{figure}{A linear transformation from $\R^3$ to $\R^3$}
\end{center} 

The sign of the determinant depends on whether the orientation of a
small loop drawn on a face is reversed (as shown above), from the
point of view of a small person standing on the shape with their head
pointing toward the outside.

The formula for the determinant of a $3 \times 3$ matrix may be
derived analogously to the $2\times 2$ case. However, this derivation
is quite messy, and you will develop a more principled approach in a
linear algebra class. So let's skip straight to the formula:
\[
\det \left[\begin{array}{ccc} a & b & c \\ d & e & f \\ g & h & i \end{array}\right] = 
aei - afh - bdi + bfg + cdh - ceg. 
\]
Unlike the formula $ad - bc$ for the $2\times 2$ matrix, this formula is not easy to memorize. Let's abbreviate $\det [ \hphantom{a} ]$ to $| \hphantom{a}|$ and write this formula as 
\[
\left|\begin{array}{ccc} a & b & c \\ d & e & f \\ g & h & i \end{array}\right| = 
+a \left|\begin{array}{cc} e & f \\  h & i \end{array}\right| 
-b \left|\begin{array}{cc} d & f \\  g & i \end{array}\right| 
+c \left|\begin{array}{cc} d & e \\  g & h \end{array}\right|. 
\]
This formula is \textit{still} not easy to memorize, so let's break it down: each term on the right-hand side consists of a factor of $+1$ or $-1$ which alternates starting with $+1$, then an entry from the top row (going from left to right), then the determinant of the matrix you get when you remove the row and column of that entry from the original matrix. These smaller matrices are called \textit{minors}, and this method of calculating the determinant is called \textbf{expansion by minors} along the first row. You can also expand by minors along any row or column (see Exercise~\ref{exer:det3b} below), but if it's an even-numbered row or column, then the signs start with $-1$ instead of $+1$. 

\begin{exercise}{}{det3a} \setcounter{subitm}{1}
  Calculate each determinant. 

\pairofprobs{$\left|\begin{array}{ccc} 1 & 2 & 3 \\  0 & 4 & 5 \\ 0 & 0  & 6 \end{array}\right|$}
{$\left|\begin{array}{ccc} -4 & 2 & 1 \\ 5 & 0 & 3 \\ -2 & 1 & 3 \end{array}\right|$}
\end{exercise}

\begin{exercise}{}{det3b}
  Expand by minors along the first \textit{column} of this matrix, and show that you get the same result as when you expand by minors along the first row. 
  \[
    \left|\begin{array}{ccc} -2 & \hphantom{-}1 & \hphantom{-}4 \\  \hphantom{-}1 & \hphantom{-}1 & \hphantom{-}2 \\ \hphantom{-}2 & \hphantom{-}0  & -1 \end{array}\right|
  \]
\end{exercise}

\begin{exercise}{}{det3c}
  Find the values of $t$ for which the determinant of the following matrix is zero.  
  \[
    \left|\begin{array}{ccc} -2 & t^2 & 4 \\  3 & 1 & 0 \\ 2 & 0  & -1 \end{array}\right|
  \]
\end{exercise}


\chapter{Vectors}

\section{Introduction to vectors} \label{sec:vectors} 

\sidenote{\href{https://www.youtube.com/watch?v=fNk_zzaMoSs}{\tbob}
 on vectors}[-7.5mm]


\begin{wrapfigure}[11]{R}{4cm}
\begin{asy}
size(4cm);
real ep = 0.1;
draw((1,-ep)--(1,ep));
label("1",(1,0),align=2.5*S);
label("1",(0,1),align=2*W);
draw((-ep,1)--(ep,1));
draw((2,0)--(0,0)--(0,2),Arrows());
draw((1/2,1)--(2,2),MidnightBlue,Arrow());
\end{asy}
\caption{A vector in $\R^2$\label{fig:arrow}}
\end{wrapfigure}
A \textbf{vector} in $\R^n$ is an arrow from one point in $\R^n$ (the
\textit{tail}) to another (the \textit{head}). See
Figure~\ref{fig:arrow}). The \textbf{length}* of a vector is the
distance from the head to the tail. Two vectors are considered
equivalent if they have the same length and the same direction. 
\sidenote{* The length of a vector is sometimes called the
  \textbf{norm} or the \textbf{magnitude} of the vector}

The \textbf{components} of a vector are the coordinates of its head
when it is translated so that its tail is at the origin. In other words,
to find the components of a vector, we subtract each coordinate of its
tail from the correponding coordinate of the head. The components
of the vector in Figure~\ref{fig:arrow} are $\langle \frac{3}{2} ,
1\rangle$---note that we use the pointy brackets to distinguish
components of a vector from coordinates of a point. We can calculate
the components by subtracting the coordinates of the tail from the
coordinates of the head. Two vectors are
equivalent if and only if their components are the same.*
\sidenote{* Actually, vectors and points kind of \textit{are} the
  same, since they are both specified by an ordered tuple of real
  numbers. The distinction is in how we use them and visualize
  them.} 

The main things we will do with vectors are (i) add two of them
together and (ii) multiply a vector by a real number (which is called
a \textbf{scalar} in this context). These are
defined as follows:  
\begin{align*}
  \langle u_1, u_2 \rangle  +   \langle v_1, v_2 \rangle &= 
          \langle u_1  + v_1, u_2 + v_2\rangle \\
  c \langle u_1, u_2 \rangle &= \langle cu_1, cu_2 \rangle. 
\end{align*}
These natural definitions of vector addition and scalar multiplication
lead to natural geometric interpretations, as shown in
Figures~\ref{fig:vectoradd} and \ref{fig:scalarmultiply}. 

\begin{center} 
\begin{minipage}{0.45\textwidth}
\begin{center} 
\begin{asy}
size(4cm);
real n = 4;
real ep = 0.1*n;
for(int i=1;i<=n;++i){
  draw((i,0)--(i,n),gray);
  draw((0,i)--(n,i),gray);
}
draw((0,n+ep)--(0,0)--(n+ep,0),Arrows());
draw((0,0)--(3,1),MidnightBlue,Arrow());
draw((0,0)--0.92*(3,1),MidnightBlue+linewidth(1.5));
draw((3,1)--(4,3),DarkOrange,Arrow());
draw((3,1)--(4,3) - 0.08*(1,2),DarkOrange+linewidth(1.5));
draw((0,0)--(4,3),SeaGreen,Arrow());
draw((0,0)--0.95* (4,3),SeaGreen+linewidth(1.5));
\end{asy}
\end{center}
\captionof{figure}{Vector addition: $\langle 3,1 \rangle + \langle 1,2
  \rangle = \langle 4,3 \rangle$\label{fig:vectoradd}}
\end{minipage}
\begin{minipage}{0.45\textwidth}
\begin{center} 
\begin{asy}
size(4cm);
real n = 4;
real ep = 0.1*n;
for(int i=1;i<=n;++i){
  draw((i,0)--(i,n),gray);
  draw((0,i)--(n,i),gray);
}
draw((0,n+ep)--(0,0)--(n+ep,0),Arrows());
draw((0,0)--1.9*(2,1),MidnightBlue+linewidth(2.5));
draw((0,0)--(4,2),MidnightBlue,Arrow());
draw((0,0)--(2,1),DarkOrange,Arrow());
draw((0,0)--0.93*(2,1),DarkOrange+linewidth(1.25));
\end{asy}
\end{center}
\captionof{figure}{Scalar multiplication: $2\langle 2,1 \rangle = \langle 4,2 \rangle$\label{fig:scalarmultiply}}
\end{minipage}
\end{center}

We typically assign names for vectors which are lowercase boldface
letters, like $\vec{u}$ or $\vec{v}$.  Looking at 
Figure~\ref{fig:scalarmultiply}, we make the following observation

\begin{obs}{Parallel vectors}{vecparallel} \bang{-5mm}
  Two vectors $\vec{u}$ and $\vec{v}$ have the
same direction if $\vec{u} = c \, \vec{v}$ for some scalar $c$. 
\end{obs}

The following exercise shows that vector operations satisfy one of the
properties suggested by the notation.* They satisfy several others,
such as commutativity
($\vec{u} + \vec{v} = \vec{v} + \vec{u}$), associativity
($(\vec{u} + \vec{v}) + \vec{w} = \vec{u} + (\vec{v} +
\vec{w})$), and so on. The basic strategy for proving all these
property-verification exercises is the same: write out what each side
of the equations in terms of components, and then simplify until you
can see that both sides are equal. 
\sidenote{* That is, we use the same notation as we use for multiplication/addition
  of real numbers because these operations satisfy many of the same
  properties.}[-12mm]

\begin{exercise}{}{vector_distribution}
  Show that scalar multiplication distributes across vector
  addition. In other words, show that for all $c\in \R$ and vectors
  $\vec{u}$ and $\vec{v}$ in $\R^n$, we have 
  \[
    c(\vec{u} + \vec{v})= c
    \vec{u} + c \vec{v}. 
  \] 
\end{exercise}

\begin{exercise}{}{vector_subtraction}
  Choose two vectors $\vec{u}$ and $\vec{v}$ with small integer
  coordinates and verify that $\vec{u}$, $\vec{v}$, and $\vec{u}
  - \vec{v}$
  fit together to form a triangle. 
\end{exercise}

The following example shows how vector ideas can be applied to geometry
problems. 

\begin{example}{}{trianglemidpoint}
  Use vectors to prove that the line joining the midpoints of two
  sides of a triangle is parallel to the third side and half its
  length.
\end{example}

\begin{solution} 
\begin{minipage}{11cm}
  Define $\vec{u}$ and $\vec{v}$ be two vectors with a common
  tail at one vertex $O$ of the triangle and heads at the other two
  vertices $A$ and $B$ as shown. Then the vectors from $O$ to the
  midpoints of $OA$ and $OB$ are $\tfrac{1}{2}\vec{u}$ and
  $\tfrac{1}{2}\vec{v}$, since the midpoint of a line segment is
  defined to be the point which is halfway between the endpoints.

  Therefore, the vector $\vec{w}$ from one midpoint to another is
  $\tfrac{1}{2} \vec{v} - \tfrac{1}{2}\vec{u}$. By the
  distributive property, this is equal to
  $\tfrac{1}{2}(\vec{v} - \vec{u})$. The vector from $A$ to $B$
  is $\vec{v} - \vec{u}$.  Therefore, $\vec{w}$ has the same
  direction as the vector from $A$ to $B$ (by
  Observation~\ref{obs:vecparallel}) and is half as long.
\end{minipage} \quad 
\begin{minipage}{5cm}
\begin{asy} 
size(5cm);
label("$O$",(0,0),align=W);
label("$A$",(2,1),align=E);
label("$B$",(1,3),align=N);
draw((0,0)--(2,1)--(1,3)--cycle);
draw((0,0)--(2,1),Arrow());
draw((0,0)--(1,3),Arrow());
draw((1,1/2)--(1/2,3/2),Arrow());
label("$\mathbf{u}$",(1,1/2),align=SE);
label("$\mathbf{v}$",(1/2,3/2),align=WNW);
\end{asy}
\end{minipage}
\end{solution}

\begin{exercise}{}{}
  Use vectors to show that the diagonals of a parallelogram bisect one
  another. 
\end{exercise}

\section{The dot product} \label{sec:dot} 

\sidenote{\href{https://www.youtube.com/watch?v=LyGKycYT2v0}{\tbob}
 on the dot product}[-7.5mm]

The fundamental vector operations of scalar multiplication and vector
addition are not sufficient to capture information about a really important
geometric concept: \textit{angle}. So we introduce a new vector
operation.

\begin{defn}{Dot product}{dotproduct}
The \textbf{dot product} of two three-dimensional vectors $\vec{u}
= \langle u_1, u_2, u_3 \rangle$
and $\vec{v} =  \langle v_1, v_2, v_3 \rangle$ is defined by 
\[
\vec{u} \cdot \vec{v} = \langle u_1, u_2, u_3 \rangle \cdot \langle v_1, v_2, v_3 \rangle =
u_1 v_1 + u_2  v_2+ u_3v_3. 
\]
\end{defn}

The dot product distributes across vector addition, and it is closely
related to length, as shown in the following exercise. We denote by
$|\vec{u}|$ the length of $\vec{u}$. 

\begin{exercise}{}{dotproduct}
  Verify that $\vec{u} \cdot (\vec{v} + \vec{w}) = \vec{u} \cdot
  \vec{v} + \vec{u} \cdot \vec{w}$ and that $\vec{u} \cdot
  \vec{u} = |\vec{u}|^2$.
\end{exercise}

Now we establish the relationship between the dot product and angle. 

\begin{example}{}{}
  Use the law of cosines to show that $\vec{u} \cdot \vec{v} =
  |\vec{u}| |\vec{v}| \cos\theta$, where $\theta$ is the angle
  between $\vec{u}$ and $\vec{v}$. 
\end{example}

\begin{solution} 
  \begin{minipage}{11cm}
    We apply the law of cosines to the triangle with sides
    $\vec{u}$, $\vec{v}$, and $\vec{u} - \vec{v}$. We get 
    \[
      |\vec{u} - \vec{v}|^2 =  |\vec{u}|^2 +  |\vec{v}|^2  -2|\vec{u}|
      |\vec{v}|\cos\theta
    \]
    The left-hand side works out to 
    \[
       |\vec{u} - \vec{v}|^2 = 
       (\vec{u} - \vec{v}) \cdot 
       (\vec{u} - \vec{v}) = 
       |\vec{u}|^2 + |\vec{v}|^2 - 2\, \vec{u} \cdot
       \vec{v}. 
    \]
    Subtracting these equations yields $\vec{u} \cdot \vec{v} =
  |\vec{u}| |\vec{v}| \cos\theta$. 
  \end{minipage} \quad 
\begin{minipage}{5cm}
\begin{asy} 
size(5cm);
real ep = 0.2;
draw((0,0)--(2,1)--(1,3)--cycle);
draw((0,0)--(2,1),Arrow());
draw((0,0)--(1,3),Arrow());
draw((1,3)--(2,1),Arrow());
label("$\theta$",(0.3,0.38));
label("$\mathbf{u}$",(1,1/2),align=SE);
label("$\mathbf{v}$",(1/2,3/2),align=WNW);
label("$\mathbf{u} - \mathbf{v}$",(3/2,2),align=ENE);
\end{asy}
\end{minipage}
\end{solution}

Particularly noteworthy is the case where $\theta$ is a right angle:
We say that two vectors are \textbf{perpendicular} or
\textbf{orthogonal} or  \textbf{normal} if they
meet at a right angle. 

\begin{obs}{Perpendicular vectors}{} \bang{-5mm}
  Two vectors $\vec{u}$ and $\vec{v}$ are perpendicular if and
  only if $\vec{u} \cdot \vec{v} = 0$. 
\end{obs}

The following example shows how handy the relation $\vec{u} \cdot \vec{v} =
  |\vec{u}| |\vec{v}| \cos\theta$ can be. 
  
  \begin{example}{}{}
    Find the angle between the diagonal of a cube and a diagonal of one
    of its faces.
    \begin{center}
      \begin{asy}[width=2cm]
        import three;
        currentlight.background = softblue; 
        defaultpen(linewidth(0.8));
        size(5cm);
        currentprojection=orthographic(4,2,1.2);
        draw(box((0,0,0),(1,1,1)));
        draw((0,0,0)--(1,1,0));
        draw((0,0,0)--(1,1,1));
      \end{asy}
    \end{center}
  \end{example}
  
\begin{solution} 
  The vector from the origin to the opposite corner of the cube is $\langle
  1,1,1\rangle$. The vector from the origin to the opposite corner of
  the bottom face is $\langle 1,1,0 \rangle$. Therefore, the angle is given by 
  \[
    \theta = \cos^{-1}\left(\frac{\mathbf{u} \cdot
        \mathbf{v}}{|\mathbf{u}||\mathbf{v}|}\right) = \cos^{-1}\left(
      \frac{1 + 1 + 0}{\sqrt{1^2 + 1^2 + 1^2}\sqrt{1^2+1^2+0^2}}
    \right) = \boxed{\cos^{-1}\left(\frac{2}{\sqrt{6}}\right)}. 
  \]
\end{solution}

\begin{exercise}{}{}
  Sketch the vectors $\mathbf{u} = \langle 4,2 \rangle$ and
  $\mathbf{v} = \langle -1, 2\rangle$ and show geometrically that they
  are perpendicular. 

  Then verify that the coordinate formula for dot product indeed gives 
  $\mathbf{u} \cdot \mathbf{v} = 0$ for these two vectors. 
\end{exercise}

We conclude this section by pointing out that one can begin with the
geometric formula
$\mathbf{u} \cdot \mathbf{v} = |\mathbf{u}| |\mathbf{v}| \cos\theta$
as the \textit{definition} of the dot product, derive the distributive
property of the dot product across vector addition, and then obtain
the formula for the dot product in the following simple manner: define
$\mathbf{i} = \langle1,0,0\rangle$,
$\mathbf{j} = \langle0,1,0\rangle$, and
$\mathbf{k} = \langle0,0,1\rangle$. Then a vector
$\mathbf{u} = \langle u_1, u_2, u_3\rangle$ can be written as
\[
\mathbf{u} = u_1 \mathbf{i} + u_2 \mathbf{j} + u_3 \mathbf{k},
\]
and similarly for $\mathbf{v}$. Then 
\begin{align*}
  \mathbf{u} \cdot \mathbf{v} &= (u_1 \mathbf{i} + u_2 \mathbf{j} + u_3
  \mathbf{k})(v_1 \mathbf{i} + v_2 \mathbf{j} + v_3 \mathbf{k})  \\
  &= u_1 v_1 \mathbf{i} \cdot \mathbf{i} + u_1 v_2 \mathbf{i} \cdot
    \mathbf{j}  + u_1 v_3 \mathbf{i} \cdot \mathbf{k}  + \\ 
  &\hspace{5mm}  u_2 v_1 \mathbf{j} \cdot \mathbf{i} + u_2 v_2 \mathbf{j} \cdot
    \mathbf{j}  + u_2 v_3 \mathbf{j} \cdot \mathbf{k}  + \\ 
  &\hspace{5mm}  u_3 v_1 \mathbf{k} \cdot \mathbf{i} + u_3 v_2 \mathbf{k} \cdot
    \mathbf{j}  + u_3 v_3 \mathbf{k} \cdot \mathbf{k},
\end{align*}
by the distributive property. This looks like a mess, but since
$\mathbf{i}$,  $\mathbf{j}$, and  $\mathbf{k}$ are 
perpendicular, six of these nine terms are zero. Furthermore, since
$\mathbf{i}\cdot \mathbf{i} = 1$ and similarly for $\mathbf{j}$ and
$\mathbf{k}$, we end up with $  \mathbf{u} \cdot \mathbf{v}  = u_1 v_1
+ u_2 v_2 + u_3 v_3$, as desired. 

\section{The cross product} \label{sec:cross} 

\sidenote{\href{https://www.youtube.com/watch?v=eu6i7WJeinw}{\tbob}
 on the cross product}[-7.5mm]

The last section introduced a vector product which reveals information
about \textit{angle}; in this section we'll see a new vector product
which gives us information about \textit{area}.

\begin{wrapfigure}[9]{R}{4cm}
  \begin{asy}[width=4cm]
    import three;
    import tube;
    unitsize(1.5cm);
    currentprojection=perspective(7.46, 2.9, 0.72); 

    draw(O--3*X,Arrow3());
    draw(O--3*Y,Arrow3());
    draw(O--3*Z,Arrow3());
    
    triple u = (1,0.75,-0.5);
    triple v = (0,2,1);
    triple w = cross(u,v) ;
    
    real eps = 0.05;
    
    draw(eps*w -- eps*w + eps*(v+u) -- eps*(v+u));
    
    draw(O--u,Arrow3());
    draw(O--v,Arrow3());
    draw(O--w,Arrow3());
    
    label("$\mathbf{u}$",0.5*u,align=SW);
    label("$\mathbf{v}$",0.5*v,align=N);
    label("$\mathbf{u}\times \mathbf{v}$",1.1*w,align=E); 
    
    draw(surface(O--u--u+v--v--cycle),LightSeaGreen+opacity(0.4));
  \end{asy}
  \caption{The cross product of $\mathbf{u}$ and $\mathbf{v}$, whose
    length is equal to the area of the parallelogram shown \label{fig:crossprod}}
\end{wrapfigure}

The \textbf{cross product} of $\mathbf{u} = (u_1,u_2,u_3)$ and
$\mathbf{v} = (v_1,v_2,v_3)$ is defined by expanding the following
`determinant' by minors along the first row:* \sidenote{* We put
  \textit{determinant} in scare quotes because the matrix entries are
  not numbers.} \sidenote{\href{https://www.youtube.com/watch?v=BaM7OCEm3G0}{\tbob} A beautiful explanation
  of this formula}[20mm]
\[
  \mathbf{u} \times \mathbf{v} = 
\left|
\begin{array}{ccc}
\mathbf{i} & \mathbf{j} & \mathbf{k} \\
u_1 & u_2 & u_3 \\
v_1 & v_2 & v_3
\end{array}
\right| =
(u_2v_3 - u_3 v_2) \mathbf{i}
- (u_1 v_3 - u_3 v_1) \mathbf{j}
 + (u_1v_2 - u_2 v_1) \mathbf{k}. 
\]

Note that the dot product of two vectors is a scalar, while the cross
product of two vectors is another vector. It turns out that this
vector is orthogonal to \textit{both} of the first two. 

\begin{example}{}{}
  Confirm that $\mathbf{u} \times \mathbf{v}$ is orthogonal to
  $\mathbf{u}$ and $\mathbf{v}$. 
\end{example}

\begin{solution}
  We compute 
  \begin{align*}
   \langle u_2v_3 - u_3 v_2, 
    - (u_1 v_3 - u_3 v_1), 
    u_1v_2 - u_2 v_1 \rangle  \cdot \langle u_1, u_2, u_3 \rangle &= \\
    ( u_2v_3 - u_3 v_2)u_1 &- (u_1 v_3 - u_3 v_1) u_2 +
    ( u_1v_2 - u_2 v_1) u_3   = 0. 
  \end{align*}
  This implies that $\mathbf{u} \times \mathbf{v}$ is orthogonal to
  $\mathbf{u}$. Swapping out $\mathbf{u}$ for $\mathbf{v}$, we see
  that $\mathbf{u} \times \mathbf{v}$ is orthogonal to
  $\mathbf{v}$ too. 
\end{solution}

\begin{lrbox}{\asybox} 
  \begin{asy}[width=1.5cm]
  import olympiad;
  defaultpen(fontsize(4)+linewidth(0.5)); 
  size(5cm);
  pair A = (1.5,0); pair B = (1,0.85); pair O = (0,0); 
  draw(O--A--A+B--B--cycle);
  real eps = 0.1;
  draw((1-eps,0)--(1-eps,eps)--(1,eps)); 
  draw(B--O--A,linewidth(0.8)); 
  label("$a$",A/2, align=S);
  label("$b$",B/2, align=NW);
  label(rotate(-90)*"$b\sin\theta$",(1,B.y/2),align=E); 
  draw(B--(1,0));
  draw(anglemark(A,O,B,6));
  label("$\theta$",(0.45,0.16));
\end{asy}
\end{lrbox}

The following exercise provides the advertised connection to
area. Recall from geometry that the area of a parallelogram with sides
of length $a$ and $b$ meeting at an angle $\theta$ is equal to $ab\sin
\theta$.*
\sidenote{* Proof: \usebox{\asybox}}

\begin{exercise}{}{}
Verify that $|\mathbf{u} \times \mathbf{v}|^2 =
|\mathbf{u}|^2|\mathbf{v}|^2 - (\mathbf{u}\cdot \mathbf{v})^2$. Use this fact to
show that 
\[
|\mathbf{u} \times \mathbf{v}| = |\mathbf{u}||\mathbf{v}|\sin\theta,
\]
where $\theta$ is the angle between $\mathbf{u}$ and $\mathbf{v}$. 
\end{exercise}

So to sum up: $\mathbf{u}  \times \mathbf{v}$ is a vector which is
orthogonal to both $\mathbf{u}$ and $\mathbf{v}$ and whose length
is equal to the area of the parallelogram spanned by $\mathbf{u}$ and
$\mathbf{v}$. Note that there are only two vectors satisfying both of
these conditions. To determine which one is $\mathbf{u}  \times
\mathbf{v}$, we use the \textit{right-hand rule}: imagine orienting
your right hand so that you can curl your fingers from $\mathbf{u}$
towards $\mathbf{v}$. The direction of your thumb is the direction of
$\mathbf{u} \times \mathbf{v}$. 

\begin{exercise}{}{}
  Find the volume of the parallelepiped spanned by
  $\langle 3,4,1 \rangle$,  $\langle -2,4,0 \rangle$, and 
  $\langle -5,5,2 \rangle$. (Hint: first find the area of the base,
  then figure out how to use a   dot product to multiply by the height.)
\end{exercise}

\chapter{Three-dimensional Geometry}

\section{Lines and planes} \label{sec:lines_and_planes} 
  
There are various ways to describe a line in 2D space using an
equation, including point-slope form and $y$-intercept form. In this
section we will learn the 3D analogue: equation descriptions of lines
and planes in space. We begin with an example.

\begin{example}{}{line}
  Describe the line $L$ in $\R^3$ passing through the points $A = (3,-4,1)$ and
$B = (2,-1,4)$. 
\end{example}

\begin{solution}
  We can tell whether a given point $(x,y,z)$ in $\R^3$ is on the line
  $L$ using vectors: $(x,y,z)$ is on $L$ if and only if the vector
  from $(3,-4,1)$ to $(x,y,z)$ is a scalar multiple of the vector from
  $(3,-4,1)$ to $(2,1,-4)$ (see Figure~\ref{fig:linecheck}). We can
  turn this into an equation: a point $(x,y,z)$ is on $L$ if and only
  if there exists $t\in \R$ such that
  \[
    t \big\langle 2-3, 1-(-4), -4-1\big\rangle = \big\langle x - 3, y -(-4), z -
    1\big\rangle. 
  \]
  Setting components equal, we find that $(x,y,z)$ is on $L$ if and
  only if there exists $t$ so that 
  \begin{align} \nonumber 
    x &= 3 - t, \\ \label{eq:par}
    y &= -4 +5t, \text{ and} \\  \nonumber 
    z &= 1 - 5t.
  \end{align}
\end{solution}

\begin{wrapfigure}[14]{R}{6cm}
  \begin{asy}
    import tube;
    unitsize(0.5cm);
    
    currentprojection=perspective(13.1,12.4,4.1); 
    
    draw(O--6*X,Arrow3());
    draw(O--6*Y,Arrow3());
    draw(O--6*Z,Arrow3());
    
    triple u = (3,-4,1); 
    triple v = (2,-1,4);
    triple z = (2.3,-2,1.5); 
    dot(u); dot(v); dot(z); 
    
    draw(u + -1*(v-u)--v + (v-u),Arrows3());
    draw(u--z,linewidth(1.2),Arrow3(10));
    draw(u--v,linewidth(1.2),Arrow3(10));
    
    label("$(x,y,z)$",z,align=E); 
    label("$A$",u,align=NW);
    label("$B$",v,align=NW);
  \end{asy}
    \caption{Checking whether $(x,y,z)$ is on the line through $A$ and
      $B$ \label{fig:linecheck}}
  \end{wrapfigure}

Note that the solution above involves a new variable $t$; this is
called a \textit{parameter}, and the form we gave as an answer is
called \textbf{parametric form}. You can imagine drawing the line by
starting with $t = 0$, so that your pen begins at $A$, and then
sweeping $t$ through the values from 0 to 1, changing the location
of your pen according to the parametric equations \eqref{eq:par}. This
gives you the portion of the line between $A$ and $B$. Then you can
let $t$ vary beyond 1 to get the rest of the ray past $B$, and you can
let $t$ vary over the negative numbers to get the part of the line on
the other side of $A$.

If we didn't want to involve $t$, note that we could solve for $t$ in
one equation and substitute into the other two, thereby obtaining
\textit{two} equations involving $x$, $y$, and $z$. This makes sense:
starting from the plane, imposing one equation on $x$ and $y$ cuts the
dimension down by one and gives a line. However, starting from 3D
space, we need to reduce the dimension by \textit{two}. So we need
two equations.

The procedure developed in Example~\ref{exam:line} works in general:
the line through $A = (a,b,c)$ and $B$ has parametric form
\begin{align*}
x &= a + v_1 t, \\
y &= b + v_2 t \\ 
z &= c + v_3 t, 
\end{align*}
where $\langle v_1, v_2, v_3 \rangle = \overrightarrow{AB}$ is the vector from $A$ to
$B$.* \sidenote{* Note that this representation is not unique, since we
  could've used $B$ (or any other point on the line) in place of $A$.}

\begin{example}{}{planeequation}
  Describe the plane $P$ passing through the points $A = (1,0,0)$, $B =
  (0,1,1)$, and $C = (0,0,2)$.
\end{example}

\begin{solution}
  We can tell whether $(x,y,z)$ is on $P$ using vectors. Define
  $\mathbf{u}$ and $\mathbf{v}$ to be the vectors from $A$ to $B$ and
  from $A$ to $C$, respectively. 
  \begin{center}
    \begin{asy}
      unitsize(1.5cm);
      import three;
      currentlight.background = softyellow; 
      
      draw(O--3*X,Arrow3(5));
      draw(O--3*Y,Arrow3(5));
      draw(O--3*Z,Arrow3(5));
      
      triple A = (1,0,0);
      triple B = (0,1,1);
      triple C = (0,0,2);
      triple D = (1,-1,2); 
      
      dot("$A$",A,align=SE);
      dot("$B$",B);
      dot("$C$",C);
      
      draw("$\mathbf{n}$",A--A+cross(B-A,C-A),Arrow3(),align=SW);
      real eps = 0.1;
      
      draw(A + eps*cross(B-A,C-A) --
      A + eps*cross(B-A,C-A) + eps*(A-C) --
      A + eps*(A-C));
      
      draw(A--D,Arrow3());
      dot("$(x,y,z)$", D,align=NW); 
      
      draw(surface(A+C-B+0.5*(A-B)--
      A+B-C+0.5*(A-B)--
      B+B-A--
      C+C-A--
      cycle),LightSeaGreen+opacity(0.5));
      draw("$\mathbf{u}$",A--B,Arrow3(),align=SE);
      draw("$\mathbf{v}$",A--C,Arrow3(),align=W);
    \end{asy}
  \end{center}
  If we can find a vector $\mathbf{n}$ which is orthogonal to $P$,
  then we can say $(x,y,z)$ is on $P$ if and only if the vector from
  $A$ to $(x,y,z)$ is orthogonal to $\mathbf{n}$. But we can take
  $\mathbf{n} = \mathbf{u} \times \mathbf{v}$, since the cross product
  of two vectors is orthogonal to both of them. So
\[
  \mathbf{n} = \langle -1,0,2\rangle \times \langle 0, -1, 1 \rangle =
  \langle 2, 1, 1 \rangle. 
\]
Now we can say that $(x,y,z)$ is on $P$ if and only if
\[
  \mathbf{n} \cdot \langle x - 1 , y - 0 , z - 0 \rangle = 0, 
\]
which simplifies to $\boxed{2x + y + z = 2}$. 
\end{solution}

\begin{obs}{Vector normal to a plane}{} \bang{-3mm}
  A vector $\mathbf{n}$ normal to the plane $ax +by + cz = d$
  can be read off from the coefficients:
  \[
    \mathbf{n} = \langle a,b,c \rangle. 
  \]
\end{obs}

One important 3D geometry problem is to find distances between points,
lines, and planes.* We define the distance between two sets to be the
\textit{minimum} distance between two points in those
sets. \sidenote{* We can ask about the distance from a point to a
  point, a point to a line, a point to a plane, a line to a line, a
  line to a plane, or a plane to a plane.}

\begin{example}{}{}
  Consider the line $\ell$ given by the parametric equation
  $(x,y,z) = (1-2t,3,t)$. Find the distance from $\ell$ to the line
  $m$ which is parallel to $\ell$ and which passes through the point
  $(9,4,1)$.
\end{example}

\begin{solution}
  \begin{minipage}[b]{0.65\textwidth}
    The parametric equations give us convenient access to
    a point on each of the two lines as well as a vector $\mathbf{v}$
    which is parallel to both lines. So we make a figure with this
    information.
    
    If we define $\mathbf{u}$ to be the vector from connecting the two
    given points, we can see by applying right-triangle trigonometry to
    the figure that the desired distance $d$ is equal to
    $|\mathbf{u}| \sin \theta$. Therefore,
    \[
      d  = \frac{|\mathbf{u}||\mathbf{v}|\sin \theta}{|\mathbf{v}|} = \frac{|\mathbf{u} \times \mathbf{v}|}{|\mathbf{v}|} =
      \frac{\sqrt{105}}{\sqrt{5}} = \boxed{\sqrt{21}}. 
    \]
  \end{minipage} \: 
  \begin{minipage}[b]{0.32\textwidth}
    \begin{asy}[width=5cm]
      import graph;
      defaultpen(fontsize(8)); 
      
      real f(real x){
        return x*x;
      }
      
      pair A = (0,0);
      pair B = (2,1);
      pair C = (-1,1);
      
      draw(A--B,Arrows());
      draw(A+C--B+C,Arrows());
      
      dot("$(1,3,0)$",A + 0.2*(B-A),align=2*SSE);
      dot("$(9,4,1)$",C + 0.8*(B-A),align=1.2*WNW); 
      
      draw(A + 0.6*(B-A) -- A + 0.6*(B-A) + 0.6*(-1,2));
      
      draw("$\mathbf{u}$",A + 0.2*(B-A) -- C + 0.8*(B-A),linewidth(1.0),Arrow(6));
      draw("$\mathbf{v}$",A + 0.2*(B-A) -- A + 0.4*(B-A),linewidth(1.0),Arrow(6),align=3.7*NE);
      
      label("$\theta$",(A + 0.2*B-A), align=NNE + NE); 
      
      label(rotate(-63)*"$|\mathbf{u}|\sin
      \theta$",(1.13,1.41),align=SW);
    \end{asy}
\end{minipage}
\end{solution}

Note the basic strategy: (i) draw a figure containing the information that
the problem gives us (a schematic diagram suffices; there is no need
to make it particularly precise), (ii) use right triangle trigonometry to
express the desired distance terms of vectors we have, (iii) use
vector formulas to calculate the desired quantity using a dot or cross
product. 

\begin{exercise}{}{}
  Find the equation of the plane passing through the points $(1,0,0)$,
  $(0,1,0)$, and $(0,0,1)$. Find the distance from that plane to the
  origin. 
\end{exercise}

\begin{exercise}{}{}
  Find the distance between the planes $x+y-2z = 3$ and $x+y-2z = 0$. 
\end{exercise}

\begin{exercise}{}{}
  Find the distance between the lines $(x,y,z) = (2t, 1-t, 4)$ and
  $(x,y,z) = (1 + t, -2t, -1-t)$. Hint: these lines are \textit{skew},
  meaning that they are not parallel but do not intersect. Begin by
  using a cross product to find a vector which is perpendicular to
  both lines. 
\end{exercise}

\section{Motion in space} \label{sec:motion_in_space} 

Consider a particle moving along the number line in such a way that
its position at time $t$ is given by $r(t)$. Then the velocity of the
particle at time $t$ is given by the first derivative $v(t) =
r'(t)$. The velocity specifies the \textit{speed} of the particle as well as
its \textit{direction} (left if negative, right if positive).

The same is true of a particle moving in 2D or 3D space: its location
is specified by a function customarily denoted $\mathbf{r}(t)$ from
$\R$ to either $\R^2$ or $\R^3$, and its derivative*
$\mathbf{v}(t) = \mathbf{r}'(t)$ at time $t$ tells us the speed of the
particle at that time (via its length) as well as the direction. We
call $\mathbf{r}$ a \textit{path}*. \sidenote{* The derivative of a
  path $\mathbf{r}$ is the vector obtained by taking the derivative of
  each component. }[-3cm] \sidenote{* For a function from $\R^1$ to
  $\R^2$ or $\R^3$ to count as a path, we require that each of its
  components be continuous. For example, $\mathbf{r}(t) = (e^t, t^2 \sin
  t)$ is a path. }


\begin{example}{}{can}
  Consider a bug which crawls counterclockwise around a cylinder of radius 1 from
  $(1,0,0)$ to $(1,0,1)$ as shown: 
  \begin{center}
    \begin{asy}
      import graph3; 
      size(4cm);
      
      currentprojection=perspective(5.7,3.1,1.26); 
      currentlight.background = softblue;
      
      draw(O--1.4*X,Arrow3());
      draw(O--1.4*Y,Arrow3());
      draw(O--1.2*Z,Arrow3());
      
      triple f(real t){
        return (cos(2*pi*t), sin(2*pi*t), t);
      }

      path3 g = graph(f,0,1);
      
      draw(unitcylinder, surfacepen = material(diffusepen=0.4*LightSeaGreen+0.6*black,
                                                                       emissivepen=0.4*LightSeaGreen+0.6*black)); 
      draw(g,DarkRed+linewidth(1.5)); 

      real t = 0.1;
      draw(f(t) -- f(t) + 0.10 * (-2*pi*sin(2*pi*t), (2*pi)*cos(2*pi*t), 1),DarkBlue+linewidth(1.2),Arrow3(5));
      label("$\mathbf{v}(t)$", f(t+0.065) + 0.2*(cos(0.1),sin(0.1),1) + 0.00 * Z,DarkBlue); 
      dot(f(0.11),DarkBlue); 
      label("$\mathbf{r}(t)$",(0,1.3,0.4),DarkRed); 
    \end{asy}
  \end{center}
  Assuming the bug moves at constant speed and makes the whole journey
  in one second, find a formula for the position and velocity of the
  bug at time $t$. 
\end{example}

\begin{solution}
  We can see that the $z$-coordinate of the bug's position increases
  at a constant rate from 0 to 1 as $t$ goes from 0 to 1, so the
  $z$-coordinate of $\mathbf{r}(t)$ is $t$.

  For the $x$ and $y$ coordinates, we need a pair of functions
  $(x(t),y(t))$ that traces out the unit circle in one second. Recall
  that cosine and sine are defined to be the functions that trace out
  the unit circle according to angle, so we can scale them so they
  make it around in 1 second instead of $2\pi$ seconds:
  \[
    (x(t),y(t)) = \left( \cos 2\pi t, \sin 2\pi t
    \right). 
  \]
  So all together we have
  \[
    \mathbf{r}(t) =  \left( \cos 2\pi t, \sin 2\pi t
      , t \right),
  \]
  which means that
  \[
    \mathbf{v}(t) = \left( -2\pi \sin 2\pi t, 2\pi\cos 2\pi t
      , 1 \right). 
  \]
\end{solution}

Similarly, the acceleration $\mathbf{a}(t)$ of a particle whose
position at time $t$ is given by $\mathbf{r}(t)$ is defined to be
$\mathbf{v}'(t) = \mathbf{r}''(t)$. 

\begin{exercise}{}{}
  An astronaut is using a rope to move in space in such a way that his
  position at time $t$ is given by 
  $\mathbf{r}(t) = (2+t) \mathbf{i} + (2+\ln t) \mathbf{j} + \left( 7
    - \frac{4}{t^2+1}\right) \mathbf{k}$. The coordinates of the space
  station doorway are $(5,4,9)$. When should the astronaut let go of the
  rope so as to drift into the doorway? 
\end{exercise}

\section{Quadric surfaces} \label{sec:quadric_surfaces} 

The \textit{graph} of an equation involving the variables $x,y,z$ is
the set of points $(x,y,z)$ in $\R^3$ which satisfy the equation. For
example, we have seen that the graph of $x + y + z = 1$ is a plane in
$\R^3$. More generally, the graph of any linear equation in $\R^3$ is
a plane. So let's step it up a notch and consider \textit{quadratic}
equations.* \sidenote{* The 2D analogues are \textit{conic sections}:
  parabolas, ellipses, and hyperbolas. These are graphs of various
  quadratic equations in two variables.} A graph of a quadratic
equation in the variables $x,y,z$ is called a \textit{quadric
  surface}. 

Perhaps the simplest quadratic equation to reason about is $x^2 + y^2 + z^2 =
1$. The left-hand side has a geometric interpretation as \textit{the
  squared distance from $(x,y,z)$ to the origin}. Therefore, a point $(x,y,z)$
satisfies this equation if and only if its squared distance to the
origin is 1. We have a name for the set of such points: the
\textit{sphere} of radius 1, centered at the origin.

The situation is not always so simple. So here's a key idea for
tackling 3D geometry problems: \textbf{slice it up}. Consider planes
of the form $z = \mathrm{constant}$, $y = \mathrm{constant}$, or
$z = \mathrm{constant}$ and see what your graph looks like in these
planes. Here's an archetypal example.

\begin{example}{}{}
  Figure out what the graph of $x^2 + y^2 - z^2 = 1$ looks like. 
\end{example}

\begin{solution}
  \begin{minipage}{0.7\textwidth} 
  We begin by finding all the points which satisfy this equation and
  the equation $z = 0$. If $(x,y,z)$ satisfies this equation and $z
  =0$, then that means that $x^2 + y^2 = 1$. Furthermore, if  $x^2 +
  y^2 = 1$ and $z = 0$, then $(x,y,z)$ satisfies the equation $x^2 +
  y^2 - z^2 = 1$. This means that the intersection of the desired
  graph and the line $z = 0$ is the circle of radius 1 centered at the
  origin.
\end{minipage}
\begin{minipage}{0.29\textwidth}
  \begin{asy} 
    import graph3;
    currentlight.background = softyellow; 
    size(4cm);
    
    draw(O--4*X,Arrow3(4));
    draw(O--4*Y,Arrow3(4));
    draw(O--2.5*Z,Arrow3(4));
    
    for(real z=-2; z<= 2; z += 0.2){
      draw(circle(c=(0,0,z),r=sqrt(1+z^2),normal=Z));
    }
  \end{asy}
\end{minipage}

Similarly, the intersection of the desired graph and the plane
  $z = 1$ is circle centered at $(0,0,1)$ and having radius
  $\sqrt{2}$. Drawing in several more of these traces*, we get a
  picture that looks like this: \sidenote{* A \textit{trace} of a figure
    is an intersection of that figure with a plane}
This is already a pretty clear picture of what the graph looks like:
it's rotationally symmetric about the $z$-axis and ``flares out'' as
you move away from the $xy$-plane. This graph is called a
\textit{one-sheeted hyperboloid}. 
\end{solution}

\begin{exercise}{}{}
  Sketch $\frac{z}{c} = \frac{x^2}{a^2} + \frac{y^2}{b^2}$, where
  $a=b=c=1$. This is called an elliptic paraboloid.
\end{exercise}

\begin{exercise}{}{}
  Sketch $\frac{z^2}{c^2} = \frac{x^2}{a^2} + \frac{y^2}{b^2}$, where
  $a=b=c=1$. This is called an elliptic cone.
\end{exercise}

\begin{exercise}{}{}
  Sketch the graph of $x^2+y^2-z^2=-1$. This is called a two-sheeted
  hyperboloid. 
\end{exercise}

\begin{exercise}{}{}
  Show that the graph of $z = y^2 - x^2$ looks like the figure
  below. This is called a hyperbolic paraboloid.
  \begin{center}
    \begin{asy}
      import graph3;
      currentlight.background = softgreen; 
      size(6cm);
      draw(O--2*X,Arrow3());
      draw(O--2*Y,Arrow3());
      draw(O--1.5*Z,Arrow3());
      
      real f(pair z) {return z.y^2 - z.x^2;}
      
      surface s=surface(f,(-1,-1),(1,1),20,Spline);
      
      draw(s, material(diffusepen=0.4*LightSeaGreen+0.6*black+opacity(0.5),
                                emissivepen=0.4*LightSeaGreen+0.6*black+opacity(0.5)),black+opacity(0.3));
    \end{asy}
  \end{center}
\end{exercise}


\section{Polar, cylindrical, and spherical coordinates} \label{sec:coordinates} 

A coordinate system is a way of identifying locations using pairs or
triples of real numbers. Rectangular coordinates---the ones commonly
denoted $(x,y)$ or $(x,y,z)$---have some nice properties, but
some tasks are much more convenient in other coordinate systems.

For example, a captain at sea wishing to communicate the location of a
nearby pirate ship would probably describe its location in terms of
the distance $r$ between the two ships and an angle $\theta$ (which
might be given with reference to the ship's orientation or as a
cardinal direction). The captain is using \textit{polar coordinates}. 

Given a point in the plane, we define $r$ to be its distance from the
origin and $\theta$ to be the angle formed between the positive
horizontal axis and the vector from the
origin to the point. The values $r$ and $\theta$ are called the radial
and angular 
polar coordinates of the point, respectively.* \sidenote{* Properly speaking,
  coordinates are \textit{functions on the plane}. For example, 
  $r = r(x,y) = \sqrt{x^2 + y^2}$}

\begin{exercise}{}{}
  Show that if the polar coordinates of a point $(x,y)$ are $r$ and
  $\theta$, then we have
  \[
    x = r\cos \theta, \quad \text { and } \quad y  = r\sin \theta. 
  \]
\end{exercise}

Correspondingly, we can coordinatize three-dimensional space by
replacing either one or two spatial coordinates with an angular
coordinate. The simplest way to do this is leave $z$ the same and
replace $(x,y)$ with polar coordinates $(r,\theta)$. In other words,
we define
\begin{align*}
  r &= \text{ distance to the }z\text{-axis} \\
  \theta &= \text{ angle of } (x,y) \text{ with respect to positive
           }x\text{-axis} \\  &=
                                \text{ angle of rotation
                                about the }z\text{-axis necessary to hit the
                                positive }yz\text{-plane}  \\ 
  z &= \text{ signed distance to the }xy\text{-plane}. 
\end{align*}
Then we can describe a point by its \textbf{cylindrical coordinates}
$(r,\theta, z)$ rather than its rectangular coordinates, and therefore
we can describe a solid in $\R^3$ by giving inequalities in the
variables $r, \theta,$ and $z$ which are all satisfied for a point if
and only if that point's cylindrical coordinates satisfy them. 

\begin{example}{}{}
  Graph* the system of cylindrical coordinate inequalities $r \leq 4$,
  $0 \leq \theta \leq \pi/3$, $0 \leq z \leq 2$. Find the volume of
  the resulting region. \sidenote{* Familiarity with coordinate
    slices (Table~\ref{table:coordinateslices} in
    Appendix~\ref{sec:polarref}) is helpful for graphing inequalities}
\end{example}

\begin{solution}
  \begin{minipage}[t]{0.7\textwidth} \parskip = 0.2 in 
  The problem is asking us to find the points whose
    cylindrical coordinates satisfy all of the given
    inequalities. Such a point is less than or equal to 4 units from
    the $z$ axis, has polar angle between 0 and $\pi/3$, and is
    between 0 and 2 units from the $z$-axis (and above it). The set of
    such points is shown to the right.

    This region is one-sixth of a cylinder whose volume is
    $\pi r^2 h = 2\pi$, so its volume is $\boxed{\frac{\pi}{3}}$.
  \end{minipage}
  \begin{minipage}[t]{0.29\textwidth}
    \begin{lrbox}{\asybox}
    \begin{asy}
      import graph3;
      currentlight.background=softyellow; 
      size(4cm);
      draw(O--1.2*X,Arrow3());
      draw(O--Y,Arrow3());
      draw(O--1.5*Z,Arrow3());
      
      currentprojection = perspective(2,5,2); 
      
      triple stop(pair p){return (p.x*cos(p.y),p.x*sin(p.y),1);}
      triple sbottom(pair p){return (p.x*cos(p.y),p.x*sin(p.y),0);}
      triple scurvy(pair p){return (cos(p.y),sin(p.y),p.x);}
      triple sside(pair p){return (p.y*cos(pi/3),p.y*sin(pi/3),p.x);}
      
      surface s = surface(stop,(0,0),(1,pi/3),20,Spline);
      surface t = surface(sbottom,(0,0),(1,pi/3),10);
      surface u = surface(scurvy,(0,0),(1,pi/3),10);
      surface v = surface(sside,(0,0),(1,1),10);

      draw(s,LightSeaGreen+opacity(0.5),black);
      draw(t,LightSeaGreen+opacity(0.5));
      draw(u,LightSeaGreen+opacity(0.5),black);
      draw(v,LightSeaGreen+opacity(0.5),black);

      draw(graph(new triple(real t) {return stop((1,t));},0,pi/3,20,Spline));
      draw(graph(new triple(real t) {return stop((t,0));},0,1,20,Spline));
      draw(graph(new triple(real t) {return stop((t,pi/3));},0,1,20,Spline));
      draw(graph(new triple(real t) {return sbottom((1,t));},0,pi/3,20,Spline));
      draw(graph(new triple(real t) {return sbottom((t,0));},0,1,20,Spline));
      draw(graph(new triple(real t) {return sbottom((t,pi/3));},0,1,20,Spline));
      draw(graph(new triple(real t) {return scurvy((t,0));},0,1,20,Spline));
      draw(graph(new triple(real t) {return scurvy((t,pi/3));},0,1,20,Spline));
    \end{asy}
  \end{lrbox} \raisebox{\dimexpr -\height + 1.5ex \relax}{\usebox{\asybox}}
\end{minipage}
\end{solution}

\begin{exercise}{}{}
  Graph the system of inequalities $0 \leq r \leq z$, \: $\pi \leq \theta
  \leq 2\pi$. 
\end{exercise}

Cylindrical coordinates have two distance coordinates and one angular
coordinate. How can we specify a point in space using one distance
coordinate and two angular coordinates? The most natural candidate for
the distance coordinate is the distance from the origin. In other
words, we define $\rho(x,y,z) = \sqrt{x^2 + y^2 + z^2}$. We call this coordinate
$\rho$ instead of $r$ to distinguish it from the radial polar coordinate.

As for the angular coordinates, let's use the cylindrical coordinate
$\theta$ for one of them. For the other, we measure the angle between
the positive $z$-axis and the vector from the origin to
$(x,y,z)$. This pair of angular coordinates might be familiar: we use
them to describe locations on the surface of the earth. In that
context, the angle $\theta$ is called longitude and the angle $\phi$
is called latitude.

Note that $\theta$ varies from 0 to $2\pi$ as one loops around the
$z$-axis. However, the angle $\phi$ varies only from 0 to $\pi$ as one
goes from the north pole to the south pole. Thus the angles $\theta$
and $\phi$ do not play symmetric roles* \sidenote{* This is because
  $\phi$ measures the angle required to rotate a vector freely so as
  to align with the positive $z$-axis, while $\theta$ measures the
  angle needed to rotate the vector \textit{about the $z$-axis} to get
  to the positive half of the $xz$-\textit{plane}. }

\begin{example}{}{spherical}
  Graph the system of inequalities $\tfrac{1}{2} < \rho \leq 1$, \: $0
  \leq \theta \leq \tfrac{\pi}{2}$, \: $0 \leq \phi \leq \tfrac{\pi}{2}$. 
\end{example}

\begin{solution}
  \begin{minipage}[t]{0.7\textwidth}
    The set of points with $\rho \leq 1$ is the set of points on or
    inside of the sphere of radius 1 centered at the origin. Imposing
    the additional constraint $\rho > \tfrac{1}{2}$ removes the sphere
    of radius $\tfrac{1}{2}$ centered at the origin.  Then the angular
    constraints carve out a portion of this hollowed out sphere, as
    shown.
  \end{minipage}
  \begin{minipage}[t]{0.29\textwidth}
    \begin{lrbox}{\asybox}
      \begin{asy} 
      import graph3;
      currentlight.background = softyellow; 
      size(4cm); 
      draw(O--1.2*X,Arrow3());
      draw(O--Y,Arrow3());
      draw(O--1.2*Z,Arrow3());
      
      currentprojection = perspective(2,5,2);

      real theta = pi/4; 

      triple soutside(pair p){return (cos(p.x)*cos(p.y),cos(p.x)*sin(p.y),sin(p.x));}
      triple sinside(pair p){return 0.5*(cos(p.x)*cos(p.y),cos(p.x)*sin(p.y),sin(p.x));}
      triple sside(pair p){return p.x*(cos(p.y)*cos(theta),cos(p.y)*sin(theta),sin(p.y));}

      surface s = surface(soutside,(0,0),(pi/2,theta),10,Spline);
      surface t = surface(sinside,(0,0),(pi/2,theta),10,Spline);
      surface u = surface(sside,(1/2,0),(1,pi/2),10,Spline);
      draw(s,LightSeaGreen+opacity(0.5),black);
      draw(t,LightSeaGreen+opacity(0.5));
      draw(u,LightSeaGreen+opacity(0.5),black);

      draw(graph(new triple(real t) {return soutside((0,t));},0,theta,20,Spline));
      draw(graph(new triple(real t) {return soutside((t,0));},0,pi/2,20,Spline));
      draw(graph(new triple(real t) {return soutside((t,theta));},0,pi/2,20,Spline));
      draw(graph(new triple(real t) {return sinside((0,t));},0,theta,20,Spline));
      draw(graph(new triple(real t) {return sinside((t,0));},0,pi/2,20,Spline));
      draw(graph(new triple(real t) {return sinside((t,theta));},0,pi/2,20,Spline));
      draw(graph(new triple(real t) {return sinside((t,theta));},0,pi/2,20,Spline));
      draw(graph(new triple(real t) {return sside((t,0));},1/2,1,20,Spline));
    \end{asy}
  \end{lrbox} \raisebox{\dimexpr -\height + 1.5ex \relax}{\usebox{\asybox}}
  \end{minipage}  
\end{solution}

\begin{exercise}{}{}
  Find a system of inequalities in spherical coordinates to describe
  the portion of the unit ball* above the plane $z =
  \tfrac{1}{2}$. \sidenote{* The unit ball is the set of points
    satisfying $x^2 + y^2 + z^2 \leq 1$.}
\end{exercise}

\begin{exercise}{}{}
  \begin{minipage}[b]{0.7\textwidth}
    Use the given figure to show that
    \begin{align*}
      x &= \rho \cos \theta \sin \phi \\
      y &= \rho \sin \theta \sin \phi \\
      z &= \rho \cos \phi. 
    \end{align*}
    Hint: use right-triangle trigonometry to write $\sqrt{x^2 + y^2}$
    and $z$ in terms of $\rho$ and $\phi$, and then use a different
    right triangle to write $(x,y,0)$ in terms of $\rho$, $\phi$, and $\theta$.
  \end{minipage}
  \begin{minipage}[b]{0.29\textwidth}
    \begin{asy}
      import graph3;
      currentlight.background = softgreen; 
      size(4cm);
      
      draw(O--1.2*X,Arrow3());
      draw(O--Y,Arrow3());
      draw(O--1.1*Z,Arrow3());

      currentprojection = perspective(2,5,2);

      triple z = (0.8,0.4,0.7);
      triple zproj = (z.x,z.y,0.0); 

      draw("$\rho$", O--z,Arrow3(),align=N);
      draw(z--zproj,dashed);
      draw(zproj--O);

      draw(surface(O--arc(O,0.1*z,0.1*Z)--cycle),green,black); 
      label("$\phi$", 0.1*(z+Z),DarkGreen);
      draw(surface(O--arc(O,0.2*zproj,0.2*X)--cycle),red,black); 
      label("$\theta$", 0.25*zproj+0.2*X,DarkRed);
      
      dot(z); 
    \end{asy}
  \end{minipage}
\end{exercise}
  
\begin{exercise}{}{}
  Determine the graph of the spherical-coordinate equation $\rho =
  2\cos\phi$. (Hint: multiply both sides by $\rho$ and then switch to
  rectangular coordinates.) 
\end{exercise}

\begin{exercise}{}{}
  Determine the graph of $\rho = \sin \phi \sin \theta$.
\end{exercise}

\begin{exercise}{}{}
  Sketch the set of points satisfying $1 < \rho < 2$ and $\phi < \pi/4$.
\end{exercise}
  
\chapter{Multivariable Differentiation}

In this chapter, we will be considering functions from $\R^n$ to
$\R^1$, where $n \geq 2$. The main objectives will be to extend
various important notions in single-variable calculus to the
higher-dimensional setting.

\section{Limits} \label{sec:limits} 

\begin{wrapfigure}[14]{R}{8cm}
  \begin{asy}[width=8cm]
    // limit2d
    defaultpen(fontsize(8)); 
    import graph;
    defaultpen(linewidth(1.5));

    draw((0,0)--(2,0),EndArrow(4));
    draw((0,0)--(0,1.2),EndArrow(4)); 

    real f(real t){
      return exp(-t^2); 
    }

    draw(graph(f,0,2),MidnightBlue);
    label("$f(x)$",(0.3,1.05),MidnightBlue); 

    real epsilon = 0.05;
    real a = sqrt(log(3/2));
    real delta = 0.045;
    real L = 2/3; 

    draw((0,L-epsilon)--(2,L-epsilon),linewidth(0.7));
    draw((0,L+epsilon)--(2,L+epsilon),linewidth(0.7)); 
    draw((a-delta,0) -- (a-delta,1.2),linewidth(0.7));
    draw((a+delta,0) -- (a+delta,1.2),linewidth(0.7)); 

    label(rotate(90)*"$a-\delta$",(a-delta,0),align=S);
    label(rotate(90)*"$a+\delta$",(a+delta,0),align=S); 

    label("$L-\epsilon$",(0,L-epsilon),align=W);
    label("$L+\epsilon$",(0,L+epsilon),align=W); 

    dot("$(a,L)$",(a,L),linewidth(4.0),align=3*E);
  \end{asy}
  \caption{The $\epsilon$-$\delta$ definition of a limit. \label{fig:oneDlimit}}
\end{wrapfigure}
Recall that $f(x)$ converges to $L$ as $x\to a$ if $f(x)$ is as close
to $L$ as desired throughout a sufficiently small neighborhood of
$a$. More precisely, $f(x)$ converges to $L$ as $x\to a$ if and only if for
every $\epsilon > 0$, there is $\delta>0$ so that $f(x)$ within
$\epsilon$ of $L$ for all $x$ satisfying $0 < |x - a| < \delta$ (see
Figure~\ref{fig:oneDlimit}). \sidenote{* We abbreviate ``$f(x)$
  converges to $L$ as $x\to a$'' to $\displaystyle{\lim_{x \to a}f(x) = L}$.}

It can be helpful to think of this definition as a game against an
adversary: the adversary chooses a positive real number $\epsilon$
which can be as small as they like. Then, after seeing the $\epsilon$
value, you get to choose a number $\delta>0$, as small as you
like. Finally, the adversary chooses an $x$ value other than $a$ in
the interval $(a-\delta, a + \delta)$. If it turns out that $|f(x) -
L| \geq \epsilon$, then the adversary wins. Otherwise, you win. We
call this the \textbf{limit game}. 

If you have a strategy for winning this game, then the limit of $f(x)$
as $x$ approaches $a$ exists and equals $L$. If the adversary has a
strategy for winning, then it is not true that $\lim_{x \to a} f(x) =
L$ (either because the limit does not exist, or because the limit
exists and equals a number other than $L$).

\begin{exercise}{}{}
  Show that $\displaystyle{\lim_{x \to 3}}\dfrac{x^2 - 9}{x - 3} = 6$ by explaining
  the winning strategy in the limit game. 
\end{exercise}

\begin{exercise}{}{}
  Suppose that $f$ is a function from $\R$ to $\R$ and that
  $a \in \R$. Show that if $f(x)$ converges to $L$ as $x\to a$ and
  $f(x)$ converges to $L'$ as $x\to a$, then $L = L'$. This fact is called
  \textit{uniqueness of limits}. 
\end{exercise}

We reviewed the definition of a limit for a single-variable function
so we could think about how to generalize the definition for a
function $f:\R^n \to \R$. For simplicity, let's take $n = 2$. What
should it mean to say
$\displaystyle{\lim_{(x,y) \to (a,b)} f(x,y)} = L$? The only aspect of 
the definition that requires revision is the part about $x$ being
within $\delta$ of $a$. But we can use standard Euclidean distance to
compare $(x,y)$ to $(a,b)$. This leads to the following definition. 

\begin{defn}{Limit of a function of two variables}{limit}
  We say $\lim_{(x,y)\to (a,b)} f(x,y) = L$ if and only if for every
  $\epsilon > 0$, there is $\delta>0$ so that $|f(x) - L| < \epsilon$
  for all $(x,y)$ satisfying $0 < \sqrt{(x -a)^2 + (y-b)^2} < \delta$. 
\end{defn}

\begin{wrapfigure}[10]{R}{5cm}
  \begin{asy}[width=5cm]
    size(8cm); 
    import graph3;

    draw(O--1.2*Z,Arrow3());
    draw(O--X,Arrow3());
    draw(O--Y,Arrow3()); 
    
    triple f(pair z) {return (z.x*cos(z.y),z.x*sin(z.y),z.x^2);}
    triple g(pair z) {return (1e-2+1/2 + z.x*cos(z.y),
                                       1e-2+1/2 + z.x*sin(z.y),
                                       (1/2 + z.x*cos(z.y))^2 +
                                       (1/2 + z.x*sin(z.y))^2);}

     real L = 1/2;
     real eps = 0.16;
     real x = 1;
     real y = 0.3; 

     currentprojection = perspective(2,4,1.4); 
     
     surface s = surface((-y,x,L-eps)--(x,x,L-eps)--(x,-y,L-eps)--(-y,-y,L-eps)--cycle);
     surface t = surface((-y,x,L+eps)--(x,x,L+eps)--(x,-y,L+eps)--(-y,-y,L+eps)--cycle); 

     pen planecolor = LightSeaGreen; 
     
     draw(s,planecolor+opacity(0.5));
     draw(surface(xscale(-0.21)*yscale(0.15)*"$L-\epsilon$",s,0.05,0.05,5e-3),gray); 
     draw(t,planecolor+opacity(0.5));
     draw(surface(xscale(-0.21)*yscale(0.15)*"$L+\epsilon$",t,0.02,0.02,5e-3),gray); 
     
     draw(surface(f,(0,0),(1,pi/2),20),LightSeaGreen+opacity(0.3),black);
     draw(surface(g,(0,0),(0.1,2*pi),50),green+opacity(0.6)); 
     draw(surface(circle(c=(1/2,1/2,0),r=0.1,normal=Z)),DarkRed+opacity(0.5));
     
     draw(arc((0,0,L+eps),(sqrt(L+eps),0,L+eps),(0,sqrt(L+eps),L+eps)),0.4*planecolor+linewidth(1.1));
     draw(arc((0,0,L-eps),(sqrt(L-eps),0,L-eps),(0,sqrt(L-eps),L-eps)),0.4*planecolor+linewidth(1.1)); 
     
     dot("$B((a,b),\delta)$",(1/2,1/2,0),align=1.5*S);
     dot((1/2,1/2,1/2));

     draw((1/2,1/2,1/2)--(1/2,1/2,0),dashed); 

     label("$f(x,y)$",(0,0,1.1),align=E); 
  \end{asy}
  \caption{The definition of a limit for a two-variable function \label{fig:twoDlimit}}
\end{wrapfigure}

One way to think about this definition is to consider the shadow$^\dagger$
of the disk* $B((a,b),\delta)$ on the graph (see
Figure~\ref{fig:twoDlimit}). The limit exists and equals $L$ if for
every $\epsilon$, there exists $\delta$ small enough that this shadow
lies entirely in the slab $L - \epsilon < z < L +
\epsilon$. \sidenote{
  * $B((a,b),r)$ means the ball of radius $r$ centered at $(a,b)$. 
  }[-16mm] \sidenote{$^\dagger$\textit{Shadow} here means the set of points
  $(x,y,f(x,y))$ where $(x,y) \in B((a,b),\delta)$.}[-1mm] 

A plane and a line are geometrically different in a way that has major
implications for thinking about limits: there are only two directions
from which to approach a point on a line, but there are infinitely
many ways of approaching a point in the plane. The following example
illustrates a sort of convergence failure which can occur in the
higher dimensional case.

\begin{example}{}{pinch}
    Show that $\displaystyle{\lim_{(x,y) \to
        (0,0)}\frac{-xy}{x^2+y^2}}$ does not exist.
\end{example}

\begin{solution}
  \begin{minipage}{0.7\textwidth}
    Let's begin by graphing the function.  It would appear,
    because of the sharp ``crease'' along the $z$-axis, that the shadow
    of any small disk around the origin (in the $xy$-plane) includes some
    points which are well above the origin and some points which are
    well below. This would imply that the limit does not exist.

    The graph suggests that the largest values occur along the line
    $y=-x$, while the smallest values occur along the line $y = x$. We
    can investigate this algebraically. If $t$ is a nonzero number, no
    matter how small, then
    \[
      f(t,-t) = \frac{-t(-t)}{t^2 + (-t)^2} = \frac{1}{2}, 
    \]
    while
    \[
      f(t,t) = \frac{-t(t)}{t^2 + t^2} = -\frac{1}{2}. 
    \]
  \end{minipage}
  \begin{minipage}{0.29\textwidth}
    \begin{asy}
    import graph3;
    size(45mm);
    defaultpen(fontsize(7));
    material surfacemat = material(diffusepen=LightSeaGreen,
                                                     emissivepen=0.2*LightSeaGreen);
    currentprojection = perspective(5,2,2);
    currentlight.background = softyellow;
    draw(O--1.2*X,Arrow3());
    draw(O--1.15*Y,Arrow3());
    draw(O--1.1*Z,Arrow3());
    real theta = pi/4; 
    real f(pair p){ if (p.x == p.y && p.y == 0) {return 0.0;}
      return min(0.5,-p.x*p.y/(p.x^2 + p.y^2));
    }
    triple g(pair p) {
      real r = p.x;
      real theta = p.y;
      return (r*cos(theta),r*sin(theta),-sin(theta)*cos(theta));
    }
    label("$\displaystyle{f(x,y) = -\frac{xy}{x^2+y^2}}$",(0,0,0.95),0.4*LightSeaGreen,align=3*E); 
    surface s = surface(g,(0,0),(1,2*pi),50, Spline); 
    draw(s,surfacemat);
    \end{asy}
  \end{minipage}
  
  Therefore, the adversary has a strategy for winning the limit game,
  no matter what $L$ is. For example, if $L \geq 0$, then the
  adversary can choose $\epsilon = \tfrac{1}{4}$, and then no matter
  which $\delta$ you choose, the adversary can select
  $(x,y) = (\frac{\delta}{10}, \frac{\delta}{10})$. Then
  $f(x,y) = -\tfrac{1}{2}$, which is not within $\epsilon$ of
  $L$. Similarly, if $L < 0$, then the adversary can choose $\epsilon
  = \tfrac{1}{4}$ again and then $(x,y) = (-\frac{\delta}{10},
  \frac{\delta}{10})$. Since the adversary has a winning strategy for any value of
  $L$, the limit does not exist.
\end{solution}

In Example~\ref{exam:pinch}, there are two directions of approach
along which $f$ has different limits. Along the line $y = x$, the
values of $f(x,y)$ approach* $-\tfrac{1}{2}$. Along the line $y = -x$,
$f(x,y)$ converges to $\tfrac{1}{2}$. This is always an obstruction to
the existence of a limit: \sidenote{* Indeed, these values are
  simply equal to $-\tfrac{1}{2}$}[-5mm]

\begin{exercise}{}{}
  Suppose that $\mathbf{r}_1$ and $\mathbf{r}_2$ are paths in the
  plane with the property that $\mathbf{r}_1(0) = (a,b)$ and
  $\mathbf{r}_2(0) = (a,b)$. If $\lim_{t \to 0}f(\mathbf{r}_1(t))$ and
  $\lim_{t \to 0}f(\mathbf{r}_1(t))$ exist and are unequal, then
  $\displaystyle{\lim_{(x,y) \to (a,b)} f(x,y)}$ does not exist. 
\end{exercise}

Now, suppose we know that the limits of $f(x,y)$ along every line
passing through the origin exist, and that they are all equal to some
common value $L$. Does
this imply that $\lim_{(x,y) \to (0,0)}f(x,y) = L$? It seems like
perhaps it should, since we've accounted for every possible angle of
approach. Remarkably, this turns out not to be the case:

\begin{example}{}{}
  Show that $\lim_{(x,y) \to (0,0)}\frac{-x^2 y }{x^4 + y^2}$ does not
  exist, even though the limits along every line through the origin
  exist and are equal. 
\end{example}

\begin{solution}
  \begin{minipage}{0.7\textwidth}
    We begin by checking the limit along the line $\mathbf{r}(t) = (t\cos
    \theta, t \sin \theta)$ (which is the line passing through the
    origin as well as the point on the unit circle whose angle with
    respect to the positive $x$-axis is $\theta$). We find
    \begin{align*}
      f(t\cos\theta, t \sin \theta) &= \frac{-t^3
                                      \cos\theta \sin \theta}{t^4 \cos^4 \theta + t^2 \sin^2 \theta} \\
                                    &= \frac{-t \cos\theta \sin \theta}{t^2 \cos^4 \theta + \sin^2
                                      \theta}. 
    \end{align*}
  \end{minipage}
  \begin{minipage}{0.29\textwidth}
    \begin{asy}
       import graph3;
       size(45mm);
       defaultpen(fontsize(7)); 
       draw(O--1.2*X,Arrow3());
       draw(O--1.15*Y,Arrow3());
       draw(O--1.1*Z,Arrow3());

       currentprojection = perspective(5,2,2);
       currentlight.background = rgb(0.98, 0.98, 0.9);

       triple g(pair p) {
         real u = p.x;
         real v = p.y;
         real xsq = 2*v^2/(1+sqrt(1+4*v^2*tan(u)^2));
         return (sqrt(xsq),
                 xsq*tan(u),
                 -1/2*sin(2*u));
       }

       label("$\displaystyle{f(x,y) = -\frac{x^2y}{x^4+y^2}}$",(0,0,0.95),0.4*LightSeaGreen,align=3*E);

       int prec1 = 80; int prec2 = 30;

       material surfacemat = material(diffusepen=LightSeaGreen,
                                                         emissivepen=0.2*LightSeaGreen);

       surface s = surface(g,(0.0,0),(pi/2,1),prec1, prec2);
       draw(s,surfacemat);
       surface s = surface(new triple(pair p) {triple t = g(p); return (-t.x,t.y,t.z);},
                    (0.0,0),(pi/2,1),prec1, prec2);
       draw(s,surfacemat);
       surface s = surface(new triple(pair p) {triple t = g(p); return (t.x,-t.y,-t.z);},
                    (0.0,0),(pi/2,1),prec1, prec2);
       draw(s,surfacemat);
       surface s = surface(new triple(pair p) {triple t = g(p); return (-t.x,-t.y,-t.z);},
                    (0.0,0),(pi/2,1),prec1, prec2);
       draw(s,surfacemat);
       \end{asy}
\end{minipage}
Since we're considering each value of $\theta$ individually, the
  $\cos \theta$ and $\sin \theta$ factors are constants. So we see that the
  numerator converges to $0$ and the denominator converges to
  $\sin^2\theta$. Therefore, as long as $\sin \theta \neq 0$, we have
  $\lim_{t\to 0}f(t\cos\theta, t \sin \theta) = 0/\sin^2\theta =
  0$. However, if $\sin \theta = 0$, then
  $f(t\cos\theta, t \sin \theta) = 0$ for all $t$, so
  $\lim_{t\to 0}f(t\cos\theta, t \sin \theta) = 0$ in that case too.
  
  However, note that if we consider the limit along the parabolic
  path $\mathbf{r}(t) = (t, -t^2)$, we get
  \[
    f(t,-t^2) = -\frac{t^2(-t^2)}{t^4+(-t^2)^2} = \frac{1}{2}
  \]
  Therefore, the limit along this path is equal to
  $-\tfrac{1}{2}$. Thus there are two paths (this one, as well as
  any straight-line path through the origin) along which $f$ has
  different limits. Therefore, the limit of $f(x,y)$ as $(x,y) \to
  (0,0)$ does not exist.
  
  Note: this makes sense graphically, because this function also has a
  crease along the $z$-axis. But now we have to follow a
  parabolic path to travel along the top ``ridge'' and realize a
  limiting value other than zero.
\end{solution}

With the notion of a multidimensional limit in hand, we can define
continuity the same as in the one-dimensional case. 

\begin{defn}{}{continuity}
  Suppose $n \geq 2$. A function $f: \R^n \to \R$ is continuous at a
  point in $\R^n$ if and only if the limit of $f$ exists at that point
  and equals the value of the function there.

  A function $f: \R^n \to \R$ is said to be continuous if it is
  continuous at every point in $\R^n$. 
\end{defn}

More generally, a function $f:D \to \R$---where $D \subset \R^n$---is
said to be continuous if it is continuous at each point in its
domain $D$. The following theorem gives us some tools for establishing
continuity. 

\begin{theo}{Continuous functions}{continuity_builders}
  \begin{enumerate} 
    \item If $f  \!:\! \R^n \to \R$ is continuous and $g: \R \to \R$
    is continuous, then* $g\circ f :\R^n \to \R$ is
    continuous. \sidenote{$g\circ f$ denotes the \textit{composition} of
      $g$ and $f$. See Appendix~\ref{a:setsandfunctions}}
  \item A sum or product of continuous functions is continuous. 
  \item The ``coordinate-extracting'' functions $f(x,y,z) = x$, $f(x,y,z) = y$, etc., are
    continuous.
  \end{enumerate}
\end{theo}

\begin{example}{}{}
  Show that $\displaystyle{\lim_{(x,y,z) \to (0,0,0)}\left(e^{\sin x} + \frac{xyz}{1 + x^2
    z^2}\right)} = 1$. 
\end{example}

\begin{solution}
  We begin by showing that $e^{\sin x} + \frac{xyz}{1 + x^2 z^2}$ is
    continuous.  Since $e^{\sin x}$ is a composition of continuous
    functions:
  \[
  (x,y,z) \mapsto x \mapsto \sin x \mapsto e^{\sin x}, 
  \]
  it's continuous by Theorem~\ref{th:continuity_builders}. Similarly,
  $\frac{xyz}{1 + x^2 z^2}$ is continuous wherever $1 + x^2 z^2 \neq
  0$, which is everywhere since $(xz)^2 \geq 0$. Finally, the sum of
  two continuous functions is continuous, so $e^{\sin x} +
  \frac{xyz}{1 + x^2 z^2}$ is continuous.

  Since the function above is continuous, its limit at each point is
  equal to its value at that point. So we substitute $x=y=z=0$ and
  find that the value of the function at the origin is
  $e^0 + \frac{0}{1+0} = 1$.
\end{solution}

We conclude this section with one more tool for showing that a limit
\textit{does} exist: \textbf{polar coordinates}.

\begin{example}{}{}
  Show that $\displaystyle{\lim_{(x,y) \to (0,0)}\frac{x^6 y}{x^4 + y^4} = 0}$. 
\end{example}

\begin{solution}
  For any point $(x,y)$, let's define $r$ and $\theta$ to be the polar
  coordinates of that point and calculate
  \[
    \frac{x^6 y}{x^4 + y^4} = \frac{(r^6 \cos^6 \theta) (r\sin \theta)}{r^4
      \cos^4 \theta + r^4\sin^4 \theta} = r^3 \left(\frac{\cos^6\theta \sin
      \theta}{ \cos^4 \theta + \sin^4\theta}\right). 
  \]
  Now note that the expression $\frac{\cos^6\theta \sin
      \theta}{ \cos^4 \theta + \sin^4\theta}$ is continuous over
    $[0,2\pi]$ and therefore bounded in absolute value* by some
    constant $C$. \sidenote{* This means that the expression is
      never less than $-C$ or greater than $C$.}[-8mm]
    \sidenote{\href{https://cocalc.com/projects/7925f475-a0bd-4621-b78a-466c24c7863c/files/find_local_maximum.ipynb}{\cocalc[-5pt]} Actually, $C = 0.34$ works.
      }[8mm]

    We can use this observation to describe a winning strategy in the
    limit game. Whatever $\epsilon$ is selected by the adversary, we
    choose $\delta$ to be $\sqrt[3]{\frac{\epsilon}{C}}$. Then, 
    no matter which $(x,y)$ pair the adversary selects, the fact that
    the polar coordinate $r$ of $(x,y)$ has to be less than
    $\sqrt[3]{\frac{\epsilon}{C}}$ implies that, regardless $(x,y)$'s
    $\theta$ value, we have 
    \[
      |f(x,y)| <  \left(\sqrt[3]{\frac{\epsilon}{C}}\right)^3 C =
      \epsilon, 
    \]
    as desired. 
  \end{solution}

  \begin{exercise}{}{}
    Show that $\displaystyle{\lim_{(x,y) \to (0,0)} \frac{x^3 + y^3}{x^2 + y^2}} =
    0$. 
  \end{exercise}

  \begin{exercise}{}{}
    Consider the function $f$ defined by $f(x,y) = \frac{x-y}{x^3-y}$
    whenever $y \neq x^3$, and $f(x,y) = 1$ when $y = x^3$. Show that
    $f$ is not continuous at $(1,1)$. Evaluate the limits along along
    $x=1$ and along $y=1$.
  \end{exercise}

  \section{Partial derivatives} \label{sec:partial}

  \milink{partial_derivative_introduction}{partial derivatives}


  Suppose $f$ is a function from $\R$ to $\R$. The derivative $f'$ of
  $f$ is the answer to the question ``how does $f(x)$ change when $x$
  changes just a little?'' More precisely, if $a\in \R$, we define
  \[
    f'(a) = \lim_{h \to 0} \frac{\overbrace{f(a+h)-f(a)}^{\text{how
          much $f$ changes}}}{\underbrace{h}_{\text{how much the input
          changes}}}
  \]
  This means that if we know $f'(a)$, then we can estimate
  $f(a+h) - f(a)$ for $h$ very small:
  \[
    f(a+h) - f(a) \approx h
    f'(a). 
  \]
  So the derivative measures \textbf{how sensitive $f(x)$ is to small
    changes in $x$}.
  
  What is the most natural corresponding idea for the derivative at
  some point $(a,b)$ of a function $f$ from $\R^2$ to $\R$? We were
  only able to adjust a value $x\in \R$ by increasing or decreasing it
  a little. A point in $\R^2$, by contrast, can be moved in any
  direction. Two directions are particularly easy to study: (i) move
  $x$ a little while holding $y$ fixed, and (ii) move $y$ a little
  while holding $x$ fixed. Accordingly, we define \textbf{partial
    derivatives}* \sidenote{* $\partial_x$ is read ``partial
    $x$''. Also, the role of $x$ here is purely as a label that means
    ``with respect to the first coordinate''. It does not represent a
    number, as the symbol $x$ usually does. }[5mm]
  \begin{align*}
    (\partial_x f)(a,b) &= \lim_{h \to 0}\frac{f(a+h,b) - f(a,b)}{h},
                          \text{ and} \\
    (\partial_y f)(a,b) &= \lim_{h \to 0}\frac{f(a,b+h) - f(a,b)}{h}. 
  \end{align*}
  Calculating partial derivatives is easy because \textit{you already
    know how to do it}. Since one of the two variables is being held
  constant, we are effectively taking a derivative with respect to a
  single-variable function.

  \begin{example}{}{}
    Find the partial derivatives of $f(x,y) = e^x \sin (xy)$ at $(x,y)
    = (1,0)$. 
  \end{example}

  \begin{solution}
    We can find the partial derivative with respect to $x$ at
    \textit{any} point $(x,y)$ by treating $y$ as constant and
    applying single-variable differentiation rules:* \sidenote{* If
      you have difficulty getting used to holding a variable
      constant, consider replacing it with some number like 17; then
      substitute back at the end}
    \begin{align*}
      (\partial_xf)(x,y) &= e^{x} y \cos\left(x y\right) + e^{x}
                           \sin\left(x y\right) \\  
      (\partial_yf)(x,y)  &= x \cos\left(x y\right) e^{x}
    \end{align*}
    So the partial derivatives at $(1,0)$ with respect to $x$ and $y$
    are 0 and $e$, respectively.
  \end{solution}

  \begin{example}{}{}
    \begin{minipage}[t]{0.7\textwidth}
      Consider the function $f$ whose graph is shown. Determine the sign
      of $(\partial_x f)(1,1)$ and the sign of $(\partial_y
      f)(1,1)$.
    \end{minipage}
    \begin{minipage}[t]{0.29\textwidth}
      \begin{lrbox}{\asybox}
        \begin{asy}[width=4cm]
        import graph3; 

        draw(O--2.2*X,Arrow3());
        draw(O--2.2*Y,Arrow3());
        draw(O--1.1*Z,Arrow3());

        currentprojection = perspective(5,2,2);
        currentlight.background = softblue; 
  
        real theta = pi/4; 

        real f(pair p){ if (p.x == p.y && p.y == 0) {return 0.0;}
          real x = p.x, y = p.y; 
          return 2*x*y*exp(-2*x^2-(y-0.8)^3); 
        }

        label("$f(x,y)$",(0,0,0.95),0.4*LightSeaGreen,align=3*E); 

        surface s = surface(f,(0,0),(2,2),20,Spline);
        draw(s,LightSeaGreen,MidnightBlue);
        dot((1,1,f((1,1))),softblue);
        label("$(1,1,f(1,1))$",(1.4,0.8,0.2),softblue+fontsize(8)); 
      \end{asy}
    \end{lrbox} \raisebox{\dimexpr -\height + 1.5ex \relax}{\usebox{\asybox}}
  \end{minipage}
  \end{example}

  \begin{solution}
      \begin{minipage}[b]{0.7\textwidth}
        If we increase $x$ a little while holding $y$ constant, then
        $f$ decreases. Therefore, $(\partial_x f)(1,1) < 0$. If we
        increase $y$ a little while holding $x$ constant, then $f$
        increases. Therefore, $(\partial_y f)(1,1) > 0$.

        Graphically, the partial derivative with respect to $x$ at a
        point is equal to the slope of the trace of the graph in the
        ``$y =\text{constant}$" plane passing through that
        point. Similarly, the partial derivative with respect to $y$
        at a point is equal to the slope of the trace of the graph in
        the ``$x =\text{constant}$" plane passing through that
        point.* \sidenote{* Thus we can think of partial derivatives
          as an application of our ``slice it up'' strategy for
          understanding three dimensional objects through two
          dimension traces}[-2cm]
  \end{minipage}
  \begin{minipage}[b]{0.29\textwidth}
    \begin{asy}[width=4.5cm]
      import graph3; 

      draw(O--2.2*X,Arrow3());
      draw(O--2.2*Y,Arrow3());
      draw(O--1.1*Z,Arrow3());

      pen softblue = rgb(0.92,0.95,0.99);
      pen softyellow = rgb(0.98, 0.98, 0.9); 

      currentprojection = perspective(5,2,2);
      currentlight.background = softyellow; 
      
      real theta = pi/4; 

      real f(pair p){ if (p.x == p.y && p.y == 0) {return 0.0;}
        real x = p.x, y = p.y; 
        return 2*x*y*exp(-2*x^2-(y-0.8)^3); 
      }

      triple g(real x) {return (x,1,f((x,1)));}
      triple h(real y) {return (1,y,f((1,y)));}

      label("$f(x,y)$",(0,0,0.95),0.4*LightSeaGreen,align=3*E); 

      surface s = surface(f,(0,0),(2,2),20,Spline);
      draw(s,LightSeaGreen+opacity(0.2),MidnightBlue);

      dot((1,1,f((1,1))),MidnightBlue);
      draw(graph(g,0,2,50),MidnightBlue);
      draw(graph(h,0,2,50),MidnightBlue);

      real eps = 0.3;
      real mx = -0.806;
      real my = 0.236; 

      draw((1,1,f((1,1))) -- (1+eps,1,eps*mx+f((1,1))),LightSeaGreen+linewidth(1.0),Arrow3(8));
      draw((1,1,f((1,1))) -- (1,1+eps,eps*my+f((1,1))),LightSeaGreen+linewidth(1.0),Arrow3(8));

      draw((1,0,0)--(1,2,0),gray);
      draw((0,1,0)--(2,1,0),gray); 
    \end{asy}
  \end{minipage}
\end{solution}

\begin{exercise}{}{whichiswhich}
  The following three graphs represent a function $f$ and its two
  partial derivatives $\partial_x f$ and $\partial_y f$, in some
  order. Which is which?
  \newsavebox{\asyboxtwo}
  \newsavebox{\asyboxthree}
  \begin{lrbox}{\asybox}
    \begin{asy}[width=4.5cm]
      import graph3; 
      picture myaxes;
      draw(myaxes,O--1.2*X,Arrow3());
      draw(myaxes,O--1.2*Y,Arrow3());
      draw(myaxes,O--1.2*Z,Arrow3());
      draw(myaxes,X--X+Y--Y,gray); 
      
      currentprojection = perspective(5,2,2);
      currentlight.background = softgreen; 
      
      real theta = pi/4; 
      
      real f(pair p){ 
        real x = p.x, y = p.y; 
        return 1 + x * y^3 - x^2; 
      }
      
      real fx(pair p){
        real x = p.x, y = p.y;
        return (y^3 - 2*x)/2; 
      }
      
      real fy(pair p){
        real x = p.x, y = p.y;
        return 3*x*y^2/3; 
      }
      
      add(myaxes); 
      
      surface s = surface(f,(0,0),(1,1),20,Spline);
      draw(s,MidnightBlue+opacity(0.4),MidnightBlue);
      draw((0,0,0)--(0,0,f((0,0))),gray);
      draw((0,1,0)--(0,1,f((0,1))),gray);
      draw((1,1,0)--(1,1,f((1,1))),gray);
      draw((1,0,0)--(1,0,f((1,0))),gray); 
    \end{asy}
  \end{lrbox}
  \begin{lrbox}{\asyboxtwo}
    \begin{asy}[width=4.5cm]
      import graph3; 
      picture myaxes;
      draw(myaxes,O--1.2*X,Arrow3());
      draw(myaxes,O--1.2*Y,Arrow3());
      draw(myaxes,O--1.2*Z,Arrow3());
      draw(myaxes,X--X+Y--Y,gray); 
      
      currentprojection = perspective(5,2,2);
      currentlight.background = softgreen; 
      
      real theta = pi/4; 
      
      real f(pair p){ 
        real x = p.x, y = p.y; 
        return 1 + x * y^3 - x^2; 
      }
      
      real fx(pair p){
        real x = p.x, y = p.y;
        return (y^3 - 2*x)/2; 
      }
      
      real fy(pair p){
        real x = p.x, y = p.y;
        return 3*x*y^2/3; 
      }
      
      add(myaxes); 
      
      surface s = surface(fx,(0,0),(1,1),20,Spline);
      draw(s,MidnightBlue+opacity(0.4),MidnightBlue);
      draw((0,0,0)--(0,0,fx((0,0))),gray);
      draw((0,1,0)--(0,1,fx((0,1))),gray);
      draw((1,1,0)--(1,1,fx((1,1))),gray);
      draw((1,0,0)--(1,0,fx((1,0))),gray); 
    \end{asy}
  \end{lrbox}
  
  \begin{lrbox}{\asyboxthree}
    \begin{asy}[width=4.5cm]
      import graph3; 
      picture myaxes;
      draw(myaxes,O--1.2*X,Arrow3());
      draw(myaxes,O--1.2*Y,Arrow3());
      draw(myaxes,O--1.2*Z,Arrow3());
      draw(myaxes,X--X+Y--Y,gray); 
      
      currentprojection = perspective(5,2,2);
      currentlight.background = softgreen; 
      
      real theta = pi/4; 
      
      real f(pair p){ 
        real x = p.x, y = p.y; 
        return 1 + x * y^3 - x^2; 
      }
      
      real fx(pair p){
        real x = p.x, y = p.y;
        return (y^3 - 2*x)/2; 
      }
      
      real fy(pair p){
        real x = p.x, y = p.y;
        return 3*x*y^2/3; 
      }
      
      add(myaxes); 
      
      surface s = surface(fy,(0,0),(1,1),20,Spline);
      draw(s,MidnightBlue+opacity(0.4),MidnightBlue);
      draw((1,1,0)--(1,1,fy((1,1))),gray);  
    \end{asy}
  \end{lrbox}
  
  \begin{center}  
    \begin{tabular}{M{0.3\textwidth}M{0.3\textwidth}M{0.3\textwidth}N}
      \usebox{\asyboxthree} & \usebox{\asybox} & \usebox{\asyboxtwo}
    \end{tabular}
  \end{center}
\end{exercise}

The following theorem says that order doesn't matter when successively
taking partial derivatives.

\begin{theo}{Clairaut's theorem}{clairaut}
  Suppose $f:D \to \R$, where $D$ is a disk in $\R^2$. If
  $\partial_x \partial_y f$ and $\partial_y \partial_x f$ exist and
  are continuous, then $\partial_x \partial_y f = \partial_y
  \partial_x f$ throughout $D$. 
\end{theo}

\begin{exercise}{}{}
  Verify the conclusion of Clairaut's theorem for $f(x,y) = e^{xy}
  \sin y$. 
\end{exercise}

\section{Linear approximation} \label{sec:linapprox} 

\milink{linear_approximation_multivariable}{linear approximation}

The following example shows that partial derivatives don't tell the
whole story when it comes to differentiating functions of multiple
variables.

\begin{example}{}{pinchder}
  Consider the function $f$ for which $f(0,0)=0$ and $f(x,y) =
  -\frac{xy}{x^2 + y^2}$ for all $(x,y) \neq (0,0)$. Show that both
  partial derivatives of $f$ at the origin are equal to zero. 
  \end{example}

  \begin{solution}
    If we move $x$ a little from $x=0$ while holding $y=0$ fixed, the
    value of $f$ doesn't change at all. Therefore,
    \[
      (\partial_x f)(0,0) = \lim_{h \to 0} \frac{f(h,0) - f(0,0)}{h} =
      \lim_{h \to 0} \frac{0}{h} =  \lim_{h \to 0} 0 = 0. 
    \]
    The same is true for the partial derivative with respect to $y$. 
  \end{solution}

  However, recall from Example~\ref{exam:pinch} that the function in
  Example~\ref{exam:pinchder} isn't even continuous at the origin! We
  haven't said yet what it is required for a function of two variables to be
  considered differentiable, but whatever the definition, we surely cannot allow
  functions which aren't continuous. 

  This shouldn't be surprising: the partial derivatives only look at
  the behavior of the function along two lonely slices. A good
  definition of differentiability at $(a,b)$ should account for how the
  function behaves in every direction around $(a,b)$.

  Another perspective on differentiability in the single-variable
  context is that \textit{differentiable functions are the ones which
    are well-approximated by linear functions}:

  \begin{theo}{}{diff_linear}
    A function $f: \R \to \R$ is differentiable at $a \in \R$ if and
    only if there exists a linear function $L(x) = c_0 + c_1(x-a) $
    such that* \sidenote{* This equation says that $L$ approximates
      $f$ so well that the difference between $f$ and $L$,
      \textit{even after being divided by the tiny number $|x-a|$}, still
      goes to 0 as $x \to a$.}
    \[
      \lim_{x \to a}\frac{f(x) - L(x)}{|x-a|} = 0. 
    \]
  \end{theo}

  This perspective on differentiability turns out to generalize very
  nicely to functions of multiple variables. Let's make it a
  definition.

  \begin{defn}{Differentiability for a function of two variables}{diff}
    A function $f: \R^2 \to \R$ is differentiable at $(a,b) \in \R^2$
    if and only if there exists a linear function
    $L(x,y) = c_0 + c_{1}(x-a) + c_{2}(y-b)$ such that
    \[
      \lim_{(x,y) \to (a,b)}\frac{f(x,y) - L(x,y)}{\sqrt{(x-a)^2 + (y-b)^2}} = 0. 
    \]
  \end{defn}

  Lots of functions are differentiable. The following theorem
  establishes a handy way to check differentiability. 

  \begin{theo}{Criterion for differentiability}{partials} 
    If the partial derivatives $\partial_x f$ and $\partial_y f$ exist
    in some disk centered at $(a,b)$ and are continuous at $(a,b)$, then
    $f$ is differentiable at $(a,b)$. 
  \end{theo}

  The most common situation is that partial derivatives exist and are
  continuous everywhere, in which case Theorem~\ref{th:partials}
  implies that $f$ is differentiable everywhere.

  \begin{example}{}{partials}
    Show that $f(x,y) = e^{xy} \sin (x^2 + y^2)$ is differentiable at
    every point in $\R^2$. 
  \end{example}

  \begin{solution}
    We can take partial derivatives of $f$ with respect to both $x$
    and $y$ and get functions which are built from $x$ and $y$ using
    addition/multiplication as well as the continuous functions
    $x\mapsto e^x$ and $x\mapsto \sin x$. Therefore, the partial
    derivatives exist and are continuous everywhere. Thus
    Theorem~\ref{th:partials} implies that $f$ is differentiable
    everywhere. 
  \end{solution}

  \begin{wrapfigure}[8]{R}{5cm}
    \begin{asy}[width=5cm]
      import graph3; 

      draw(O--2.2*X,Arrow3());
      draw(O--2.2*Y,Arrow3());
      draw(O--1.1*Z,Arrow3());

      pen softblue = rgb(0.92,0.95,0.99);
      pen softyellow = rgb(0.98, 0.98, 0.9); 

      currentprojection = perspective(5,2,2);
  
      real theta = pi/4; 

      real f(pair p){ if (p.x == p.y && p.y == 0) {return 0.0;}
        real x = p.x, y = p.y; 
        return 2*x*y*exp(-2*x^2-(y-0.8)^3); 
      }
      
      real mx = -0.805541521045397;
      real my = 0.236292179506650;  
      
      real L(pair p) {return f((1,1)) + mx*(p.x - 1) + my*(p.y - 1);}
      
      triple g(real x) {return (x,1,f((x,1)));}
      triple h(real y) {return (1,y,f((1,y)));}
      
      label("$f(x,y)$",(0,0,0.95),0.4*LightSeaGreen,align=3*E); 
      
      surface s = surface(f,(0,0),(2,2),20,Spline);
      draw(s,LightSeaGreen+opacity(0.2),MidnightBlue);
      
      real w = 0.5; 

      surface t = surface(L,(1-w,1-w),(1+w,1+w));
      draw(t,LightSeaGreen+opacity(0.5)); 

      dot((1,1,f((1,1))),MidnightBlue);

      real eps = 0.3;

      draw((1,1,f((1,1))) -- (1+eps,1,eps*mx+f((1,1))),MidnightBlue+linewidth(1.0),Arrow3(8));
      draw((1,1,f((1,1))) -- (1,1+eps,eps*my+f((1,1))),MidnightBlue+linewidth(1.0),Arrow3(8));

      draw((1,0,0)--(1,2,0),gray);
      draw((0,1,0)--(2,1,0),gray); 
    \end{asy}
    \caption{A plane tangent to the graph of a function
      $f$ \label{fig:tangentplane}}
  \end{wrapfigure}
  
  In the denominator we replaced $|x-a|$, whose geometric meaning is
  the distance from $x$ to $a$ on the number line, with the formula
  for the distance from $(x,y)$ to $(a,b)$ in the plane.

  Graphically, Definition~\ref{defn:diff} says that a function is
  differentiable at $(a,b)$ if we can draw a plane which is tangent*
  to the graph of $f$ at the point $(a,b,f(a,b))$. \sidenote{* This
    is a little bit of a lie, because to be true strictly speaking it
    would require that we define what it means to be \textit{tangent}
    in terms of Definition~\ref{defn:diff}, which would be
    circular. So take this as intuition only.}

  In Theorem~\ref{th:diff_linear}, the coefficients of the
  approximating function $L$ are the value of the function $f$ at $a$
  and the derivative of $f$ at $a$. The coefficients in
  Definition~\ref{defn:diff} are also named quantities: as suggested
  by Figure~\ref{fig:tangentplane}, $c_1$ is the value of the function
  at $(a,b)$ and $c_1$ and $c_2$ are the two partial derivatives at
  $(a,b)$:

  \vspace{12pt} 

  \begin{theo}{Linear Approximation}{linapprox}
    If $f: \R^2 \to \R$ is differentiable at $(a,b) \in \R^2$, then
    \[
      \lim_{(x,y) \to (a,b)}\frac{f(x,y) - \overbrace{\left[f(a,b) + (\partial_x
        f)(a,b)(x-a) + (\partial_y
        f)(a,b)(y-b)\right]}^{L(x,y)}}{\sqrt{(x-a)^2 + (y-b)^2}} = 0.
    \]
  \end{theo}

  Let's see how this theorem can be used numerically. 

  \begin{example}{}{approximate}
    Consider the function $f(x,y) = \frac{e^{xy}}{e(1+x^2)}$. Use a
    tangent plane to approximate $f(0.99,0.98)$.
  \end{example}

  \begin{solution}
    Noticing that $(0.99,0.98)$ is very close to $(1,1)$, we
    differentiate $f(x,y)$ with respect and with respect to $y$ and
    find that* \sidenote{* The bar notation means ``substitute''}
    \begin{align*}
      (\partial_x f)(1,1) &= \left.\left(
                            \frac{y e^{x y}}{e(x^{2} + 1)} -
                            \frac{2 \, x e^{xy}}{{e\left(x^{2} +
                            1\right)}^{2}}\right)\right|_{(x,y)
                            =(1,1)} = 0. \\
      (\partial_y f)(1,1) &=    \left.\frac{x e^{x y}}{e(x^{2} +
                            1)}\right|_{(x,y) = (1,1)} = \frac{1}{2}. 
    \end{align*}
    Therefore, $f(0.99,0.98) \approx f(1,1) + 0(0.99-1) +
    \frac{1}{2}(0.98-1) = \frac{1}{2} + \frac{1}{2} \cdot
    (-\frac{1}{50}) = 0.49$. \sidenote{\href{https://cocalc.com/projects/7925f475-a0bd-4621-b78a-466c24c7863c/files/substitute.ipynb}{\cocalc} \, The actual value is
      $0.490197\ldots$ } 
  \end{solution}
  
  \section{Multivariable optimization} \label{sec:optim} 

  \milink{local_extrema_introduction_two_variables}{local extrema}

  The following problem is a typical example of a single-variable
  optimization problem.

  \begin{example}{}{optim1d}
    Find the maximum and minimum of $f(x) = |(x - 1)(3-x)|$ over the interval
    $[0,3]$. 
  \end{example}

  \begin{solution}
    \begin{minipage}[b]{0.7\textwidth}
      Since $f$ is continuous over the closed and bounded interval
      $[0,3]$, we know by the extreme value theorem that it has a
      maximum and a minimum value over $[0,3]$. Furthermore, these
      extrema must be realized at either a \textit{critical point}, at
      which $f$ is either not differentiable or has derivative zero, or
      else at an endpoint of the interval.
      
      We check that $f$ is not differentiable at $1$ or $3$ (see the
      graph). Also, we can solve $f'(x) = 0$ to find that $f$ has a
      horizontal tangent line at $x = 2$.
      
      Finally, we can check the values of $f$ at the endpoints 0 and 3,
      as well as the critical points strictly between them, namely 1 and
      2. We find that the maximum value is $f(0) = \boxed{3}$, and the
      minimum value is $\boxed{0}$, which occurs at $x =1$ and at
      $x = 3$. 
    \end{minipage} \hspace{5mm}
    \begin{minipage}[b]{0.29\textwidth}
      \begin{asy}[width=4cm]
        defaultpen(fontsize(8));
        import graph;
        real f(real x){ return abs((1-x)*(x-3));}
        draw(graph(f,0,3,500),MidnightBlue);
        draw((0,0)--(3,0)); 
        draw((0,0)--(0,4),Arrow());
        label("$x$",(3.2,0.0),MidnightBlue);
        real eps = 0.05;
        draw((1,-eps)--(1,eps));
        label("1",(1,0),align=2*S); 
        draw((2,-eps)--(2,eps));
        label("2",(2,0),align=2*S); 
        draw((3,-eps)--(3,eps));
        label("3",(3,0),align=2*S); 
        label("$f(x)=|(x-1)(3-x)|$",(1.4,3),align=N,MidnightBlue);  
      \end{asy}
    \end{minipage}
\end{solution}

How does this story change when we consider a function of multiple
variables? For concreteness, let's suppose $D = [0,1]^2$ and that
$f:D \to \R$ is a continuous function. Consider the graph of the
function
  \[
    f(x,y) = -x^2 - y^2 + x + \frac{2}{3} y + \frac{23}{36}, 
  \]
  shown in Example~\ref{exam:optim2d}. We can see that any extremum must
  occur either (i) at a point somewhere on the boundary of the square,
  or (ii) a point inside the square where the tangent plane is
  parallel to the $xy$-plane (or where the function is not
  differentiable).

  This observation gives us a strategy for finding the extrema of
  a function $f:D \to \R$, where $D \subset \R^2$: (i) set both
  partial derivatives of $f$ equal to 0 and solve to find critical
  points inside $D$ (also include any points where $f$ is not
  differentiable), and (ii) find the extreme values of $f$ on*
  $\partial D$. \sidenote{* The notation $\partial D$ means ``the
    boundary of $D$'', which is the set of points $p \in \R^2$ such that
    any small disk centered at $p$ includes points in $D$ and points
    not in $D$.}[-2cm]
  
  We can see that this is considerably more complicated that the
  single-variable optimization: we have to solve a system of equations
  to find interior critical points, and we have to find any boundary critical
  points as well. If $D$ is a rectangle, for example, then finding
  the extrema of $f$ on the boundary of $D$ boils down to doing four
  single-variable optimization problems (one for each side of the
  square). Let's see how this works out for the function shown above.

  \begin{example}{}{optim2d}
    \begin{minipage}[t]{0.6\textwidth}
      Find the extreme values of the function
      $f(x,y) = -x^2 - y^2 + x + \frac{2}{3} y + \frac{23}{36}$ over
      the square $[0,1]^2$.
    \end{minipage}
    \begin{minipage}[t]{0.39\textwidth}
      \begin{lrbox}{\asybox}
        \begin{asy}[width=5cm]
          import graph3; 
          
          picture myaxes;
          draw(myaxes,O--1.2*X,Arrow3());
          draw(myaxes,O--1.2*Y,Arrow3());
          draw(myaxes,O--1.2*Z,Arrow3());
          
          currentprojection = perspective(5,2,2);
          currentlight.background = softblue; 
          
          real f(pair p){ 
            real x = p.x, y = p.y; 
            return 1 - (x-1/2)^2 - (y-1/3)^2;
          }
          
          add(myaxes); 
          surface s = surface(f,(0,0),(1,1),20,Spline);
          draw(s,MidnightBlue+opacity(0.4),MidnightBlue);
          draw(surface((0,0)--(1,0)--(1,1)--(0,1)--cycle),white+opacity(0.5));
          label("$f(x,y)$", (0,0,1), MidnightBlue, align=E); 
        \end{asy}
      \end{lrbox} \raisebox{\dimexpr -\height + 1.5ex \relax}{\usebox{\asybox}}
    \end{minipage}
  \end{example}

  \begin{solution}
    We begin by finding the critical points inside the square. We find
    \begin{align*}
      (\partial_xf)(x,y) &= -2x + 1  \\
      (\partial_yf)(x,y) &= -2y + \frac{2}{3}. 
    \end{align*}
    These quantities are both equal to zero only when $x =
    \tfrac{1}{2}$ and $y = \frac{1}{3}$. So $(1/2,1/3)$ is the only critical
    point inside the square.

    To optimize $f$ along the $x = 0$ side, we look at
    \[
      f(0,y) = -y^2 + \frac{2}{3}y + \frac{23}{36}, 
    \]
    which has a critical point at $y = 1/3$. So $(0,1/3)$ is a
    \textbf{boundary critical point}, and we should also check the two
    endpoints $(0,0)$ and $(0,1)$. Similarly, for the other three
    sides, we identify the points $(1,1/3)$, $(1/2,0)$, $(1/2,1)$, as
    boundary critical points as well as the other two corners $(1,1)$
    and $(1,0)$. So, all together:
    \[
      \renewcommand\arraystretch{1.4}
      \begin{array}{c|ccccccccc}
       (x,y) &  (0,0) & (1,0) & (0,1) & (1,1) & (0,1/3) & (1,1/3) & (1/2,0) &
                                                                      (1/2,1)
        & (1/2,1/3) \\ \hline
        f(x,y) & 23/36 & 23/36 & 11/36 & 11/36 & 3/4 & 3/4 & 8/9 & 5/9
        & 1 \\
        f(x,y) & 0.62 & 0.62 & 0.31 & 0.31 & 0.75 & 0.75 & 0.88 & 0.56
        & 1
      \end{array}
    \]
    So the maximum value is $\boxed{1}$ and the minimum value is
    $\boxed{\tfrac{11}{36}}$. 
  \end{solution}

  \begin{exercise}{}{}
    Find the maximum value of $f(x,y) = 10x^2y-x$ over the closed
    triangle with vertices $(0,0)$, $(1,0)$, and $(0,1)$.
  \end{exercise}
  
  \section{Directional derivative and gradient} \label{sec:dd_and_grad} 

  \milink{directional_derivative_gradient_introduction}{directional derivatives and the gradient}

  \begin{wrapfigure}[7]{R}[1cm]{7cm} 
    \begin{asy}[width=7cm]
      import graph3;

      draw(O--2.2*X,Arrow3());
      draw(O--2.2*Y,Arrow3());
      draw(O--1.1*Z,Arrow3());
      
      currentprojection = perspective(5,2,2);
  
      real theta = pi/4; 
      
      real f(pair p){ if (p.x == p.y && p.y == 0) {return 0.0;}
        real x = p.x, y = p.y; 
        return 2*x*y*exp(-2*x^2-(y-0.8)^3); 
      }
      
      real mx = -0.805541521045397;
      real my = 0.236292179506650;  

      real L(pair p) {return f((1,1)) + mx*(p.x - 1) + my*(p.y - 1);}
      
      triple g(real x) {return (x,1,f((x,1)));}
      triple h(real y) {return (1,y,f((1,y)));}

      label("$f(x,y)$",(0,0,0.95),0.4*LightSeaGreen,align=3*E); 
      
      surface s = surface(f,(0,0),(2,2),20,Spline);
      draw(s,LightSeaGreen+opacity(0.2),MidnightBlue);

      real w = 0.5; 
      surface t = surface(L,(1-w,1-w),(1+w,1+w));
      dot((1,1,f((1,1))),MidnightBlue);
      real eps = 0.3;
      draw(circle(c=(1,1,0),r=eps,normal=Z),0.8*white+dashed); 
      pair u = (2/sqrt(5),1/sqrt(5));
      triple j(real t) {return (1 + u.x*t, 1+u.y*t, f((1+u.x*t, 1+u.y*t)));}
      draw(graph(j,-1,1.12,100),MidnightBlue); 
      
      draw((1,1,f((1,1))) -- (1+eps*u.x,1+eps*u.y, f((1+eps*u.x,1+eps*u.y))),
      MidnightBlue+linewidth(1.0),Arrow3(6));

      draw((1,0,0)--(1,2,0),gray);
      draw((0,1,0)--(2,1,0),gray);
      draw("$(a,b) + h\mathbf{u}$",(1,1,0)--(1,1,0) + (0.95*eps*u.x,0.95*eps*u.y,0),gray+fontsize(8),Arrow3(4),align=10*SW);
      draw((1.65,0.95,0)--(1.02+eps*u.x,0.98+eps*u.y,0),gray,Arrow3(2)); 
      dot((1+eps*u.x,1+eps*u.y,0), gray+linewidth(1.5)); 
      draw((1,1,f((1,1))) --
      (1,1,f((1,1))) + (eps*u.x,eps*u.y,0) --
      (1+eps*u.x,1+eps*u.y,f((1+eps*u.x,1+eps*u.y))),0.2*white,align=2*SE);
      
      draw((1,1.2,0.15)--(1+eps*u.x,1+eps*u.y+0.01,0.04+0.5*(eps*mx*u.x + eps*my*u.y) + f((1,1))),gray,Arrow3(2));
      
      triple A = (1,1,0);
      triple uu = 1.12*(u.x,u.y,0);
      triple zz = 0.5 * Z; 
      
      draw(surface(A-uu -- A+uu -- A+uu+zz -- A-uu+zz--cycle),opacity(0.2)+MidnightBlue); 
      
      label("$f((a,b) + h \mathbf{u}) - f(a,b)$",(1,1.2,0.1),black+fontsize(8), align=NE);
      
      dot("$(a,b)$", (1,1,0), gray+fontsize(8)+linewidth(1.5), align=10*WSW);
      draw((1.27,0.8,0)--(1.02,0.98,0),gray,Arrow3(2)); 
    \end{asy}
    \caption{The derivative of $f$ in the direction $\mathbf{u}$ \label{fig:dirder}}
  \end{wrapfigure}
  
  The two partial derivatives of a function $f: \R^2 \to \R$ tell us
  how $f$ changes when $(x,y)$ is wiggled a bit, but only in the four
  cardinal directions.* What about all the other directions? Suppose
  that $\mathbf{u}$ is a \textbf{unit vector} in $\R^2$, meaning that its
  length is 1. (see
  Figure~\ref{fig:dirder}). \sidenote{* That is: \\ up/down,
    left/right.}[-6mm]

  \begin{defn}{Directional derivative}{dirder}
    The \textbf{directional derivative} of a function $f:\R^2 \to \R$
    in the direction $\mathbf{u} \in \R^2$ is defined by
    \[
      D_{\mathbf{u}}(f)(a,b) = \lim_{h \to 0}\frac{f( (a,b) + h
        \mathbf{u}) - f(a,b)}{h}. 
    \]  
  \end{defn}

  In other words, move $(x,y)$ a small distance $h$ in the
  $\mathbf{u}$ direction, measure how much $f$ changed, and then
  divide $h$.

  If the partial derivatives of $f$ exist and are continuous around
  $(a,b)$, then we can work out the derivative in the
  $\mathbf{u} = \langle u_2, u_2 \rangle$ direction in terms of the
  partial derivatives at $(a,b)$ by breaking down a $\mathbf{u}$-step
  into a $\langle u_1, 0\rangle$ step and a $\langle 0, u_2\rangle$
  step: the value of $f$ changes by approximately
  $u_1h (\partial_xf)(a,b)$ as we change the input value from $(a,b)$
  to $(a+u_1h, b)$, and then by $u_2h (\partial_yf)(a+u_1h,b)$ as we
  move from $(a+u_1h, b)$ to $(a+u_1h, b + u_2 h)$. But
  $(\partial_yf)(a+u_1h,b)$ is approximately
  $(\partial_yf)(a,b)$ for small $h$, since $\partial_y f$ is
  continuous around $(a,b)$. This leads to the following theorem.

  \begin{theo}{Directional derivative formula}{dirder}
    If $f$ is differentiable at $(a,b)$ and $ \mathbf{u}  \in \R^2$, then
    \[
      D_{\mathbf{u}} f(a,b) = (\partial_xf)(a,b)u_1 +
      (\partial_yf)(a,b)u_2 =  (\nabla f)(a,b) \cdot \mathbf{u}, 
    \]
    where $\nabla f = \langle \partial_x f, \partial_y f \rangle$. 
  \end{theo}

  The quantity $\nabla f$ introduced in Theorem~\ref{th:dirder}---the
  vector partial derivatives of $f$---is called the \textbf{gradient}
  of $f$. Observe that the directional derivative of $f$ in the
  $\mathbf{u}$ direction is equal to
  $\nabla f \cdot \mathbf{u} = |\nabla f| \cos \theta$, where $\theta$
  is the angle between $\nabla f$ and $\mathbf{u}$. Since $\cos\theta$
  is maximized when $\theta = 0$, we see that \textbf{the gradient of
    $f$ at $(a,b)$ is $f$'s direction of maximum increase at
    $(a,b)$}. Furthermore, the direction opposite to the gradient is
  the direction of maximum decrease, and $f$ has zero derivative in
  any direction orthogonal to the gradient.

  \begin{example}{}{}
    Some of the level curves of a function $f(x,y)$ are shown. Sketch
    the direction of the gradient at the marked point.
    \begin{center}
      \begin{asy}[width=6cm]
      defaultpen(fontsize(10));
      import contour; 
            
      draw((0,0)--(2,0),Arrow());
      draw((0,0)--(0,2),Arrow());

      real f(real x, real y) {return (x-1)^2-(y-1)^2;}
      int n=10;
      real[] c=new real[n];
      for(int i=0; i < n; ++i) c[i]=(i-n/2)/n;
      
      srand(123); 
      
      pen[] p=sequence(new pen(int i) {
        return (c[i] >= 0 ? solid : gray)+fontsize(6pt);
      },c.length);
      
      Label[] Labels=sequence(new Label(int i) {
        return Label(c[i] != 0 ? (string) c[i] : "",Relative(unitrand()),(0,0),
        UnFill(1bp));
      },c.length);
      
      guide[][] g = contour(f,(0.1,0.1),(1.9,1.9),c); 
      
      draw(Labels,g,p);
      
      pair z = arcpoint(g[7][0],0.6); 
      dot(z, MidnightBlue+linewidth(4.0));
    \end{asy}
  \end{center}
\end{example}

\begin{solution}
  \begin{minipage}[t]{0.7\textwidth}
    The key idea here is that a function neither increases or
    decreases along its level curve. Therefore, $f$ has zero
    directional derivative in the direction of any line tangent the
    level curve passing through a given point. This means that the
    \textbf{gradient of $f$ is orthogonal to $f$'s level curve} at
    any given point. So the gradient looks like the figure shown
    (zoomed in).
  \end{minipage}
  \begin{minipage}[t]{0.29\textwidth}
    \begin{lrbox}{\asybox}
    \begin{asy}[width=4cm]
      defaultpen(fontsize(10));
      import contour; 
      
      size(0,8cm);
      
      draw((0,0)--(2,0),Arrow());
      draw((0,0)--(0,2),Arrow());

      real f(real x, real y) {return (x-1)^2-(y-1)^2;}
      int n=10;
      real[] c=new real[n];
      for(int i=0; i < n; ++i) c[i]=(i-n/2)/n;
      
      srand(123); 
      
      pen[] p=sequence(new pen(int i) {
        return (c[i] >= 0 ? solid : gray)+fontsize(6pt);
      },c.length);
      
      Label[] Labels=sequence(new Label(int i) {
        return Label(c[i] != 0 ? (string) c[i] : "",Relative(unitrand()),(0,0),
        UnFill(1bp));
      },c.length);
      
      guide[][] g = contour(f,(0.1,0.1),(1.9,1.9),c); 
      
      draw(Labels,g,p);
      
      pair z = arcpoint(g[7][0],0.6); 
      dot(z, MidnightBlue+linewidth(4.0));
      
      pair gradf(pair z) {return (2*z.x-2, 2-2*z.y);}
      real eps = 0.05 ;
      draw(z--z+eps*gradf(z), MidnightBlue+linewidth(1.0), Arrow(3));
      
      eps *= 3; 
      clip(box(z-(eps,eps),z+(eps,eps)));
    \end{asy}
  \end{lrbox} \raisebox{\dimexpr -\height + 1.5ex \relax}{\usebox{\asybox}}
\end{minipage}
\end{solution}

The gradient of a function from $\R^3$ to $\R$ is also defined to
  be the vector of partial derivatives*: 
  $\nabla f = \langle \partial_x f, \partial_y f, \partial_z f
  \rangle$. \sidenote{* More precisely, the gradient at each point is
  the vector of partial derivatives of $f$ at that point. Thus the
  gradient is actually a function from $\R^3$ to $\R^3$.} The formula
$D_{\mathbf{u}} f = \nabla f \cdot \mathbf{u}$ holds for
differentiable $f:\R^3 \to \R$ and $\mathbf{u} \in \R^3$. 

  \begin{example}{}{}
    Find the equation of a plane tangent to the ellipsoid $x^2 + y^2 +
    2z^2  = 4$ at the point $(1,1,1)$. 
  \end{example}

  \begin{solution}
    The ellipsoid is a level set of the function $f(x,y,z) = x^2 +
    y^2 + 2z^2$. Since the gradient of a function at a point is orthogonal to
    the level set of the function at that point, it follows that the
    vector 
    \[
      (\nabla f)(1,1,1) = \left.\langle 2x, 2y, 4z \rangle\right|_{(x,y,z) =
        (1,1,1)} = \langle 2, 2, 4 \rangle
    \]
    is orthogonal to the desired plane.  So a point $(x,y,z)$ is on the plane
    if and only if the vector from $(1,1,1)$ to $(x,y,z)$ is
    orthogonal to $\langle 2, 2, 4 \rangle$. Thus the equation of the
    plane is
    \[
      \langle 2, 2, 4 \rangle \cdot \langle x - 1, y - 1, z - 1
      \rangle = 0 \implies \boxed{x + y + 2z = 4}. 
    \]
  \end{solution}

  \begin{exercise}{}{}
    Show that if $f$ is a differentiable function with a local maximum
    at some point $p$ inside its domain, then $(D_{\mathbf{u}}f)(p) =
    0$ for any vector $\mathbf{u}$. 
  \end{exercise}

  \section{The multivariable chain rule} \label{sec:chainrule} 

  \milink{chain_rule_multivariable_introduction}{the chain rule}

  The basic idea of the chain rule is that when considering how
  $f(g(t))$ changes when we increase $t$ by some small amount $h$, we
  can note that $g(t)$ changes by approximately $hg'(t)$, and that
  change in the input to $f$ induces a change of
  \[
    \left(
      \overbrace{hg'(t)}^{\text{change in input to $f$}} 
    \right) \left( 
    \overbrace{f'(g(t))}^{\text{sensitivity of
        $f$ to change in input}}
    \right) 
  \]
  in the value of $f(g(t))$.

  The simplest multivariable generalization of this idea is make a
  function from $\R$ to $\R$ by composing function from $\R$ to $\R^2$
  with a function $f: \R^2 \to \R$. Let's look at an example.

  \begin{example}{}{chainrule}
    Suppose $f(x,y) = \sin xy \cos y$ and $\mathbf{r}(t) = (e^t,
    t^2)$. Find the derivative of $f \circ \mathbf{r}$. 
  \end{example}

  \begin{solution}
    We can calculate directly
    \[
      (f \circ \mathbf{r})(t) = f(\mathbf{r}(t)) = \sin (t^2e^t) \cos
      t^2. 
    \]
    So the desired derivative is
    \begin{align*}
      \cos(t^2e^t)\left[ t^2 e^t + 2te^t\right] &\cos t^2  -
      \sin(t^2 e^t) 2t \sin t^2  \\ &=
      t^2 e^t \cos(t^2 e^t) \cos t^2 + 2te^t \cos(t^2 e^t) \cos t^2 -
      2t\sin(t^2 e^t) \sin t^2. 
    \end{align*}
  \end{solution}

  The multivariable chain rule gives as an alternative approach which
  takes advantage of partial derivatives. Let's write
  $\mathbf{r}(t) = \langle r_1(t), r_2(t) \rangle$.  When we change
  $t$ by $h$, the value of $f(\mathbf{r}(t))$ changes as follows: 
  \[
    f\left(
      \overbrace{r_1(t)}^{\text{changes by $hr_1'(t)$}},
      \overbrace{r_2(t)}^{\text{changes by $hr_2'(t)$}}
    \right)
  \]
  The change of $hr_1'(t)$ in the first argument induces a change of
  $hr_1'(t) (\partial_xf)(\mathbf{r}(t))$ in the value of $f$, while
  the change of $hr_2'(t)$ in the second argument induces a change of
  $hr_2'(t) (\partial_yf)(\mathbf{r}(t))$. Dividing by $h$ and taking
  $h \to 0$ gives the following theorem. 
  
  \begin{theo}{Multivariable chain rule}{chainrule}
    If $f : \R^2 \to \R$ and $\mathbf{r} = \langle r_1,  r_2\rangle :
    \R \to \R^2$,
    are differentiable, then
    \begin{equation} \label{eq:chainrule} 
      (f\circ \mathbf{r})'(t) = (\partial_x f)(\mathbf{r}(t))r_1'(t) +
      (\partial_y f)(\mathbf{r}(t))r_2'(t). 
    \end{equation}
  \end{theo}

  You can write \eqref{eq:chainrule} using the more suggestive
  notation
  \[
    \frac{\d f}{\d t} = \frac{\partial f}{\partial x}\frac{\d x}{\d t} +
    \frac{\partial f}{\partial y}\frac{\d y}{\d t}, 
  \]
  where $x$ and $y$ represent $r_1$ and $r_2$. Although this formula
  is more memorable, it does involve some abuse of notation: the
  symbols $x$ and $y$ are being used* as independent variables (in the
  partial derivative expressions) and as function names (in $\d x/\d t$
  and $\d y/\d t$). Also, on the left-hand side $f$ looks like it's being
  treated as a function of a single variable; actually
  $f\circ \mathbf{r}$ is the single-variable function that one has in mind here.
  \sidenote{* using the same symbol to mean different things
    (relying on context to distinguish) is called
    \textit{overloading}}[-1cm]

  \begin{exercise}{}{}
    Verify that applying the multivariable chain rule to
    Example~\ref{exam:chainrule} gives the same result we found by calculating that
    derivative directly. 
  \end{exercise}

  \begin{exercise}{}{}
    Find the derivative with respect to $t$ of the function
    $g(t) = t^t$ by writing the function as $f(x(t),y(t))$ where
    $f(x,y) = x^y$ and $x(t) = t$ and $y(t)=t$.
  \end{exercise}

  
  \section{Optimization with Lagrange multipliers} \label{sec:lagrange} 

  \begin{wrapfigure}[10]{R}[1cm]{5cm}
    \begin{asy}[width=4cm]
      import graph3;
      defaultpen(fontsize(8)); 
      picture myaxes;
      draw(myaxes,O--1.2*X,Arrow3());
      draw(myaxes,O--1.2*Y,Arrow3());
      draw(myaxes,O--1.2*Z,Arrow3());
      
      currentprojection = perspective(5,2,2);
      
      triple f(pair p){ 
        real x = 0.5 + p.x*cos(p.y), y = 0.5 + p.x*sin(p.y); 
        return (x,y,1 - (x-1/2)^2 - (y-1/3)^2);
      }
      
      triple g(real t){
        return f((0.5,t)); 
      }

      add(myaxes); 

      surface s = surface(f,(0,0),(0.5,2*pi),20,Spline);
      draw(s,MidnightBlue+opacity(0.4),MidnightBlue);
      draw(graph(g,0,2*pi,100),MidnightBlue);
      
      path3 c = circle(c=(0.5,0.5,0),r=0.5,normal=Z); 

      triple h = (-0.1,0.1,1e-2);
      triple j = (h.x,-h.y,1e-2); 
      triple p = (0.7,0.7,1e-2);
      
      surface s = surface(rotate(-10,Z)*(p+h--p+j--p-h--p-j--cycle));
      draw(c); 
      draw(surface(c),white+opacity(0.2));
      draw(surface(xscale(-0.5)*yscale(0.5)*"$D$",s,0,0,1e-3),MidnightBlue); 

      label("$f(x,y)$", (0,0,1.1), MidnightBlue, align=2*E);
    \end{asy}
    \caption{The graph of a function $f$ defined on a disk
      $D$ \label{fig:lagrange}}
  \end{wrapfigure}
  
  Consider the function
  \[
    f(x,y) = -x^2 - y^2 + x + \frac{2}{3} y + \frac{23}{36}, 
  \]
  which we optimized over the square $[0,1]^2$ in
  Example~\ref{exam:optim2d}. In that case, we identified possible
  extreme values on the boundary of the square by doing a
  single-variable optimization along each edge of the square. But
  suppose that we want to find the maximum and minimum values of $f$
  over a disk $D$ (see Figure~\ref{fig:lagrange})? Let's consider the
  disk $D$ of radius $\tfrac{1}{2}$ centered at
  $\left(\tfrac{1}{2}, \tfrac{1}{2}\right)$.

  We could actually take a similar approach to this problem. We can
  parametrize* the boundary of the disk as \sidenote{* To
    \textit{parametrize} a curve means to find a path which traces it
    out}
  \[
    \mathbf{r}(t) = \left\langle \tfrac{1}{2} + \tfrac{1}{2} \cos t,
    \tfrac{1}{2} + \tfrac{1}{2} \sin
    t \right\rangle, \quad 0 \leq t 
    < 2\pi. 
  \]
  Then the single-variable function $t\mapsto f(\mathbf{r}(t))$ can be
  optimized over $[0,2\pi]$ using the standard single-variable technique (as in
  Example~\ref{exam:optim1d}).\sidenote{\href{https://cocalc.com/projects/7925f475-a0bd-4621-b78a-466c24c7863c/files/minimize_on_circle.ipynb}{\cocalc}
    The minimum is $0.\overline{5}$} 
  
  However, this approach is limited because it requires a
  parametrization of the boundary of $D$, which is not always
  convenient. Suppose that $\partial D$ is specified as a level set of
  some function $g: \R^2 \to \R$. For example, the
  circle in Figure~\ref{fig:lagrange} is a level set $\left\{(x,y) \, :
  g(x,y) = \tfrac{1}{2}\right\}$ of the function
  \[
    g(x,y) = \left(x-\tfrac{1}{2}\right)^2 + \left(y-\tfrac{1}{2}\right)^2. 
  \]
  Let's derive an approach to finding the extreme values on the
  boundary which begins with the functions $f$ (the \textit{objective}
  function) and $g$ (the \textit{constraint} function).

  \begin{wrapfigure}[18]{R}[1cm]{6cm}
    \begin{asy}[width=6cm]
      import graph3;
      defaultpen(fontsize(8)); 
      picture myaxes;
      draw(myaxes,O--1.2*X,Arrow3());
      draw(myaxes,O--1.2*Y,Arrow3());
      draw(myaxes,O--1.2*Z,Arrow3());
      
      currentprojection = perspective(5,2,2);
      
      triple f(pair p){ 
        real x = 0.5 + p.x*cos(p.y), y = 0.5 + p.x*sin(p.y); 
        return (x,y,1 - (x-1/2)^2 - (y-1/3)^2);
      }
      
      triple g(real t){
        return f((0.5,t)); 
      }

      add(myaxes); 

      surface s = surface(f,(0,0),(0.5,2*pi),20,Spline);
      draw(s,MidnightBlue+opacity(0.4),MidnightBlue);
      draw(graph(g,0,2*pi,100),MidnightBlue);
      
      path3 c = circle(c=(0.5,0.5,0),r=0.5,normal=Z); 

      triple h = (-0.1,0.1,1e-2);
      triple j = (h.x,-h.y,1e-2); 
      triple p = (0.7,0.7,1e-2);
      
      surface s = surface(rotate(-10,Z)*(p+h--p+j--p-h--p-j--cycle));
      draw(c); 
      draw(surface(c),white+opacity(0.2));
      draw(surface(xscale(-0.5)*yscale(0.5)*"$D$",s,0,0,1e-3),MidnightBlue); 

      label("$f(x,y)$", (0,0,1.1), MidnightBlue, align=2*E);
    
      real t = 0.5;
      pair gamma(real t){
        return (1/2 + 1/2*(cos(t)), 1/2 + 1/2*sin(t));
      }

      real fx(real t){
        real x = gamma(t).x; 
        return -2*x + 1; 
      }
      
      real fy(real t){
        real y = gamma(t).y ;
        return -2*y + 2/3; 
      }

      triple p(triple z){
        return (z.x,z.y,0); 
      }

      dot(f(t)); dot(p(f(t)));
      
      real eps = 0.1; 
      
      draw(f(t)--f(t)+eps*(fx(t),fy(t),fx(t)^2+fy(t)^2),MidnightBlue,Arrow3(5));

      draw("$\nabla f(p)$",p(f(t))--p(f(t)+eps*(fx(t),fy(t),fx(t)^2+fy(t)^2)),Arrow3(4),align=2*SSW);
      
      real t = 1.5707963256474227; 
      dot(g(t));
      dot(p(g(t)));
      draw(g(t)--g(t)+eps*(fx(t),fy(t),fx(t)^2+fy(t)^2),MidnightBlue,Arrow3(5));
      draw("$\nabla
      f(q)$",p(g(t))--p(g(t)+eps*(fx(t),fy(t),fx(t)^2+fy(t)^2)),Arrow3(4),align=3*SE);
    \end{asy}
    \caption{$q$ is a boundary critical point and $p$ is not \label{fig:lagrange2}}
  \end{wrapfigure}
  
  Imagine a bug moving around on the edge of the graph in
  Figure~\ref{fig:lagrange}. How can it tell that it is \textit{not}
  at a maximum or minimum? One approach is to calculate the gradient
  of $f$ at its location. If the gradient of $f$ is not orthogonal to
  $\partial D$, then the value of the function can be increased by
  sliding a bit in one direction* and can be decreased by sliding a
  bit in the opposite direction. So, for example, in
  Figure~\ref{fig:lagrange2}, a bug at the point $p$ could increase
  the value of $f$ at its location by moving slightly clockwise and
  decrease the value of $f$ by moving slightly counterclockwise around
  $\partial D$. \sidenote{* Specifically, the direction where $\nabla
    f$ is leaning (that is, the direction whose dot product with $\nabla f$
    is positive)}[-1cm]

  Therefore, if the gradient of $f$ at a point is \textit{not}
  orthogonal to $\partial D$, then $f$ does not have an extreme
  value there. So to find points where $f$ might have an extreme value
  on $\partial D$, we can restrict our attention to the points where
  $\partial f$ is orthogonal to $\partial D$.

  We can simplify this idea further: recall that the gradient of $g$ at
  each point is orthogonal to the level set of $g$ passing through
  that point. It follows that if $\partial D$ is a level set of $g$ 
  and $p \in \partial D$ is a point where $f$ has an extreme value, then
  $\nabla g$ and $\nabla f$ are both orthogonal to $\partial
  D$. This means that they are parallel! By
  Observation~\ref{obs:vecparallel}, this means that there exist a scalar
  $\lambda$ such that $\nabla f = \lambda \nabla g$.

  \begin{theo}{Method of Lagrange multipliers}{lagrange}
    Suppose $f: \R^n \to \R$ and $g:\R^n \to \R$ are differentiable
    functions and $c \in \R$. If the restriction of $f$ to the level
    set $\{\mathbf{x} \in \R^n \,: g(\mathbf{x}) = c\}$ has a local
    extremum at $\mathbf{x} \in \R^n$, then
    \[
      \nabla f (\mathbf{x}) = \lambda \nabla g (\mathbf{x}). 
    \]
  \end{theo} \sidenote{* So far we've been considering the
    restriction of $f$ to the boundary of a region $D$, but the region
    $D$; any level set of a differential function $g$ will do}[-2cm]

  Let's see how this works in practice.

  \begin{example}{}{lagrange}
    Find the maximum and minimum values of
    \[f(x,y) = -x^2 - y^2 + x + \frac{2}{3} y + \frac{23}{36}\]
    over the disk of radius $\tfrac{1}{2}$ centered at $\left(
      \tfrac{1}{2}, \tfrac{1}{2}\right)$. 
  \end{example}

  \begin{solution}
    The only interior critical point is
    $\left(\tfrac{1}{2},\tfrac{1}{3}\right)$, as in
    Example~\ref{exam:optim2d}. To find boundary critical points, we
    set up the Lagrange equations:
    \begin{align} \nonumber
      \partial_x f &= \lambda \, \partial_x g \\ \nonumber
      \partial_y f &= \lambda \, \partial_y g \implies \\ 
      -2x + 1 &= \lambda(2x - 1)  \\ \label{eq:eq2}
      -2y + \tfrac{2}{3} &= \lambda(2y-1). 
    \end{align}
    We're looking for pairs $(x,y)$ which satisfy both of these
    equations \textbf{and} the equation
    \begin{equation} 
          g(x,y) = \left(x-\tfrac{1}{2}\right)^2 +
          \left(y-\tfrac{1}{2}\right)^2 = \tfrac{1}{4} \\ 
    \end{equation}
    since
    $(x,y)$ must be on $\partial D$. So have three equations and three
    variables: $x$, $y$, and $\lambda$. The first equation implies
    that either $\lambda = -1$ or $x = \tfrac{1}{2}$.

    In the case $x = \frac{1}{2}$, we can use the constraint equation
    to conclude that $y = 0$ or $y = 1$. In either case, we can
    substitute into \eqref{eq:eq2} to get a value for $\lambda$ so
    that all three equations are satisfied. So, we have
    $(x,y) = (1/2, 0)$ and $(1/2,1)$ as boundary critical points.

    If $x \neq \tfrac{1}{2}$, then $\lambda = -1$. Substituting into
    \eqref{eq:eq2} gives a contradiction, which means that we've
    already found all the boundary critical points.

    Finally, we evaluate $f$ at the interior critical point and the
    two boundary critical points:
    \begin{center}
      \begin{tabular}{M{1.2cm}|M{1.2cm}M{1.2cm}M{2cm}N}
        $(x,y)$ & $\left(\frac{1}{2}, 0\right)$  & $\left(\frac{1}{2},
                                1\right)$ &
                                            $\left(\frac{1}{2},
                                            \frac{1}{3} \right)$ &
        \\[12pt] \hline  \\[-8pt] 
        $f(x,y)$ & $\frac{8}{9}$ & $\frac{5}{9}$ & 1 &  \\
      \end{tabular}
    \end{center}
    So the maximum of $f$ over $D$ is $1$, and the minimum is
    $\frac{5}{9}$. 
  \end{solution}

  The following example is a 3D application of Lagrange multipliers. 
  
  \begin{example}{}{lagrange3d}
    Find the maximum possible volume of a box made with 72 square
    centimeters of cardboard and having sides and a bottom but no
    top. 
  \end{example}

  \begin{solution}
    Denote by $x,y$, and $z$ the dimensions (in centimeters) of the
    cardboard. Then the amount of cardboard used is
    \[
      g(x,y,z) = 2yz + 2xz + xy = 72, 
    \]
    while the objective function is the volume $f(x,y,z) =
    xyz$. Setting up the Lagrange equations, we get
    \begin{align*}
      yz &= \lambda(2z + y) \\
      xz &= \lambda(2z + x) \\
      xy &= \lambda(2x + 2y) \\
      2yz + 2xz + xy &= 72, 
    \end{align*}
    where the last one is the constraint equation. Multiplying the
    first two equations by $x$ and $y$, respectively, and the setting
    the resulting right-hand sides equal implies that either $\lambda
    = 0$ or $z= 0$ or  $x = y$. Since $\lambda = 0$ or $z = 0$ clearly
    give zero volume (and thus not the maximum volume), it follows
    that $x=y$. Substituting $y$ for $x$ in the third equation gives*
    \sidenote{* Once again, we can divide by $x$ because we know
      that $ x= 0$ wouldn't make sense for the maximum volume. }
    \[
      x^2 = 4 \lambda x \implies \lambda = \tfrac{x}{4}. 
    \]
    Substituting this into the second equation and simplifying, we get
    $x = 2z$. Finally, substituting into the constraint equation gives
    $z = \sqrt{6}$, which in turn implies $x = y = 2\sqrt{6}$. 
  \end{solution}

  \begin{exercise}{}{}
    Find the points on the ellipse $\left(\frac{x - 1}{2}\right)^2 +
    (y-2)^2 = 1$ which are nearest and farthest from the origin. Hint:
    for the objective function, use \textit{squared} distance rather
    than distance. 
  \end{exercise}
  
  \chapter{Multivariable Integration}

  To find the area under the graph of a continuous function $f$ over
  the unit interval $[0,1]$, we first approximate the area by
  splitting $[0,1]$ into many short intervals and sum up the areas of
  rectangles approximating the area under the graph over each short
  interval:
  \begin{center}
    \begin{asy}
      defaultpen(fontsize(10)); 
      size(12cm);
      import graph;
      real f(real x){ return 1/4*x*x;}
      picture gr; 
      draw(gr,graph(f,0,2),MidnightBlue);
      draw(gr,(0,0)--(2,0));
      draw(gr,(0,0)--(0,1));
      real eps = 0.05; 
      draw(gr,Label("1",Relative(1),align=S),(2,0)--(2,-eps)); 
      label(gr,"$x$",(2,0),align=E,MidnightBlue);
      label(gr,"$f(x)$",(0,1.05),align=N,MidnightBlue);
      
      int n = 15;
      add(gr); 
      for(int i=0;i< 2*n; ++i){
        filldraw(box((i/n,0),((i+1)/n,f(i/n))),softgreen,MidnightBlue); 
      }
      label("$\displaystyle{\sum_{i=0}^{n-1} f\left(\frac{i}{n}\right) \frac{1}{n}}$", (0.6,0.6)); 
      add(shift((3,0))*gr);
      draw((2.3,1/2)--(2.7,1/2),MidnightBlue,Arrow(3));
      label("$\displaystyle{\int_0^1 f(x) \, \mathrm{d}x}$", (3.6,0.6)); 
      filldraw(shift((3,0))*(graph(f,0,2)--(2,0)--cycle),softgreen,MidnightBlue);
      real eps = 0.05;
    \end{asy}
  \end{center}
  This approximation converges to the actual area under the graph as
  $n \to \infty$.

  In this section we will work out how to generalize this concept
  to integrate of functions of multiple variables over regions in
  $\R^2$ or $\R^3$.* \sidenote{* See Appendix~\ref{sec:centralidea} for a
    more in-depth discussion}
  
  \section{Double integration} \label{sec:double}

  \milink{double_integral_introduction}{Double integrals} 
  
  We can state the definition of the integral, described above, more
  informally and generally: split the region of integration into many
  tiny pieces, multiply the volume* of each piece by the value of the
  function at some point on that piece*, and add up the results. If we
  take the number of pieces to $\infty$ and the piece size to zero,
  then this sum should converge to a number, and if it does then we
  declare that number to be the value of the integral.
  \sidenote{* Recall our convention that 1D volume is length and 2D
    volume is area.}[-1.35cm] \sidenote{* It doesn't ultimately matter
    where we evaluate the function, since the piece is very
    small and the function is continous}[1cm]
  
  Stated at this level of generality, the definition of the integral
  applies to a function $f: \R^2 \to \R$ over a region $D
  \subset \R^2$. See Figure~\ref{fig:integraldef}. 
  \begin{defn}{Integral over a 2D region}{twoDintegral}
    Suppose $f: \R^2 \to \R$ is a continuous function, and that $D$ is a region in
    $\R^2$. Then the integral of $f$ over $D$, denoted*
    $\iint_D f \, \d A$, is defined to be
    \begin{equation} \label{eq:def2D}
      \iint_D f \, \d A = \lim_{n\to\infty} \sum_{(i,j) \, : \left(\frac{i}{n,} \,\frac{j}{n}\right) \in D}
      f\left(\frac{i}{n,}\,\frac{j}{n}\right)
      \overbrace{\frac{1}{n^2}}^{\Delta A} \:,  
    \end{equation}
    where the sum includes one term for each integer pair $(i,j)$ such
    that $(i/n,j/n)$ is in the region $D$.   \sidenote{* The $A$ in $\d A$ stands for area.}[-25mm]
  \end{defn}

  \begin{figure}
    \centering
    \begin{asy}[width=0.3\textwidth]
      import graph3;
      defaultpen(fontsize(8)); 
      picture myaxes;
      draw(myaxes,O--1.2*X,Arrow3());
      draw(myaxes,O--1.2*Y,Arrow3());
      draw(myaxes,O--1.2*Z,Arrow3());
      
      currentprojection = perspective(5,2,2);
      
      triple f(pair p){ 
        real x = 0.5 + p.x*cos(p.y), y = 0.5 + p.x*sin(p.y); 
        return (x,y,1 - (x-1/2)^2 - (y-1/3)^2);
      }
      
      triple g(real t){
        return f((0.5,t)); 
      }

      add(myaxes); 

      surface s = surface(f,(0,0),(0.5,2*pi),20,Spline);
      draw(s,MidnightBlue+opacity(0.4),MidnightBlue);
      draw(graph(g,0,2*pi,100),MidnightBlue);
      
      path3 c = circle(c=(0.5,0.5,0),r=0.5,normal=Z); 

      triple h = (-0.1,0.1,1e-2);
      triple j = (h.x,-h.y,1e-2); 
      triple p = (0.7,0.7,1e-2);
      
      surface s = surface(rotate(-10,Z)*(p+h--p+j--p-h--p-j--cycle));
      draw(c); 
      draw(surface(c),white+opacity(0.2));
      draw(surface(xscale(-0.5)*yscale(0.5)*"$D$",s,0,0,1e-3),MidnightBlue); 

      label("$f(x,y)$", (0,0,1.1), MidnightBlue, align=2*E);
    \end{asy}
    \begin{asy}[width=0.3\textwidth]
      import graph3; 
      picture myaxes;
      draw(myaxes,O--1.2*X,Arrow3());
      draw(myaxes,O--1.2*Y,Arrow3());
      draw(myaxes,O--1.2*Z,Arrow3());
      
      currentprojection = perspective(5,2,2);

      triple f(pair p){ 
        real x = 0.5 + p.x*cos(p.y), y = 0.5 + p.x*sin(p.y); 
        return (x,y,1 - (x-1/2)^2 - (y-1/3)^2);
      }

      real h(real x, real y){
        return 1 - (x-1/2)^2 - (y-1/3)^2 ;
      }
      
      add(myaxes); 

      int n = 10;
      surface mybox(triple a, triple b){
        triple delta = b - a; 
        return shift(a.x,a.y,a.z)*scale(delta.x,delta.y,delta.z)*unitcube; 
      }
      
      for(int i=0;i<n;++i){
        for(int j=0;j<n;++j){
          if ((i/n - 1/2)^2 + (j/n - 1/2)^2 < 1/4){
            draw(mybox((i/n,j/n,0),((i+1)/n,(j+1)/n,h(i/n,j/n))),softgreen+opacity(0.5), MidnightBlue+linewidth(0.7));
          }
        }
      }
      path3 c = circle(c=(0.5,0.5,0),r=0.5,normal=Z); 
      triple h = (-0.1,0.1,1e-2);
      triple j = (h.x,-h.y,1e-2); 
      triple p = (0.7,0.7,1e-2);
      surface s = surface(rotate(-10,Z)*(p+h--p+j--p-h--p-j--cycle));
      draw(c); 
      draw(surface(c),white+opacity(0.2));
      label("$f(x,y)$", (0,0,1.1), MidnightBlue, align=2*E);
    \end{asy}
    \begin{asy}[width=0.3\textwidth, inline=false]
      import graph3; 
      picture myaxes;
      draw(myaxes,O--1.2*X,Arrow3());
      draw(myaxes,O--1.2*Y,Arrow3());
      draw(myaxes,O--1.2*Z,Arrow3());
      
      currentprojection = perspective(5,2,2);
      
      triple f(pair p){ 
        real x = 0.5 + p.x*cos(p.y), y = 0.5 + p.x*sin(p.y); 
        return (x,y,1 - (x-1/2)^2 - (y-1/3)^2);
      }

      real h(real x, real y){
        return 1 - (x-1/2)^2 - (y-1/3)^2 ;
      }
      
      add(myaxes); 
      
      int n = 20;
      
      surface mybox(triple a, triple b){
        triple delta = b - a; 
        return shift(a.x,a.y,a.z)*scale(delta.x,delta.y,delta.z)*unitcube; 
      }
      
      for(int i=0;i<n;++i){
        for(int j=0;j<n;++j){
          if ((i/n - 1/2)^2 + (j/n - 1/2)^2 < 1/4){
            draw(mybox((i/n,j/n,0),((i+1)/n,(j+1)/n,h(i/n,j/n))),softgreen+opacity(0.3), MidnightBlue+linewidth(0.3));
          }
        }
      }
      
      path3 c = circle(c=(0.5,0.5,0),r=0.5,normal=Z); 
      
      triple h = (-0.1,0.1,1e-2);
      triple j = (h.x,-h.y,1e-2); 
      triple p = (0.7,0.7,1e-2);
      
      surface s = surface(rotate(-10,Z)*(p+h--p+j--p-h--p-j--cycle));
      draw(c); 
      draw(surface(c),white+opacity(0.2));
      label("$f(x,y)$", (0,0,1.1), MidnightBlue, align=2*E);
    \end{asy}
    \caption{The integral of $f$ over a disk $D$, defined as a
      limit of sums of volumes of narrow boxes}
    \label{fig:integraldef}
  \end{figure}
  
  As in the single-variable case, this Riemann-sum definition is not
  generally practical for precise evaluation of integrals. The
  fundamental theorem of calculus* is the primary tool for evaluating
  integrals in single-variable calculus, and fortunately we can
  bootstrap our way up from 1D integrals to 2D integrals by applying
  our primary strategy for tackling higher dimensional problems: slice
  it up. Let's start by considering integrals over rectangular regions
  $D$. \sidenote{* The fundamental theorem of calculus says that the
    integral of $f$ from $a$ to $b$ is equal to $F(b)-F(a)$ where $F$
    is an antiderivative of $f$}[-1cm]

  \begin{example}{}{integral2d}
    Find the integral of $f(x,y) = y \sin(\pi x y)$ over the square $[0,1]^2$. 
  \end{example}

  \begin{solution}
    \begin{minipage}{0.64\textwidth}
      Let's slice up the desired solid using many
      `$y = \text{constant}$' cuts, producing many thin slices like the
      one shown. The volume of one of these slices, situated at a
      particular $y$-value, is given by* the thickness $\Delta y$ times
      the area $A(y)$ under the graph of the single-variable function
      $x\mapsto f(x,y)$. So we can use the fundamental theorem of
      calculus to compute \sidenote{* $\ldots$ignoring an error, having to
        do with the top of the slice not being flat---this error tends
        to zero as the number of slices tends to infinity}[-1cm]
      \begin{align*}
        A(y) &= \int_0^1 y \sin (\pi xy) \, \d x = 
               \left.-\frac{\cos\left(\pi x y\right)}{\pi}\right|_0^1 \\
             &= \frac{1- \cos \pi y }{\pi}. 
      \end{align*}
      Once we have each area $A(y)\Delta y$, we can add them all up and
      take $\Delta y \to 0$ (as the number of slices goes to $\infty$) to
      find that the desired volume is
    \end{minipage}
    \begin{minipage}{0.35\textwidth}
      \begin{asy}[width=5cm]
        import graph3; 
        import multitools;
        picture myaxes;
        draw(myaxes,O--1.2*X,Arrow3());
        draw(myaxes,O--1.2*Y,Arrow3());
        draw(myaxes,O--1.2*Z,Arrow3());

        add(myaxes); 

        currentprojection = perspective(5,3,2);
        currentlight.background = softyellow; 
        
        real f(pair p){ 
          real x = p.x, y = p.y; 
          return y*sin(pi*x*y); 
        }
        
        // slab starting and ending y-values
        real lo = 0.41;
        real hi = 0.45; 
        
        // draw slab
        undersurface(f,
        new pair[] {(0,lo),(0,hi),(1,hi),(1,lo),(0,lo)},
        LightSeaGreen+opacity(0.3),
        MidnightBlue);
        
        // draw graph of f
        surface s = surface(f,(0,0),(1,1),20,Spline);
        draw(s,MidnightBlue+opacity(0.4),MidnightBlue);
        
        // label front face
        real r = 0.2; // top edge of A(y) label
        real u = 0.70; // right edge of A(y) label
        real v = 0.08; // bottom edge of A(y) label 
        surface t = surface((u,hi,v)--(1,hi,v)--(1,hi,r)--(u,hi,r)--cycle);
        draw(surface(xscale(-0.75)*"$A(y)$", t, 0, 0.0, 2e-3));
        
        // draw base surface
        draw(surface((0,0)--(1,0)--(1,1)--(0,1)--cycle),white+opacity(0.2));

        // Text labels
        real eps = 0.02; 
        label("$f(x,y) = y \sin (\pi xy)$", (0,0,1), MidnightBlue, align=2*E);
        draw("$\Delta y$", (1+eps,lo,0.0)--(1+eps,hi,0.0),Bars3(2,dir=X),align=SSW);
        draw((1.08,0.5*(lo+hi),0)--(1.04,0.5*(lo+hi),0),Arrow3(2));
        draw("$y$",(1+eps,eps,0)--(1+eps,lo-eps,0),Bars3(2,dir=X),align=2*SW); 
      \end{asy}
    \end{minipage}
    
    \[
      \sum_{\text{all slices}} A(y) \Delta y \to \int_{0}^1 A(y) \,
      \d y. 
    \]
    We can again evaluate this integral using the fundamental theorem to get
    a final answer of $\boxed{\frac{1}{\pi}}$. \sidenote{ 
      \href{https://cocalc.com/projects/7925f475-a0bd-4621-b78a-466c24c7863c/files/riemann_sum.ipynb}{\cocalc} For some
      confidence that our answer is reasonable, we can calculate a
      Riemann sum for this integrand.  
    } 
\end{solution}

  We can express this process more succinctly as
  \begin{equation} \label{eq:iteratedintegral} 
    \int_0^1 \int_0^1 y \sin (\pi xy) \, \d x \, \d y =
    \int_0^1 \frac{1-\cos \pi y}{\pi}\, \d y = \frac{1}{\pi}. 
  \end{equation}
  The first expression in \eqref{eq:iteratedintegral} is called an
  \textbf{iterated integral}, since it expresses an integral over a 2D
  region in terms of two successive single-variable integrals.

  Let's see how this works over a non-rectangular region.

  \begin{example}{}{integraltriangle}
    Find the integral over the triangle $T$ with vertices $(0,0)$,
    $(2,0)$, and $(0,3)$ of the function $f(x,y) = x^2y$, by first finding the area
    under each `$y=\text{constant}$' slice.
  \end{example}

  \begin{solution}
    \begin{minipage}{0.65\textwidth}
      As in the previous example, we slice up the desired
      volume making many `$y=\text{constant}$' cuts of thickness
      $\Delta y$, yielding thin slices such that each one has volume
      (very close to) $A(y)\Delta y$, where $y$ is the slice's signed
      distance from the $xz$-plane and $A(y)$ is the area of the
      cross-section (see figure). Since this cross section is an area
      under a curve, we can find it by integrating $x\mapsto f(x,y)$
      over the set of relevant $x$-values:* \sidenote{* We can find
        the formula $2 - \tfrac{2}{3}y$ by writing an equation for the
        line connecting $(2,0)$ to $(0,3)$ and solving for $x$}
      \[
        A(y) = \int_{0}^{2 - \frac{2}{3} y} f(x,y) \, \d x. 
      \]
      Thus
      $A(y) = \frac{1}{3}\left( 2- \frac{2}{3} y\right)^3 y
      $. 
    Finally, adding up all these areas and taking $\Delta y \to 0$
    gives the result 
    \[
      \int_0^3 A(y) \, \d y = \int_0^3 \left(-\frac{8}{81} \, y^{4} +
        \frac{8}{9} \, y^{3} - \frac{8}{3} \, y^{2} + \frac{8}{3} \, y \right)\,
      \d y = \boxed{\frac{6}{5}}. 
    \]
  \end{minipage}
  \begin{minipage}{0.34\textwidth}
    \begin{asy}[width=5cm]
      import graph3; 
      import multitools; 
      picture myaxes;
      draw(myaxes,O--2.2*X,Arrow3());
      draw(myaxes,O--3.2*Y,Arrow3());
      draw(myaxes,O--2.2*Z,Arrow3());
      
      currentprojection = perspective(6,3,2.25);
      currentlight.background = softyellow; 
 
      real f(pair p){ 
        real x = p.x, y = p.y; 
        return x^2 * y; 
      }
      
      real lo = 1.42;
      real hi = lo + 0.04; 
      
      add(myaxes);
      
      triple f2(pair p){
        real x = p.x, y = p.y; 
        return (x*(3-y)/3,y,f((x*(3-y)/3,y))); 
      }
      
      surface s = surface(f2,(0,0),(2,2.99),20,Spline);
      draw(s,MidnightBlue+opacity(0.4),MidnightBlue);
      
      // draw bottom shadow
      draw(surface((0,0)--(2,0)--(0,3)--cycle),white+opacity(0.2));

      // draw slab
      real delta = 0.125; 
      for(lo = 0; lo < 2.99; lo += delta){
        real  hi = lo + delta; 
        real xx(real y) {return 2 - 2*y/3;}
        undersurface(f,
        new pair[] {(0,hi), (xx(hi),hi), (xx(lo),lo), (0,lo), (0,hi)},
        LightSeaGreen+opacity(0.2),
        MidnightBlue);
      }
      
      // Text labels
      
      label("$f(x,y) = x^2 y$", (0,0,2.0), MidnightBlue, align=2*E);
      real eps = 0.05; 
      draw("3",(0,3,0)--(0,3,-eps),align=S);
      draw("2",(2,0,0)--(2,0,-eps),align=SSE);
    \end{asy}

    \begin{asy}[width=5cm]
      import graph3; 
      import multitools; 
      picture myaxes;
      draw(myaxes,O--2.2*X,Arrow3());
      draw(myaxes,O--3.2*Y,Arrow3());
      draw(myaxes,O--2.2*Z,Arrow3());
      
      currentprojection = perspective(6,3,2.25);
      currentlight.background = softyellow; 
      
      real f(pair p){ 
        real x = p.x, y = p.y; 
        return x^2 * y; 
      }
      
      real lo = 0.125*11;
      real hi = lo + 0.125; 
      
      add(myaxes);
      
      triple f2(pair p){
        real x = p.x, y = p.y; 
        return (x*(3-y)/3,y,f((x*(3-y)/3,y))); 
      }
      
      surface s = surface(f2,(0,0),(2,2.99),20,Spline);
      draw(s,MidnightBlue+opacity(0.4),MidnightBlue);
      
      // draw bottom shadow
      draw(surface((0,0)--(2,0)--(0,3)--cycle),white+opacity(0.2));
      
      // draw slab
      
      real xx(real y) {return 2 - 2*y/3;}
      undersurface(f,
      new pair[] {(0,hi), (xx(hi),hi), (xx(lo),lo), (0,lo), (0,hi)},
      LightSeaGreen+opacity(0.4),
      MidnightBlue);
      
      // Text labels
      real eps = 0.05; 
      //label("$f(x,y) = y \sin (\pi xy)$", (0,0,2.0), MidnightBlue, align=2*E);
      draw("$\Delta y$", (xx(lo)+2*eps,lo,0.0)--(xx(lo)+2*eps,hi,0.0),Bars3(4,dir=X),align=SSW);
      draw("$2-\frac{2}{3}y$", (0.5+eps,hi,0.0)--(0.5+eps,hi,0.0),align=3*ESE);
      real eps = 0.03; 
      draw((2*eps,hi+eps,0)--(2-2/3*(hi+eps),hi+eps,0),Bars3(4,dir=Y)); 
      draw((1-1/3*(hi+eps)+0.16,hi+eps+0.15,0)--
      (0.6*(2-2/3*(hi+eps)),hi+eps+0.01,0),Arrow3(3)); // arrow from 2 - 2/3*y
      draw((1.38,0.5*(lo+hi),0)--(xx(lo)+4*eps,0.5*(lo+hi),0),Arrow3(3)); // arrow from delta y
      draw("$y$",(xx(lo)+2*eps,eps,0)--(xx(lo)+2*eps,lo-eps,0),Bars3(4,dir=X),align=SW);
      
      real eps = 0.05; 
      draw("3",(0,3,0)--(0,3,-eps),align=S);
      draw("2",(2,0,0)--(2,0,-eps),align=SSE);
    \end{asy}
  \end{minipage}
\end{solution}

  Let's summarize what we figured out in
  Example~\ref{exam:integraltriangle}. 
  \begin{theo}{Iterated integrals for two-variable functions}{iterated}
    \begin{minipage}{0.7\textwidth}
      Suppose that
      \begin{itemize}[itemsep = 6pt]
      \item $D$ is a region in $\R^2$,
      \item 
        $f: D \to \R$ is a continuous function, and
      \item for all
      $y\in \R$, the intersection of $D$ and horizontal line through
      $(0,y)$ is a segment $[c(y),d(y)] \times \{y\}$.
    \end{itemize}
    Then
    \[
      \iint_D f \, {\d}A = \int_a^b \int_{c(y)}^{d(y)} f(x,y) \, \d x \,
      \d y. 
    \]
  \end{minipage}
  \begin{minipage}{0.29\textwidth}
    \begin{asy}[width=5cm] 
    import graph;
    defaultpen(fontsize(8)); 
    currentlight.background = softred; 

    real eps = 0.1;
    real x = 4.0, y = 3.5; 

    draw((-eps,0)--(x,0),Arrow()) ;
    draw((0,-eps)--(0,y),Arrow()) ;
    
    path p = (1,1){N}..(1,2)..{NE}(2,3);
    path q = (3,1){N}..(3,2)..{W}(2,3);
    path r = (0,1.8)--(4,1.8);
    path s = intersectionpoint(p,r)--intersectionpoint(q,r);
    
    draw(r,dashed);
    
    draw("$y$",(3.2,eps/2)--(3.2,1.8-eps/2),Bars,align=E); 
    
    fill(p--reverse(q)--cycle,0.9*softred+0.2*red);
    
    label("$D$",(2,2.7),0.6*softred+0.2*red);
    
    real eps = 0.1; 
    draw("$a$",(0,1)--(eps,1),align=W);
    draw("$b$",(0,3)--(eps,3),align=W); 
    
    draw(p); 
    draw(q);
    
    draw((1,1)--(3,1));

    draw("$[c(y),d(y)] \times \{y\}$",
    s,
    MidnightBlue+linewidth(1.0));
  \end{asy}
\end{minipage}
\end{theo}

In light of
Theorem~\ref{th:iterated}, we sometimes write the area differential*
as $\d A = \d x \, \d y$. We can describe the procedure in
Theorem~\ref{th:iterated} more casually: \sidenote{* Think of $\d A$ merely as a reminder that the positive
  quantity $\Delta A$ involved in the corresponding Riemann sums
  represents an area}[-1cm]
\begin{obs}{Limits of integration over a 2D
    region}{evaluating_integrals}
      To set up an iterated integral to evaluate $\iint_D f \, {\d}A$ (where $f$ is continuous and $D$ is a
    region such that the intersection of every horizontal line with
    $D$ is a segment): 
  \begin{enumerate}
\item Find the least and greatest $y$ values for any point in
  $D$. These are your \textbf{outer limits} of integration. 
\item For each fixed horizontal line which intersects $D$, identify
  the least and greatest values of $x$ for any point which is in $D$
  \textit{and on that line}, expressed in terms of the vertical
  position $y$ of the line. These are the \textbf{inner limits} of
  integration, and they may depend on $y$.
\end{enumerate}
\end{obs}

The role of $x$ and $y$ in Observation~\ref{obs:evaluating_integrals} can
be reversed (in which case we have vertical rather than horizontal
lines in Step 2). The following exercise shows how this can be useful. 

\begin{exercise}{}{}
  Find
  \[
    \int_0^{1/2} \int_{2y}^1 4e^{x^2}\, \d x\, \d y
  \]
  by first rewriting it as an integral over a 2D
  region and then reversing the order of integration.
\end{exercise}


\section{Triple integration} \label{sec:triple} 

\milink{triple_integral_introduction}{triple integrals}

We interpret the integral of a single-variable function as an area and
the integral of a two-variable function as a volume. So how should we
interpret the integral of a function of \textit{three} variables over
a region $D$ in $\R^3$?  \textit{Four-dimensional volume} is a reasonable
answer, but of course this is unsatisfactory from a visualization
point of view, since we don't have access to four spatial dimensions
with which to visualize.

Therefore, let's consider a physics interpretation of integration
which permits a visualization \textit{not} involving the graph of the
function being integrated.

\begin{example}{}{}
  Consider a square plate occupying the square $[1,3]^2$ whose density
  at each point is $\sigma(x,y) = xy$ kilograms per square meter.* Find
  the mass of the plate. \sidenote{* See the figure in the solution,
  where darker color indicates a denser portion of the plate}
\end{example}

\begin{solution}
  \begin{minipage}{0.7\textwidth}
    Let's imagine physically cutting the plate into small
    squares, computing the mass of each one, and adding up the
    resulting masses. The mass of a small plate of area
    $\Delta x \Delta y$ containing the point $(x,y)$ is approximately
    the area density times the area:
    $\sigma(x,y) \, \Delta x \, \Delta y$. The sum of these masses is
    a Riemann sum (see Definition~\ref{defn:twoDintegral})
    which converges as the number of small squares goes to $\infty$ to
    the integral 
    \[
      \int_1^3 \int_1^3 xy \, \d x \, \d y = \boxed{8} \text{ kilograms}. 
    \]
    \end{minipage} 
    \begin{minipage}{0.29\textwidth} 
      \begin{asy}[width=4.5cm]
        import graph; 
        import palette; 
        
        currentlight.background = softyellow; 
        
        real eps = 0.1;
        real x = 4.0, y = 4.0; 
        
        draw((-eps,0)--(x,0),Arrow()) ;
        draw((0,-eps)--(0,y),Arrow()) ;
        
        pen[] Palette = Gradient(100 ... new pen[] {0.8*white+0.2*Gold, 0.8*Gold});
        
        real f(real x, real y) {return x*y;}
        
        image(f, Automatic, (1,1), (3,3), 100, Palette); 
        
        draw(box((1,1),(3,3)));
        
        for(real x = 1.0; x <= 3.0; x += 0.2){
          draw((x,1)--(x,3));
          draw((1,x)--(3,x)); 
        }
        
        real eps = 0.1; 
        draw("$1$",(0,1)--(eps,1),align=2*W);
        draw("$3$",(0,3)--(eps,3),align=2*W);
        draw("$1$",(1,0)--(1,eps),align=2*S);
        draw("$3$",(3,0)--(3,eps),align=2*S); 
      \end{asy}
    \end{minipage}
  \end{solution}

Let's do a three-dimensional example.

\begin{example}{}{}
  Consider a cubical block occupying $D = [1,2]^3$ whose density at each
  point is $\rho(x,y) = x^2 + y^2 + z^2$ kilograms per cubic meter.*
  Find the mass of the block. 
\end{example}

\begin{solution}
  \begin{minipage}{0.65\textwidth} 
  We cut the cube into $n^3$ small cubes, where $n$ is a large
    integer. The mass one of these cubes with bottom, back* corner
    $(x,y,z)$ is approximately equal to the product of its volume
    $\tfrac{1}{n^3}$ and the approximate density $\rho(x,y,z)$
    throughout the small cube. So the approximate volume is
    \sidenote{* Any corner, or indeed any point in the cube, would
      give the same result---we choose the bottom, back corner (the
      point nearest the origin) for concreteness}[-1cm]
    \[
      \sum_{\text{all cubes}} \rho\left(x,y,z\right) \frac{1}{n^3}. 
    \]
    Intuitively, this sum should converge to a limit as $n\to\infty$,
    and if so, then we should define the limiting value to be the
    integral of $\rho$ over $D$. Let's state this idea for any
    continuous function $f:\R^3 \to \R$: we define the integral of $f$ over $D$
    by 
    \[
      \iiint_D f(x,y,z) \, {\d}V = \lim_{n\to\infty} \sum_{(i,\,j,\,k) \,:
        \left(\frac{i}{n,} \, \frac{j}{n,} \, \frac{k}{n} \right) \in D}
      f\left(\frac{i}{n,} \, \frac{j}{n,} \, \frac{k}{n}
      \right)\frac{1}{n^3}. 
    \]
  \end{minipage}
  \begin{minipage}{0.34\textwidth}
    \begin{asy}[width=5cm]
      import graph3;
      import palette; 

      picture myaxes;
      draw(myaxes,O--1.8*X,Arrow3());
      draw(myaxes,O--2.1*Y,Arrow3());
      draw(myaxes,O--2.1*Z,Arrow3());
     
      currentprojection=perspective(8,5,5); 
      currentlight.background = softyellow;
      
      add(myaxes); 
      
      real eps = 0.1;
      real x = 4.0, y = 4.0; 

      pen[] Palette = Gradient(100 ... new pen[] {0.8*white+0.2*Gold, 0.8*Gold});
      
      void drawbox(picture pic = currentpicture,
      real x, real y, real z, real delta,
      material surfacepen=Gold, pen wirepen=MidnightBlue)
      {
        draw(pic, surface((x+delta,y,z)--
        (x+delta,y+delta,z)--
        (x+delta,y+delta,z+delta)--
        (x+delta,y,z+delta)--cycle),
        surfacepen
        );
        draw(pic, surface((x,y+delta,z)--
        (x+delta,y+delta,z)--
        (x+delta,y+delta,z+delta)--
        (x,y+delta,z+delta)--cycle),
        surfacepen
        );
        draw(pic, surface((x,y,z+delta)--
        (x+delta,y,z+delta)--
        (x+delta,y+delta,z+delta)--
        (x,y+delta,z+delta)--cycle),
        surfacepen
        );
        draw(pic,(x+delta,y+delta,z+delta)--(x+delta,y+delta,z),wirepen);
        draw(pic,(x+delta,y+delta,z+delta)--(x+delta,y,z+delta),wirepen);
        draw(pic,(x+delta,y+delta,z+delta)--(x,y+delta,z+delta),wirepen);
      }
      
      real delta = 0.1; 
      for(real x = 1; x<= 2; x+= delta){
        for(real y = 1; y<= 2; y+= delta){
          for(real z = 1; z<= 2; z+= delta){
            drawbox(x,y,z,delta,surfacepen=material(diffusepen=Gold+
            opacity((x^2 + y^2 + z^2)/11),
            emissivepen=0.3*white+
            opacity((x^2 + y^2 + z^2)/11)),nullpen); 
          } 
        }
      }
    \end{asy}
  \end{minipage}

We can calculate the integral by slicing up the region of
  integration into thin slabs along `$z = \text{constant}$' slices,
  and then performing double integrals to find the area of each
  slab. This works the same as double iterated integration, but with
  one extra step. Rather than writing $\Delta z$ and then taking a
  limit to turn $\Delta z$ into $dz$, we'll skip to the limit and work
  directly with $dz$* \sidenote{* This is a general theme: we
    contract the following two steps into a single step (by writing
    $dz$ instead of $\Delta z$ from the outset): (i) reason
    about sums involving a small but positive quantity $\Delta z$, and
    (ii) replace the sum with an integral over the relevant $z$ values
    and replace $\Delta z$ with $dz$ }
  \begin{align*}
    \text{mass} &= \int_1^2 \overbrace{\int_1^2 \int_1^2 (x^2 + y^2 + z^2) \, \d x
                  \, \d y \, {\d}z}^{\text{mass of slice from $z$ to $z + dz$}} \\
                &= \int_1^2 \int_1^2 \left(y^{2} + z^{2} + \frac{7}{3}\right) \, \d y
                  \, {\d}z \\
                &= \int_1^2 \left(z^2 + \frac{14}{3}\right) \, {\d}z \\
                &= \boxed{7} \text{ kilograms}. 
  \end{align*}
\end{solution}


The following theorem summarizes the idea of integrating in 3D by
breaking down the 3D region of integration into 2D slices. 

\begin{theo}{Iterated integrals for three-variable functions}{iterated3}
  Suppose $f$ is a continuous function over a region $D$ which is
  bounded between the planes $z = a$ and $z = b$. For each
  $z \in (a,b)$, define $D_z \subset \R^2$ to be the region*
  \sidenote{* $D_z$ is the region obtained by intersecting $D$ with
    the plane which is $z$ units from the $xy$-plane and then dropping
    off the third coordinate.}
  \[
    D_z = \{(x,y) \in \R^2 \, : \, (x,y,z) \in D\}. 
  \]
  Then
  \[
    \iiint _D f \, {\d}V =
    \int_a^b \left[\iint_{D_z} f(x,y,z) \, \d x \, \d y \right] \, {\d}z, 
  \]
\end{theo}

Let's break this theorem down into a simple algorithm 
(the following observation is the 3D analogue of
Observation~\ref{obs:evaluating_integrals}): 

\begin{obs}{Limits of integration over a 3D
    region}{evaluating_3d_integrals}
  To set up an iterated integral to evaluate $\iiint_D f \, {\d}V$: 
  \begin{enumerate}[itemsep=3pt, topsep = 8pt]
\item Find the least and greatest $z$ values for any point in
  $D$. These are your \textbf{outer limits} of integration. 
\item For each fixed `$z=\text{constant}$' plane which intersects $D$, identify
  the least and greatest values of $y$ for any point which is in $D$
  \textit{and on that plane}, expressed in terms of the vertical
  position $z$ of the plane. These are the \textbf{middle limits} of
  integration, and they may depend on $z$.
\item For each line of the form `$z=\text{constant and
  }y=\text{constant}$', find the least and greatest values of $x$ for
  any point which is in $D$ \textit{and on that line}. These are your
  \textbf{inner limits} of integration, and they may depend on both
  $z$ and $y$. 
\end{enumerate}
\end{obs}

\begin{example}{}{}
  Find the volume of the tetrahedron with vertices $(0,0,0)$,
  $(2,0,0)$, $(0,3,0)$, and $(0,0,4)$ using a triple integral. 
\end{example}

\begin{solution}
  \begin{minipage}{0.6\textwidth} \parskip = 0.2 in 
    The volume of a region is equal to the integral of the constant
    function 1 over that region:
    \[
      \text{volume}(D) = \iiint_D 1 \, {\d}V, 
    \]
    because $\iiint_D 1 \, {\d}V$ is equal to the mass of a solid
    occupying the region $D$ and having density 1 at every point. But
    if a solid has a constant mass density of 1, then its mass is
    equal to its volume.* \sidenote{* Alternatively, note that
      calculating the Riemann sums approximating $\iiint_D 1 \, {\d}V$
      amounts to splitting $D$ into small pieces and summing their
      volumes.}

    So we set up our iterated integral: the least and greatest values
    of $z$ are 0 and 4, so those are our outer limits. For a fixed
    value of $z$, the least and greatest values of $y$ for a point in
    $D$ are 0 and $3 - \tfrac{3}{4}z$, respectively. Finally, for
    fixed $y$ and $z$, the least and greatest values of $x$ for a
    point in $D$ are 0 and the point on the plane $6x + 4y + 3z = 12$
    with the given values of $y$ and $z$ (see figure).
  \end{minipage} \: 
  \begin{minipage}{0.38\textwidth}
    \begin{asy}
      size(6cm);
      import three;
      currentprojection = perspective(5,3,3); 
      currentlight.background = softyellow;

      real k = 5;
      draw(O--k*Z,Arrow3());
      draw(O--3*X,Arrow3());
      draw(O--4*Y,Arrow3()); 
      
      triple A = 2*X, B = 3*Y, C = 4*Z;
      
      pen surfacepen = LightSeaGreen+opacity(0.4); 
      pen wirepen = MidnightBlue+linewidth(1.0); 

      draw(surface(A--O--B--cycle),surfacepen,wirepen);
      draw(surface(A--O--C--cycle),surfacepen,wirepen);
      draw(surface(C--O--B--cycle),surfacepen,wirepen);
      draw(surface(A--B--C--cycle),surfacepen,wirepen); 
      
      draw(A--B--C--cycle, wirepen);
      
      for(int i=0; i<3; ++i) {
        draw(O--(new triple[] {A,B,C})[i], wirepen);
      }
      
      real eps = 0.1; 
      draw("2",A--A+(0,0,-eps),align=S);
      draw("3",B--B+(0,0,-eps),align=S);
      draw("4",C--C+(0,eps,0),align=NE);
      
      real h = 1.5;
      path3 p = (0,0,h)--(2-h/2,0,h)--(0,3-3*h/4,h)--cycle; 
      draw(p,wirepen);
      draw(surface(p),grey);

      real eps = 0.05; 
      draw("$z$",3.5*Y+eps*Z--3.5*Y+h*Z, Bars3(2),align=E);
      
      dot("$y = 3-\frac{3}{4}z$",(0,3-3*h/4,h),align=3*NE);
      draw((0,3-3*h/4,h)+0.2*(0,1,1)--(0,3-3*h/4,h)+0.05*(0,1,1),Arrow3());
      
      triple q = arcpoint((2-h/2,0,h)--(0,3-3*h/4,h),0.5);
      draw(q+0.01*Z--(0,q.y,q.z+0.01),black+linewidth(2.0)); 
      dot(q);
      label("$x = 2 - \frac{2}{3}y - \frac{1}{2} z$", q + 0.5*X + 0.7*Y - 0.2*Z, fontsize(8)); 
      draw(q + 0.5*X - 0.2*Y - 0.15*Z -- q - 0.05*Z - 0.02*Y,Arrow3(5));

      real a = 3;
      real b = 3.25; 
      draw((-eps,-eps,h)--(a,-eps,h)--(a,b,h)--(-eps,b,h)--cycle,dashed); 
    \end{asy}
  \end{minipage}
  
  So we get
  \begin{align*}
    \text{volume}(D) &= \int_{0}^{4}\int_{0}^{3-\frac{3}{4}z}\int_{0}^{2 - \frac{2}{3}y -
                       \frac{1}{2}z} 1 \, \d x \, \d y \, {\d}z  \\
                     &= \int_{0}^{4}\int_{0}^{3-\frac{3}{4}z} \left(2 - \frac{2}{3}y -
                       \frac{1}{2}z \right) \, \d y \, {\d}z \\ 
                     &= \int_{0}^{4} \frac{3}{16}(z-4)^2 \, {\d}z \\
                     &= \boxed{4}. 
  \end{align*}
\end{solution}

There is nothing special about the order $\d x \, \d y \, {\d}z$---any way of
slicing up the region gives the same result. The region of integration
for the following exercise can be sliced up six different ways, and
you can check that the integral is the same with respect to all the
different orders of integration. 

\begin{exercise}{}{}
  Write the iterated integral
  \[
    \int_0^1 \int_{\sqrt{x}}^1\int_0^{1-y} f(x,y,z) \,\d z \, \d y \, \d x. 
  \]
  as an integral over a 3D region. Then sketch that region and use
  your figure to rewrite the integral in five other ways, using the
  five other permutations of $(x,y,z)$. 
\end{exercise}

\newpage 

\section{Polar, cylindrical, and spherical integration} \label{sec:polar_int}

Some regions in $\R^2$ are more conveniently described in polar
coordinates than rectangular coordinates. If we are integrating a
function $f: \R^2 \to \R$ over such a region, it is helpful to work
directly in polar coordinates. Let's do an example.

\begin{example}{}{polarintegral}
  (i) Find the area of the region $D$ enclosed by solution set of the polar
  coordinate equation $r = 1 + \cos \theta$. (ii) Integrate the
  function $f(x,y) = x + y$ over $D$. 
\end{example}

\begin{solution}
        \begin{minipage}{0.57\textwidth}
        \parskip = 0.2 in 
    (i) Let's slice $D$ into small pieces using equally
    spaced cuts along
    rays and circles of the form $r = \text{constant}$ and
    $\theta = \text{constant}$, as shown. This decomposes $D$
    into a set of \textbf{coordinate patches}. The figure suggests that
    these pieces farther away from the origin are larger than the ones
    that are close to the origin, which leads us to investigate the
    area of each patch. 

    To find the area of the set of points with radial polar coordinate between
    $r$ and $r+\Delta r$ and angular polar coordinate between $\theta$
    and $\theta + \Delta \theta$, we note that this region is
    approximately a rectangle. The straight side length is $\Delta r$,
    and the curvy side length is $r \Delta \theta$, because the perimeter of
    the circle of radius $r$ is $2\pi r$, and the angle represents
    $\frac{\theta}{2\pi}$ of the whole circle. So the area is
    approximately $r
    \Delta r \Delta \theta$.

    Now, for fixed $\theta$, we can add up all the coordinate patches
    between $\theta$ and $\theta + \Delta \theta$, and this sum of
    areas is approximately equal to the product of 
    $\Delta \theta$ and the integral
    \[
      \int_0^{1+\cos \theta} r \, {\d}r. 
    \]
      \end{minipage} 
      \begin{minipage}{0.42\textwidth}
        \begin{asy}[width=7cm]
          import graph; 
          defaultpen(fontsize(6));
          picture inset; 

          currentlight.background = softyellow; 


          real f(real x){return 1 + cos(x);}
          path q = polargraph(f,0,2*pi);
          filldraw(q--cycle,softgreen,MidnightBlue);
          filldraw(inset,q--cycle,softgreen,MidnightBlue);
          real x = 2.4; 
          draw((-x,0)--(x,0),Arrows());
          draw((0,-x)--(0,x),Arrows());
          
          int k = 6;
          int n = 20; 
          fill(2*k/n*expi(2*pi*5/40)--2*(k+1)/n*expi(2*pi*5/40)--
          2*(k+1)/n*expi(2*pi*4/40)--2*k/n*expi(2*pi*4/40)--cycle,softred);
          fill(inset, 2*k/n*expi(2*pi*5/40)--2*(k+1)/n*expi(2*pi*5/40)--
          2*(k+1)/n*expi(2*pi*4/40)--2*k/n*expi(2*pi*4/40)--cycle,softred);
          
          int n = 40;
          real th, r;
          path p;
          real[][] tms;
          
          for(int i=0; i<n; ++i){
            th = 2*pi*i/n; 
            draw((0,0)--f(th)*expi(th));
            draw(inset,(0,0)--f(th)*expi(th));
          }
          
          int n = 20;
          for(int i=0; i<n; ++i){
            r = 2*i/n; 
            p = circle((0,0), r);
            tms = intersections(p,q);
            if (tms.length > 1){
              draw(subpath(p,0,tms[0][0]),MidnightBlue);
              draw(inset, subpath(p,0,tms[0][0]),MidnightBlue);
              draw(subpath(p,tms[1][0],length(p)),MidnightBlue);
            }
          }
          
          int k = 6;
          
          path B = box((0.35,0.25),(0.65,0.55));
          
          draw(B);
          draw(inset,B); 
          clip(inset,B);
          
          transform t = shift((-4,-0.5))*scale(5); 
          
          for(int i=0;i<=3;++i){
            if (i != 1) {
              draw(point(B,i)--t*point(B,i),gray);
            }
            else {
              path p = point(B,i)--t*point(B,i); 
              draw(intersectionpoint(p,(0.35,0)--(0.35,5))--t*point(B,i),gray);
            }
          }
          
          draw(inset,rotate(45)*"$\Delta r$",
          2*k/n*expi(2*pi*5/40)--2*(k+1)/n*expi(2*pi*5/40),
          gray+linewidth(1.5),
          align=NW);
          draw(inset,rotate(-45)*"$r\, \Delta \theta$",
          arc((0,0),2*(k+1)/n,180/pi*2*pi*4/40,180/pi*2*pi*5/40),
          gray+linewidth(1.5),
          align=NE);
          dot(inset,"$(r,\theta)$",2*(k+1)/n*expi(2*pi*4/40),align=2*ESE); 
          
          draw(inset,"$r\, \Delta r\, \Delta \theta$",(2*(k+1/2)/n*expi(2*pi*4.5/40)));
          
          add(t*inset,UnFill); 
        \end{asy}
      \end{minipage}

      Adding up these areas over all $\theta$ from $0$ to $2\pi$, we
      get
      \[
        \int_0^{2\pi} \int_0^{1+\cos \theta} r \, \d r \, \d\theta =
        \int_0^{2\pi}\frac{1}{2}(1+\cos \theta)^2 \, \d\theta
        =  \boxed{\frac{3\pi}{2}}. 
              \]
              
              
    (ii) We can find this integral using the same procedure as above,
    except that at the step where we calculate the area of a patch, we
    also need to multiply it by the value of the function at some point
    in the patch. Since our function is defined in terms of $x$ and
    $y$, we need to substitute $x = r\cos \theta$ and $x = r\sin
    \theta$ to discover the value of $f$ at the point whose polar
    coordinates are $(r,\theta)$. So we get
    \sidenote{\href{https://cocalc.com/projects/7925f475-a0bd-4621-b78a-466c24c7863c/files/polar_integral.ipynb}{\cocalc} This one is tedious if done by
      hand, so we use computer algebra assistance.}
    \[
      \int_0^{2\pi} \int_0^{1+\cos \theta} (r \cos\theta + r\sin
      \theta)r \, {\d}r \, \d\theta = \boxed{\frac{5\pi}{4}}. 
    \]
\end{solution}

We can see from this example that the ideas for setting up an iterated
polar integral are similar to those for 
rectangular integration:
\begin{obs}{Iterated polar integration}{iteratedpolar}
\begin{enumerate}[itemsep = 6pt, topsep = 5pt, leftmargin=12pt]
\item Find the least and greatest values of $\theta$ for any point in
  the region of integration, and 
\item For each fixed value of $\theta$, find the least and greatest
  values of $r$ for any point on the ray of angle $\theta$ and in the
  region of integration
\item Include the area differential* $\d A = r \, {\d}r \, \d\theta$
\item Substitute $x = r \cos\theta$ and $y=r\sin \theta$ into $f$, so
  that your integrand varies appropriately as $r$ and $\theta$ vary 
  \sidenote{* Don't forget the extra factor of $r$ in the polar area
    differential!}[-1cm]
\end{enumerate}
\end{obs}

This same basic idea can be carried out in cylindrical and spherical
coordinates. The ingredients we need are (i) the volume differential
$\d V$ expressed in terms of cylindrical and spherical coordinates, and
(ii) the formulas for $x, y, z$ in terms of $r, \theta, z$ and in
terms of $\rho, \theta, \phi$. This information is listed in
Appendix~\ref{sec:polarref}. The only surprising entry in the tables
of that appendix is the spherical coordinate volume differential $\d V =
\rho^2 \sin \phi \, {\d}\rho \, {\d} \phi \, \d\theta$.

\begin{example}{Spherical coordinate volume differential}{voldiff}
  Explain why the volume differential in spherical coordinates is $\d V =
\rho^2 \sin \phi \, {\d}\rho \, {\d} \phi \, \d\theta$. 
\end{example}

\begin{solution}
  \begin{minipage}{0.7\textwidth} \parskip = 0.2 in The volume
    differential arises from slicing up the region of integration into
    small coordinate ``patches'', each of which consists of those
    points whose three spherical coordinates lie in the intervals*
    $[\rho, \rho+\Delta \rho]$, $[\theta, \theta+\Delta \theta]$, and
    $[\phi, \phi+\Delta \phi]$, respectively (see the top figure,
    where a wedge has been decomposed into spherical coordinate
    patches). Thus we must calculate the approximate volume of one
    such patch. \sidenote{* Here $(\rho, \theta, \phi)$ are the
      spherical coordinates of one of the corners of the patch.}

    When $\Delta \rho$, $\Delta \theta$, and $\Delta \phi$ are all
    very small, the coordinate patch is approximately a rectangular
    prism. The dimensions of this rectangular prism, as marked in the
    lower figure, are $\Delta \rho$, $\rho\, \Delta \phi$, and
    $\rho \sin \phi \, \Delta \theta$. 
  \end{minipage}
  \begin{minipage}{0.35\textwidth} 
    \begin{center}
      \begin{asy}
        import graph3;
        size(4cm); 

        picture myaxes(real h){
          picture result; 
          draw(result,O--h*X,Arrow3());
          draw(result,O--h*Y,Arrow3());
          draw(result,O--h*Z,Arrow3());
          return result;
        }

        currentlight.background = softyellow; 
        
        material surfacemat = material(opacity=0.4,diffusepen=0.5*black+0.5*LightSeaGreen,
                                                                             emissivepen=0.5*black+0.5*LightSeaGreen); 
        pen surfacepen = LightSeaGreen+opacity(0.4); 
        pen wirepen = MidnightBlue;
        
        triple xyz(real rho, real theta, real phi){
          return rho*(cos(theta)*sin(phi), sin(theta)*sin(phi), cos(phi)); 
        }
        
        int n = 20;
        real phi = pi/4;
        real eps = 2*pi/n; 
        
        void drawpatch(picture pic=currentpicture,
        real r1, real r2,
        real theta1, real theta2,
        real phi1, real phi2, bool opaque = false){ 
          real[] rvals = {r1, r2};
          real[] thvals = {theta1, theta2};
          real[] phvals = {phi1, phi2}; 
          for(int i=0; i<rvals.length; ++i){
            real r = rvals[i]; 
            draw(pic,surface(new triple(pair p){return xyz(r,p.x,p.y);},
            (thvals[0],phvals[0]),
            (thvals[1],phvals[1]),
            10,
            Spline), opaque ? surfacemat : surfacepen, nullpen);
          } 
          for(int i=0; i<thvals.length; ++i){
            real th = thvals[i]; 
            draw(pic,surface(new triple(pair p){return xyz(p.x,th,p.y);},
            (rvals[0], phvals[0]),
            (rvals[1], phvals[1]),
            10,
            Spline), opaque ? surfacemat : surfacepen, nullpen);
          }
          
          for(int i=0; i<phvals.length; ++i){
            real ph = phvals[i]; 
            draw(pic,surface(new triple(pair p){return xyz(p.x,p.y,ph);},
            (rvals[0], thvals[0]),
            (rvals[1], thvals[1]),
            10,
            Spline), opaque ? surfacemat : surfacepen, nullpen);
          }
          
          for(int i=0; i<thvals.length; ++i){
            for(int j=0; j<phvals.length; ++j){
              real th = thvals[i];
              real ph = phvals[j];
              draw(pic,graph(new triple(real t){return xyz(t,th,ph);},
              rvals[0],
              rvals[1],
              10,
              Spline), wirepen);
            }
          }
          for(int i=0; i<rvals.length; ++i){
            for(int j=0; j<phvals.length; ++j){
              real r = rvals[i];
              real ph = phvals[j];
              draw(pic,graph(new triple(real t){return xyz(r,t,ph);},
              thvals[0],
              thvals[1],
              10,
              Spline), wirepen);
            }
          }
          for(int i=0; i<thvals.length; ++i){
            for(int j=0; j<rvals.length; ++j){
              real th = thvals[i];
              real r = rvals[j];
              draw(pic,graph(new triple(real t){return xyz(r,th,t);},
              phvals[0],
              phvals[1],
              10,
              Spline), wirepen);
            }
          }
        }
        
        real r1 = 0.8, r2 = 1;
        real theta1 = pi/4, theta2 = pi/4 + eps;
        real phi1 = theta1, phi2 = theta2; 
        
        int m = 8, n = 20, p = 20;
        
        for(int i=0; i<m; ++i){
          for(int j=3; j<5; ++j){
            for(int k=0; k<p/2; ++k){
              real r = i/m, theta = 2*pi*j/n, phi = pi*k/p;
              drawpatch(r,r+1/m,theta,theta+2*pi/n,phi,phi+pi/n,opaque=true); 
            }
          }
        }
        
        currentprojection = perspective(5,0.4,1.8);
        add(myaxes(1.25)); 
      \end{asy}
    \end{center}
  \end{minipage}
  
  \begin{minipage}[t]{0.5\textwidth} \parskip = 0.2 in
    To see why the top edge length is $\rho \sin \phi \Delta \theta$,
    note that the dashed circle in the figure has radius
    $\rho \sin \phi$, since the \textit{cylindrical} radial coordinate
    $r$ satisfies the equation $r = \rho \sin \phi$. Thus the volume
    of the patch is approximately
    $\rho^2 \sin \phi \Delta \rho\, \Delta \phi \, \Delta \theta$.
  \end{minipage}
  \begin{minipage}[t]{0.5\textwidth}
    \begin{center}
      \begin{lrbox}{\asybox}
      \begin{asy}
        import graph3;
        size(5cm);

        currentlight.background = softyellow; 

        picture myaxes(real h){
          picture result; 
          draw(result,O--h*X,Arrow3());
          draw(result,O--h*Y,Arrow3());
          draw(result,O--h*Z,Arrow3());
          return result;
        }
        
        material surfacemat = material(opacity=0.4,diffusepen=0.5*black+0.5*LightSeaGreen,
                                                                             emissivepen=0.5*black+0.5*LightSeaGreen); 
        pen surfacepen = LightSeaGreen+opacity(0.4); 
        pen wirepen = MidnightBlue;
        
        triple xyz(real rho, real theta, real phi){
          return rho*(cos(theta)*sin(phi), sin(theta)*sin(phi), cos(phi)); 
        }
        
        int n = 20;
        real phi = pi/4;
        real eps = 2*pi/n; 
        
        void drawpatch(picture pic=currentpicture,
        real r1, real r2,
        real theta1, real theta2,
        real phi1, real phi2, bool opaque = false){ 
          real[] rvals = {r1, r2};
          real[] thvals = {theta1, theta2};
          real[] phvals = {phi1, phi2}; 
          for(int i=0; i<rvals.length; ++i){
            real r = rvals[i]; 
            draw(pic,surface(new triple(pair p){return xyz(r,p.x,p.y);},
            (thvals[0],phvals[0]),
            (thvals[1],phvals[1]),
            10,
            Spline), opaque ? surfacemat : surfacepen, nullpen);
          } 
          for(int i=0; i<thvals.length; ++i){
            real th = thvals[i]; 
            draw(pic,surface(new triple(pair p){return xyz(p.x,th,p.y);},
            (rvals[0], phvals[0]),
            (rvals[1], phvals[1]),
            10,
            Spline), opaque ? surfacemat : surfacepen, nullpen);
          }
          
          for(int i=0; i<phvals.length; ++i){
            real ph = phvals[i]; 
            draw(pic,surface(new triple(pair p){return xyz(p.x,p.y,ph);},
            (rvals[0], thvals[0]),
            (rvals[1], thvals[1]),
            10,
            Spline), opaque ? surfacemat : surfacepen, nullpen);
          }
          
          for(int i=0; i<thvals.length; ++i){
            for(int j=0; j<phvals.length; ++j){
              real th = thvals[i];
              real ph = phvals[j];
              draw(pic,graph(new triple(real t){return xyz(t,th,ph);},
              rvals[0],
              rvals[1],
              10,
              Spline), wirepen);
            }
          }
          for(int i=0; i<rvals.length; ++i){
            for(int j=0; j<phvals.length; ++j){
              real r = rvals[i];
              real ph = phvals[j];
              draw(pic,graph(new triple(real t){return xyz(r,t,ph);},
              thvals[0],
              thvals[1],
              10,
              Spline), wirepen);
            }
          }
          for(int i=0; i<thvals.length; ++i){
            for(int j=0; j<rvals.length; ++j){
              real th = thvals[i];
              real r = rvals[j];
              draw(pic,graph(new triple(real t){return xyz(r,th,t);},
              phvals[0],
              phvals[1],
              10,
              Spline), wirepen);
            }
          }
        }
        
        real r1 = 0.8, r2 = 1;
        real theta1 = pi/4, theta2 = pi/4 + eps;
        real phi1 = theta1, phi2 = theta2; 
        
        add(myaxes(1.25));
        currentprojection = perspective(5,0.4,0.3);
        
        drawpatch(r1, r2, theta1, theta2, theta1, theta2);
        
        triple M = 0.5*(xyz(r1,theta2,phi2)+xyz(r2,theta2,phi2)); 
        label("$\Delta \rho$", M,align=2*SE); 
        draw(M - 0.07*Z + 0.07*Y -- M,Arrow3(4.5));
        
        triple M = xyz(r2,theta2,0.5*phi1+0.5*phi2);
        label("$\rho \, \Delta \phi$", M,align=2.5*NE);
        draw(M + 0.06*(Y+Z) -- M, Arrow3(4.5)); 
        
        triple M = 0.5*(xyz(r2,theta1,phi1)+xyz(r2,theta2,phi1));
        label("$\rho \, \sin \phi  \, \Delta \theta$", M, align=5*N); 
        draw(M + 0.11*Z -- M, Arrow3(4.5)); 
        
        draw(circle((0,0,r2*cos(phi1)),r2*sin(phi1),normal=Z),dashed+gray);
        draw(xyz(r2,theta1,phi1)--(0,0,r2*cos(phi1))--xyz(r2,theta2,phi1));
        triple A = (0,0,r2*cos(phi1)); 
        triple B = xyz(r2,theta1,phi1);
        triple C = 0.7*A + 0.3*B;
        triple D = 0.3*A + 0.7*B; 
        triple h = -0.1*Z;
        draw(surface(yscale(-0.8)*"$\rho \sin \phi$",surface(C--D--D+h--C+h--cycle),0,0,0)); 
        
        draw(xyz(r1,theta2,phi1)--O--xyz(r1,theta2,phi2),dashed+gray);
        draw(xyz(r1,theta1,phi1)--O--xyz(r1,theta1,phi2),dashed+gray);  
      \end{asy}
    \end{lrbox} \raisebox{\dimexpr -\height + 1.5ex \relax}{\usebox{\asybox}}
  \end{center}
  \end{minipage}
\end{solution}

\begin{example}{}{spherical-mass}
  Consider a solid whose density at each point $(x,y,z)$ is
  $\rho(x,y,z) = \frac{1}{x^2 + y^2 + z^2}$ and which occupies the region
  enclosed by the cone $z = \sqrt{x^2+y^2}$ and the plane $z = 1$.
  Find the mass of the solid. 
\end{example}

\begin{solution}
  \begin{minipage}{0.65\textwidth}
    Let's set up the iterated integral with the order
    $d\rho\, {\d}\phi\, \d\theta$. The solid has points with $\theta$
    values as small as $0$ and as large as $2\pi$, so the outer limits
    will be $0$ and $2\pi$.

    For any given value of $\theta$, there are points with that
    $\theta$ value whose $\phi$ value is as small as zero (for the
    points on the positive $z$-axis) and as large as $\tfrac{\pi}{4}$
    (for the points on the cone $z = \sqrt{x^2+y^2}$). So the middle
    limits are $0$ and $\tfrac{\pi}{4}$.

    Finally, for any given $\phi$ and $\theta$, the solid contains
    points with $z$ as small as 0 and as large as
    $\frac{1}{\cos \phi}$ (by right-triangle trigonometry; see
    figure).
  \end{minipage} \hspace{5mm} 
  \begin{minipage}{0.34\textwidth}
    \begin{asy}[width=5cm]
      import smoothcontour3;
      import graph3; 
      
      size(8cm); 

      currentprojection = perspective(5,2,2);
      currentlight.background = softyellow;

      real h = 1.5; 
      draw(O--h*X,Arrow3());
      draw(O--h*Y,Arrow3());
      draw(O--h*Z,Arrow3());
      
      pen surfacepen = LightSeaGreen+opacity(0.4);
      pen wirepen = MidnightBlue; 
      
      triple xyz(real rho, real theta, real phi){
        return rho*(cos(theta)*sin(phi), sin(theta)*sin(phi), cos(phi)); 
      }
      
      real phi = pi/4; 
      
      draw(surface(new triple(pair p){return xyz(p.x,p.y,phi);},
      (0,0),
      (sqrt(2), 2*pi),
      10,
      Spline),
      surfacepen);
      path3 p = circle((0,0,1),1,Z);
      
      draw(p,wirepen); 
      draw(surface(p),surfacepen); 
      
      triple t = (1/3,2/3,1);
      
      draw(O--t--Z,wirepen);
      draw(t--1.4*t,wirepen+dotted); 
      draw("1",O--Z,wirepen,align=W);
      real eps = 0.2; 

      draw("$\phi$",arc(O,eps*Z,eps*t/length(t)),align=dir(80)); 
    \end{asy}
  \end{minipage}

For the integrand, we should substitute the spherical coordinate
formulas for $x$, $y$, and $z$. However, we know that it will simplify
to $\frac{1}{\rho^2}$, since $\rho^2 = x^2 + y^2 + z^2$. So we get
\sidenote{\href{https://cocalc.com/projects/7925f475-a0bd-4621-b78a-466c24c7863c/files/cone_integral.ipynb}{\cocalc}
  Help with calculating this integral}
\[
  \int_{0}^{2\pi}\int_0^{\pi/4} \int_0^{\sec \phi} \rho^{-2} \rho^2 \sin
  \phi \, {\d}\rho \, {\d}\phi \, \d\theta =
  \int_{0}^{2\pi}\int_0^{\pi/4} \sec\phi \sin
  \phi \, {\d}\rho \, {\d}\phi \, \d\theta = (2\pi)(\tfrac{1}{2} \ln 2) =
  \boxed{\pi \ln 2}.
\]
\end{solution}

\begin{exercise}{}{}
  What proportion of the volume of the unit sphere lies above the
  plane $z = \tfrac{1}{2}$?
\end{exercise}

\begin{exercise}{}{}
  Find \[\int_{-2}^2 \int_{-\sqrt{4-x^2}}^{\sqrt{4-x^2}}
  \int_{\sqrt{x^2 + y^2}}^2 (x^2 + y^2) \, {\d}z \, \d y \, \d x\] by writing
  it as integral over a 3D region and then rewriting that integral
  using cylindrical coordinates. 
\end{exercise}

\section{Change of variables} \label{sec:changeofvariables}

\milink{double_integral_change_variables_introduction}{change of variables}

Sometimes we want to integrate a function over a region which is not
conveniently described using any of the standard coordinate
systems. In this section we will develop a general program for
integrating with respect to a custom-designed coordinate system.

\begin{example}{}{change_of_variables}
  Find $\iint_D y^2 \, {\d}A$, where $D$ is the region in the first
  quadrant bounded by the lines
  through the origin of slope $\tfrac{1}{2}$ and $2$, as well as the
  hyperbolas $xy = 1$ and $xy = 3$. 
\end{example}

\begin{solution}
  \begin{center}
    \begin{asy}
      size(12cm); 
      import graph;
      
      currentlight.background = softyellow; 

      picture left, right;
      size(left, 6cm);
      size(right, 6cm);
      
      real eps = 0.1;
      real a = 4;
      real b = 3.5; 
      
      draw(left,(-eps,0)--(a,0),Arrow());
      draw(left,(0,-eps)--(0,b),Arrow());
      draw(right,(-eps,0)--(a,0),Arrow());
      draw(right,(0,-eps)--(0,b),Arrow());
      
      path p = graph(new pair(real t){ return (t,1/t);}, 0.5, 2, 40, Spline);
      path q = graph(new pair(real t){ return (t,3/t);}, 1, 3, 40, Spline);
      path r = (0,0)--(1.5,3.0);
      path s = (0,0)--(3,1.5);
      fill(right,buildcycle(p,s,q,r),softgreen); 
      draw(right, Label("$xy = 1$",Relative(1)), p);
      draw(right, Label("$xy = 3$",Relative(1)), q); 
      draw(right, Label("$y = 2x$",Relative(1)), r);
      draw(right, Label("$y = \frac{1}{2}x$",Relative(1)), s);
      
      filldraw(left, box((1/2,1),(2,3)),softgreen);
      
      label(left,"$R$",(2,1),align=SE);
      label(right,"$D$",(2,2)); 
      
      int m = 9;
      int n = 12; 
      for(int i=1; i<m; ++i){
        real u = 1/2+3/2*i/m;
        draw(left, (u,1)--(u,3));
        draw(right, graph(new pair(real v) {return (sqrt(v/u),sqrt(u*v));}, 1, 3, Spline)); 
      }
      for(int j=1; j<n; ++j){
        real v = 1+2*j/n; 
        draw(left,(1/2,v)--(2,v));
        draw(right, graph(new pair(real u) {return (sqrt(v/u),sqrt(u*v));}, 1/2, 2, Spline)); 
      }
      
      draw(left,Label("$\frac{1}{2}$",Relative(1)),(1/2,-eps)--(1/2,0),align=2*S);
      draw(left,Label("$2$",Relative(1)),(2,-eps)--(2,0),align=2.5*S);
      draw(left,Label("$1$",Relative(1)),(-eps,1)--(0,1),align=2*W);
      draw(left,Label("$3$",Relative(1)),(-eps,3)--(0,3),align=2.5*W); 
      
      label(left,"$u$",(a,0),align=S);
      label(left,"$v$",(0,b),align=E);
      label(right,"$x$",(a,0),align=S);
      label(right,"$y$",(0,b),align=E);
      
      draw("$T$",(2.5,2){ENE}..{ESE}(4.5,2),Arrow(),align=N); 
      
      add(left);
      add(shift((5,0))*right);
    \end{asy}
  \end{center}
  Let's cut the region along lines of the form $y = ux$ where $u$
  ranges over equally spaced values between $\tfrac{1}{2}$ and $2$,
  and along hyperbolas of the form $xy = v$ where $v$ ranges over
  $[1,3]$, as shown in the figure on the right above. 

  Note that each point $(x,y)$ in the first quadrant can be identified
  by its $u$ and $v$ values.* So $u$ and $v$ provide a coordinate
  system for the first quadrant. We can visualize the relationship
  between $(u,v)$ and $(x,y)$ as a transformation $T$ that maps each
  $(u,v)$ pair to its corresponding $(x,y)$ pair, as shown in the
  figure. If we want to find a formula for this map, we can solve the
  system $y = ux$ and $xy = v$ for $x$ and $y$ to find that
  $y = \sqrt{uv}$ and $x = \sqrt{v/u}$.  \sidenote{* These specify,
    respectively, which line through the origin and which hyperbola
    the point is on}[-5mm]

  To integrate $f(x,y) = x$ over $D$, we must find the area of each of
  these small patches, multiply each area by the value of $f$
  somewhere on the patch, and sum the resulting products. Each patch
  is the image under $T$ of a rectangle of the form
  $[u, u+\Delta u] \times [v, v + \Delta v]$. The area of this
  rectangle is $\Delta u \Delta v$, and the transformation distorts
  its area by an amount that we can approximate by treating the
  transformation as linear around $(u,v)$ and using the fact that area
  distortion is measured by the determinant. 

  Writing $T(u,v) =
  (\sqrt{v/u}, \sqrt{uv}) = (g(u,v), h(u,v))$, we see that $T$ maps
  the point $(u+\Delta u, v)$ to
  \begin{equation} \label{eq:Tv}
    T(u,v) + ((\partial_u g)(u,v) \Delta u,
    (\partial_u h)(u,v) \Delta u), 
  \end{equation}
  and
  $(v, v+\Delta v)$ to
  \begin{equation} \label{eq:Tu}
    T(u,v) + ((\partial_v g)(u,v) \Delta v,
    (\partial_v h)(u,v) \Delta v). 
  \end{equation}
  Thus the area of the image of $[u, u+\Delta u] \times [v, v + \Delta
  v]$ under $T$ is approximately* \sidenote{* The four entries of the
    matrix below come from the coefficients of $\Delta u$ and $\Delta
    v$ in \eqref{eq:Tu} and \eqref{eq:Tv}}[-1cm]
  \[
    \left| \det \left[
      \begin{array}{cc}
        \partial_u g & \partial _v g \\
        \partial_u h & \partial _v h
      \end{array} \right]
    \right| \Delta u \Delta v,  =
    \left| \det \left[
      \begin{array}{cc}
        -\frac{1}{2}\sqrt{v}u^{-3/2} & \frac{1}{2\sqrt{uv}} \\
        \frac{\sqrt{v}}{2\sqrt{u}} & \frac{\sqrt{u}}{2\sqrt{v}}
      \end{array} \right]
      \right| \Delta u \Delta v  = \frac{1}{2u}
      \Delta u \Delta v. 
    \]
    As usual, we may take $\Delta u, \Delta v \to 0$ to get 
    $\displaystyle{\int_{1}^{3}\int_{1/2}^{2}uv \frac{1}{2u}  
    \d u \, {\d}v = \boxed{3}}$. 
  \end{solution}

  The matrix $\left[
      \begin{array}{cc}
        \partial_u g & \partial _v g \\
        \partial_u h & \partial _v h
      \end{array} \right]$ is called the \textit{Jacobian matrix}, and
    its determinant is called the \textit{Jacobian determinant}. Often we just say
    ``Jacobian'', relying on context to distinguish. It can be written
    as $\left|\frac{\partial (x,y)}{\partial (u,v)} \right|$ for
    short.* \sidenote{* Note that in this new notation, the symbols
      $g$ and $h$ are replaced by $x$ and $y$. Thus we are abusing notation by
      regarding $x$ and $y$ as functions of $u$ and $v$, even though
      they also represent independent variables}[-1cm]
    
    The following theorem summarizes the technique we developed in
    Example~\ref{exam:change_of_variables}. 

  \begin{theo}{Multivariable change of variables}{change_of_variables}
    Suppose that $T: \R^2 \to \R^2$ is a differentiable transformation
    that maps a region $R$ one-to-one onto a region $D$. Then for any
    continuous function $f$, we have 
    \[
      \iint_D f(x,y) \, \d x\, \d y = \iint_R f(T^{-1}(x,y)) \left|
        \frac{\partial (x,y)}{\partial (u,v)} \right| \, {\d}u \, {\d}v. 
    \]
  \end{theo}

  

  \begin{exercise}{}{}
    Use the change of variables $x = u^2 - v^2$, $y = 2uv$ to evaluate the
integral $\iint_R y \,\d A$, where $R$ is the region above by the $x$-axis
bounded by the parabolas $y^2 = 4-4x$ and $y^2 = 4+4x$. 
\end{exercise}

\begin{exercise}{}{}
  Find the integral of $\frac{(x-y)^2}{x+y+2}$ over the square whose
  vertices are the four points of intersection between the axes and
  the unit circle.
\end{exercise}


  \newpage 

\chapter{Vector Calculus} 

\section{Vector fields and line integrals} \label{sec:vector_fields}

\milink{vector_field_overview}{vector fields}

\begin{wrapfigure}[18]{R}[1cm]{7cm}
  \begin{asy}[width=7cm]
    import multitools;     
    currentprojection=perspective(5,2.5,2);

    path3 gravity(triple t){
      return length(t) == 0 ? O--O : O--(-t)/length(t)^3;
    }
    
    triple A = (-1,-1,-1);
    triple B = (1,1,1);
    
    add(VectorPlot3D(gravity,A,B,5,5,5,maxlength=0.2,minlength=0.025,p=LightSeaGreen));
    
    real l = 2.4;
    pen axispen = gray+opacity(0.5);
    triple newO = (-1,-1,-1); 
    draw(newO--newO+l*X,axispen,Arrow3());
    draw(newO--newO+l*Y,axispen,Arrow3());
    draw(newO--newO+l*Z,axispen,Arrow3()); 
    
    real eps = 0.05; 
    draw("0",(-1,-1,0)--(-1,-1-eps,0),gray,align=1.5*W);
    draw("0",(-1,0,-1)--(-1,0,-1-eps),gray,align=1.5*S);
    draw("0",(0,-1,-1)--(0,-1,-1-eps),gray,align=1.5*S);    
  \end{asy}
  \caption{A gravitational vector field\label{fig:gravity}}
\end{wrapfigure}

So far we have been considering functions from $\R^n$ to $\R^1$ (where
$n$ is $2$ or $3$). In this chapter we work with functions
from $\R^n$ to $\R^2$ or $\R^3$. How should such functions be
visualized? Let's begin by considering how they arise in applications.

\sidenote{* Many other physical phenomena, such as electromagnetic
  forces, heat flow, fluid flow are modeled as vector fields. } To
describe the gravitational force* felt by a particle, we would use a
function with a three-dimensional input (to specify the particle's
location) as well as a three-dimensional output, to specify the
direction and magnitude of the force. It is natural to represent this
function by drawing a small arrow indicating the output vector at
several points in space, because this makes it easy to imagine how the
force changes as the particle moves around (see
Figure~\ref{fig:gravity}).

This picture suggests the term \textbf{vector field} for a function
from $\R^n$ to $\R^n$, where $n > 1$. The gravitational vector field
plotted in Figure~\ref{fig:gravity} is
\begin{equation} \label{eq:gravity} 
  \mathbf{F}(x,y,z) = -\frac{GMm}{(x^2+y^2+z^2)^{3/2}}\left\langle x, y, z
  \right\rangle. 
\end{equation}
  where $G$ is the gravitational constant and $Mm$ is the product of
  the masses of particle and the attracting body. 
  
\begin{exercise}{}{}
  \begin{minipage}[t]{0.58\textwidth}
    The vector plot shown represents the velocity of water on the
    surface of a river. The water is
    flowing due east, and it is flowing faster near the south end of
    the river than the north. Come up with a vector field $\mathbf{F}$
    whose vector plot looks approximately like the one shown.
  \end{minipage}
  \begin{minipage}[t]{0.4\textwidth}
    \begin{lrbox}{\asybox}
      \begin{asy}
        import graph;
        size(6cm); 
        
        real eps = 0.2; 
        pair a=(0,0), b = (2,1);

        path vector(pair z) {return (0,0)--(1-z.y^2,0);}

        draw(a--(b.x+eps,0),Arrow(4));
        draw(a--(0,b.y+eps),Arrow(4));

        real eps = 0.03; 
        draw(Label("2",Relative(1)),(b.x,0)--(b.x,-eps),align=S);
        draw(Label("1",Relative(1)),(0,b.y)--(-eps,b.y),align=W);

        add(vectorfield(vector,(eps,eps),b));
      \end{asy}
    \end{lrbox}
    \hfill \raisebox{\dimexpr-\height+1.5ex\relax}{\usebox{\asybox}}
  \end{minipage}
\end{exercise}

\milink{line_integral_vector_field_introduction}{line integrals}
Now suppose that rather than remaining stationary, our particle moves
along a path in the presence of a force field (see
Figure~\ref{fig:trigvectorfield}).* Sometimes the particle is moving
with the force field and getting a boost from it, whereas other times
it's working against the force field. How much net work does it take
to move along the path? \sidenote{* A
  vector field in which the vectors represent a physical force}[-1cm]

If the force field were constant and the path straight, then we get
the answer from physics: the work is equal to the product of the
magnitude $F$ of the force, the distance $d$ traveled, and the cosine of the
angle $\theta$ between the force and the path. Alternatively, we may
interpret the force and distance as vectors $\mathbf{F}$ and
$\mathbf{d}$ and write the work as a dot product: 
\[
  W = F d \cos \theta = \mathbf{F} \cdot \mathbf{d}. 
\]

\begin{wrapfigure}[18]{R}[1cm]{6cm}
  \begin{asy}
    import graph;
    size(6cm);
    
    pair a=(0,0);
    pair b=(2pi,2pi);
    
    path vector(pair z) {return (0,0)--(sin(z.x),cos(z.y));}
    
    add(vectorfield(vector,a,b,gray,arrow=Arrow(3)));

    path p = graph(new pair (real t) {return ((1-t)*(2-t),1+t);}, 0, 4, Spline); 

    draw(p);

    dot(Label("$\mathbf{r}(a)$",Fill(white+opacity(0.6))),point(p,0),fontsize(6),align=1.5*SE);
    dot("$\mathbf{r}(b)$",point(p,size(p)-1),fontsize(6),align=S);
    real eps = 0.4; 
    draw((-eps,-eps)--(b.x+2*eps,-eps),Arrow());
    draw((-eps,-eps)--(-eps,b.y+2*eps),Arrow()); 
    \end{asy}
  \caption{The path of a particle moving through a vector field\label{fig:trigvectorfield}}
\end{wrapfigure}

So how do we bootstrap our way from constant force and a straight path
to varying force and a curvy path? We can cut up the path into small
pieces, handle each small piece by treating the force as approximately
constant and the path as approximately straight, and then adding up
the amount of work for each small piece. We will assume that our path
$\mathbf{r}(t)$ is differentiable.* \sidenote{* The curve could
  alternatively by \textit{piecewise} differentiable, meaning that the
  curve is non-differentiable at only finitely many points}[-1cm]

Suppose we have a path $C$ parametrized as $\mathbf{r}(t)$ where $t$
ranges from $a$ to $b$. Over the time interval $[t,t+\Delta t]$, the
particle is displaced by* the vector $\mathbf{r}'(t) \, \Delta t$, and
the force it feels over that time is* to $\mathbf{F}(\mathbf{r}(t))$.
Therefore, the contribution from the time period $[t, t+\Delta t]$ is
equal to \sidenote{* Approximately, with an error that vanishes as
  $\Delta t \to 0$}
\[
  \mathbf{F}(\mathbf{r}(t)) \cdot \mathbf{r}'(t) \, \Delta t. 
\]
Summing all these contributions and taking $\Delta t \to 0$, we arrive
at the formula
\[
  W = \int_a^b \mathbf{F}(\mathbf{r}(t)) \cdot \mathbf{r}'(t) \, {\d}t =
  \int_C \mathbf{F} \cdot \d\mathbf{r}, 
\]
where the last expression is an abbreviation for the middle
expression. We call an integral of the form $\int_C \mathbf{F} \cdot
\d\mathbf{r}$ a \textbf{line integral}. 

\begin{example}{}{}
  Find the line integral of $\mathbf{F}(x,y) = \langle xy, y \rangle$
  along the parabola $y = x^2$ from 
  $(0,0)$ to $(2,4)$
\end{example}

\begin{solution}
  Let's parametrize the parabola using the $x$ coordinate as the
  parameter:
  \[
    \mathbf{r}(t) = \langle t, t^2 \rangle. 
  \]
  Note that the point $(2,4)$ is visited at time $t=2$, while the
  origin is visited at time $t=0$. Therefore, 
  \[
    W = \int_0^2 \langle t(t^2), t^2 \rangle \cdot \langle 1, 2t \rangle
    \, {\d}t = \int_0^2 (t^3 + 2t^3) \, {\d}t = \boxed{12}. 
  \]
\end{solution}

The following theorem states that the choice of parametrization of a
curve doesn't matter when computing a line integral. This makes sense
physically, since the formula $W = Fd$ does not involve time, and our
derivation of the line integral formula was based on $W=Fd$. The role
of the parametrization was merely to provide a convenient way to split
up the path into short pieces. Exercise~\ref{exer:IOP} below gives an
example.

\begin{theo}{Independence of parametrization}{IOP}
  If $C$ is a curve parametrized by $\mathbf{r}_1$ over $[a,b]$ and
  also by $\mathbf{r}_2$ over $[c,d]$, then
  \[
    \int_a^b\mathbf{F}(\mathbf{r}_1(t))  \cdot \,
    \mathbf{r}'_1(t) \, {\d}t =
    \int_c^d \mathbf{F}(\mathbf{r}_2(t))  \cdot \,
    \mathbf{r}'_2(t) \, {\d}t. 
  \]
  In other words, $\int_C \mathbf{F} \cdot \d\mathbf{r}$ depends only
  on the curve $C$, not the choice of parametrization. 
\end{theo}

\begin{exercise}{}{IOP} \parskip = 6 pt
  (i) Compute the line integral of
  $\mathbf{F} = \langle x^2,-xy\rangle$ over the portion of the unit
  circle in the first quadrant, using the parametrization
  $\mathbf{r}(t) = \langle \sin t, \cos t \rangle$. 

  (ii) Perform the same line integral using the parametrization
  $\mathbf{r}(t) = \langle t, \sqrt{1-t^2} \rangle$.
\end{exercise}

\begin{exercise}{}{}
  Consider the vector field $\mathbf{F}$ and path $C$ shown in
  Figure~\ref{fig:trigvectorfield}. Is
  $\int_C \mathbf{F} \cdot \d\mathbf{r}$ positive or negative?
\end{exercise}

\section{The fundamental theorem of vector calculus} \label{sec:line_integrals}

\milink{gradient_theorem_line_integrals}{the gradient theorem for line
integrals}

\begin{wrapfigure}[17]{R}[1cm]{6cm}
  \begin{asy}[width=5cm]
    import graph;
    picture vectorfieldmid(path vector(pair), pair a, pair b,
    int nx=nmesh, int ny=nx, bool truesize=false,
    real maxlength=truesize ? 0 : maxlength(a,b,nx,ny),
    bool cond(pair z)=null, pen p=currentpen,
    arrowbar arrow=Arrow, margin margin=PenMargin)
    {
      picture pic;
      real dx=1/nx;
      real dy=1/ny;
      bool all=cond == null;
      real scale;
      
      if(maxlength > 0) {
        real size(pair z) {
          path g=vector(z);
          return abs(point(g,size(g)-1)-point(g,0));
        }
        real max=size(a);
        for(int i=0; i <= nx; ++i) {
          real x=interp(a.x,b.x,i*dx);
          for(int j=0; j <= ny; ++j)
          max=max(max,size((x,interp(a.y,b.y,j*dy))));
        }
        scale=max > 0 ? maxlength/max : 1;
      } else scale=1;
      
      for(int i=0; i <= nx; ++i) {
        real x=interp(a.x,b.x,i*dx);
        for(int j=0; j <= ny; ++j) {
          real y=interp(a.y,b.y,j*dy);
          pair z=(x,y);
          if(all || cond(z)) {
            path g=scale(scale)*vector(z);
            if(truesize)
            draw(z,pic,g,p,arrow);
            else
            draw(pic,shift(-point(g,size(g)-1)/2)*shift(z)*g,p,arrow,margin);
          }
        }
      }
      return pic;
    }
    
    size(6cm);

    pair a = (-2,-2);
    pair b = ( 2, 2);

    path vector(pair z) {return (0,0)--(-z.y,z.x);}

    add(vectorfieldmid(vector,a,b,gray,arrow=Arrow(3)));

    path p = graph(new pair (real t) {return (cos(t),sin(t));}, 0, pi, Spline); 
    pen wirepen = MidnightBlue; 
    draw(p,wirepen);
    add(arrow(p,invisible,6,FillDraw(wirepen),Relative(0.25)));
    add(arrow(p,invisible,6,FillDraw(wirepen),Relative(0.75)));
    
    path p = graph(new pair (real t) {return (cos(t),-sin(t));}, 0, pi, Spline); 
    pen wirepen = DarkOrange; 
    draw(p,wirepen);
    add(arrow(p,invisible,6,FillDraw(wirepen),Relative(0.25)));
    add(arrow(p,invisible,6,FillDraw(wirepen),Relative(0.75)));
    dot(point(p,0));
    dot(point(p,size(p)-1)); 
    
    real eps = 0.4; 
    draw((a.x-eps,0)--(b.x+eps,0),Arrow());
    draw((0,a.y-eps)--(0,b.y+eps),Arrow()); 
  \end{asy}
  \caption{The vector field $\mathbf{F}(x,y) = \langle -y,x \rangle$
    and two semicircular paths\label{fig:swirl}}
\end{wrapfigure}

In general, the line integral of $\mathbf{F}$ over a path between two
points depends on the path, not just the starting and ending points.
For example, in Figure~\ref{fig:swirl}, the line integral along the
blue (top) path is positive, while the line integral along the orange
(bottom) path is negative.

However, there is an important class of vector fields which are
path-independent, meaning that the value of $\int_C \mathbf{F} \cdot
\d\mathbf{r}$ depends only on the starting and ending points of $C$.
These are the vector fields which can be written as the gradient of a
function from $\R^n$ to $\R^1$. For example, $\mathbf{F}(x,y,z) =
\langle -2x, -2y, z \rangle$ is the gradient of the function
\[
  f(x,y,z) = -x^2-y^2 + \tfrac{1}{2}z^2. 
\]
Such vector fields are called \textbf{conservative}. 

If we calculate the line integral of $\nabla f$ along a curve $C$
parametrized by $\mathbf{r}(t) = \langle r_1(t), r_2(t), r_3(t) \rangle$,
then the contribution from the portion of the curve from
$\mathbf{r}(t)$ to $\mathbf{r}(t+\Delta t)$ is*
\sidenote{* approximately, with an error that vanishes a  $\Delta t
  \to 0$}[1cm]
\[
  \langle \partial_x f, \partial_y f, \partial_z f \rangle \cdot
  \langle r_1'(t), r_2'(t), r_3'(t) \rangle \, \Delta t, 
\]
which by the chain rule is* the change in
$f(\mathbf{r})(t)$ over that interval. Therefore, the line
integral of $\nabla f$ along a path is equal to the change in $f$ from
the beginning to the end of the path.

\begin{theo}{Fundamental theorem for line integrals}{fund_theorem}
  If $C$ is a path from $\mathbf{a}$ to $\mathbf{b}$ and $f$ is a
  differentiable function, then 
  \[
    \int_C \nabla f \cdot \, {\d}\mathbf{r} = f(\mathbf{b}) - f(\mathbf{a}). 
  \]
\end{theo}

\begin{example}{}{}
  Suppose $\mathbf{F}(x,y,z) = \langle 2  x y^{3} z, 3  x^{2} y^{2}
  z + y, x^{2} y^{3} \rangle$ and that $C$ is circular arc from the
  origin to the point $(1,1,1)$ and passing through the point
  $(1/2,1/2,1)$. Find $\int_C \mathbf{F} \cdot d \mathbf{r}$. 
\end{example}

\begin{solution}
  Finding a parametrization for $C$ seems computationally messy.
  However, if $\mathbf{F}$ is conservative, then we can use
  Theorem~\ref{th:fund_theorem}. Integrating $2  x y^{3} z$ with
  respect to $x$, we see that if $\mathbf{F} = \nabla f$ for some
  function $f$, then we would have 
  \[
    f(x,y,z) = x^2 y^3 z + C_1(y,z), 
  \]
  where $C_1(y,z)$ denotes a function not depending on $x$. Similarly,
  we can integrate the second and third components with respect to $y$
  and $z$ to find that
  \begin{align*}
    f(x,y,z) &= x^2 y^3 z + \frac{1}{2}y^2 + C_2(x,z) \\
    f(x,y,z) &= x^2 y^3 z + C_3(x,y). 
  \end{align*}
  We see that these three conditions are simultaneously satisfied by
  the function $f(x,y,z) = x^2 y^3 z + \tfrac{1}{2}y^2$. So the
  desired line integral is equal to 
  \[
    f(1,1,1) - f(0,0,0) = \frac{3}{2} - 0  = \boxed{\frac{3}{2}}. 
  \]
\end{solution}

The following theorem provides a convenient way to check whether a
two-dimensional vector field is conservative. 

\begin{theo}{}{conservativetest}
  A vector field $\mathbf{F}(x,y) = \langle M(x,y), N(x,y)\rangle$
  which is differentiable on $\R^2$ is
  conservative if and only if
  \begin{equation} \label{eq:DE}
    \partial_x N = \partial_y M
  \end{equation}
\end{theo}
To see where \eqref{eq:DE} comes from, note that this equation follows directly
from Clairaut's theorem for conservative fields $\mathbf{F}$. 
So the more interesting aspect of Theorem~\ref{th:conservativetest} is
the converse direction: merely checking $\partial_y M = \partial N_x$
establishes existence or nonexistence of a gradient function. 

\begin{exercise}{}{}
  Show that the gravitational force in \eqref{eq:gravity} is
  conservative.
\end{exercise}

\begin{exercise}{}{}
  (i) Try to apply Theorem~\ref{th:conservativetest}  to the vector
  field
  \[
    \mathbf{F}(x,y) = \left\langle
      -\frac{y}{{x^2+y^2}},
      \frac{x}{{x^2+y^2}}
    \right\rangle. 
  \]  
  (ii) Show by plotting this vector field that it is not
  conservative. How does this square with Theorem~\ref{th:conservativetest}. 
\end{exercise}

\section{Green's theorem} \label{sec:greens}

\milink{greens_theorem_idea}{Green's theorem}

\begin{wrapfigure}[9]{R}[1cm]{5cm}
  \includegraphics[width=5cm]{figures/planimeter}
  \caption{A planimeter \label{fig:planimeter}}
\end{wrapfigure}

Is it possible to engineer a simple mechanical device that displays
the area bounded by a curve traced out on paper? This seems
surprising, since computing the area would seem to require some
inspection of the region inside the curve. However, the
\textit{planimeter}* can calculate area of a region based on the
motion of its wheels as its tip traverses the boundary of the region. The
design of the planimeter takes advantage of the following beautiful
relationship between line integrals along the boundary of a curve and
area integrals over the region it encloses. \sidenote{* Invented
  in 1854}[-1cm]

\begin{theo}{Green's theorem}{green}
  If $\mathbf{F} = \langle M, N\rangle$ is a vector field* on $\R^2$
  with continuous partial derivatives and if $D$ is a region bounded by a
  simple, counterclockwise oriented, piecewise smooth curve $C$, then
  \sidenote{* $\mathbf{F} = \mathbf{F}(x,y)$ and similarly for $M$
    and $N$; we will begin using these abbreviations for notational
    convenience; bear in mind that $\mathbf{F}$ always represents
  a vector field which implicitly depends on $(x,y)$ or $(x,y,z)$}
  \[
    \int_C \mathbf{F} \cdot \d\mathbf{r} = \iint_D \left(\partial_x N -
      \partial_y M \right) \, {\d}A. 
  \]
\end{theo}

\begin{example}{}{}
  Verify Green's theorem in the case where $D$ is the unit disk and
  $\mathbf{F}(x,y)= \langle 0, x \rangle$. 
\end{example}

\begin{solution}
  We parametrize the unit disk trigonometrically as $(\cos t, \sin
  t)$, and we calculate the line integral
  \[
    \int_0^{2\pi} \langle 0, \cos t \rangle \cdot \langle -\sin t, \cos
    t \rangle \, {\d}t = \int_0^{2\pi} \cos^2 t \, {\d}t = \pi. 
  \]
  This last integral can be done with a trick: note that 
  \[
    \int_0^{2\pi} (\cos^2 t +\sin ^2 t)\, {\d}t = \int_0^{2\pi} 1 \, {\d}t =
    2\pi. 
  \]
  However, the contributions of $\int_0^{2\pi} \cos^2 t\, {\d}t$ and
  $\int_0^{2\pi} \sin^2 t\, {\d}t$ are equal, since their graphs over the
  region of integration are the same up to a shift. So each is
  equal to $\pi$.

  \begin{center}
    \begin{asy}
      defaultpen(fontsize(6)); 
      size(4cm);
      import graph;
      
      real f(real t) {return cos(t)^2;}
      real g(real t) {return sin(t)^2;}
      path p = graph(f,0,2*pi,20,Spline);
      path q = graph(g,0,2*pi,20,Spline);
      transform t = shift((0,-1.2)); 
      fill(p--(2*pi,0)--(0,0)--cycle,softgreen);
      fill(t*(q--cycle),softred);
      label("$\cos^2t $",t*(pi/2,0.6),red);
      label("$\sin^2 t$",(pi,0.6),0.5*green);
      draw(p);draw(t*q);
      real eps = 0.1; 
      draw(t*((0,0)--(0,1+eps)));
      draw(t*((0,0)--(2*pi+eps,0)));
      draw(((0,0)--(0,1+eps)));
      draw(((0,0)--(2*pi+eps,0)));
      draw(Label("$2\pi$",Relative(1),align=S),(2*pi,0)--(2*pi,-eps));
    \end{asy}
  \end{center}
  The integrand for the double integral is $\partial_x N - \partial_y
  M = 1 - 0 = 1$, so the value of the double integral is the area of
  the unit disk, which is equal to $\pi$. Thus the conclusion of
  Green's theorem is satisfied. 
\end{solution}

\begin{tcolorbox}[title = Proving Green's theorem,
  colback=white!20, colframe=black!60, parbox = false]  
  \begin{minipage}[b]{0.38\textwidth}
    The idea of the proof of Green's theorem is to cut $D$ into small
    rectangles along grid lines (shown with small gaps for visual
    clarity). Green's theorem holds approximately on each small
    rectangle $R = [x,x + \Delta x] \times [y,y + \Delta y]$, because
    the left and right sides of $R$ contribute to
    $\iint_R \mathbf{F} \cdot \d\mathbf{r}$ approximately
\end{minipage}
\begin{minipage}[b]{0.6\textwidth}
  \hfill 
  \begin{asy}[width=9cm]
    void orientedbox(picture pic=currentpicture,
    real a, real b, real w, real h,
    pen p=currentpen,
    arrowbar arr=Arrow(2,position=0.48)) {
      draw(pic,(a,b)--(a+w,b),     p,arr);
      draw(pic,(a+w,b)--(a+w,b+h), p,arr);
      draw(pic,(a+w,b+h)--(a,b+h), p,arr);
      draw(pic,(a,b+h)--(a,b),     p,arr); 
    }
    
    int m=8, n=4;
    real w = 2;
    real h = 1;
    real k = 0.93; 
    
    for(int i=1;i<=m;++i) {
      for(int j=1;j<=n;++j) {
        orientedbox(w*i/m,h*j/n,k*w/m,k*h/n, (i+j) % 2 == 1 ? SeaGreen : MidnightBlue); 
      }
    }
    
    real eps = 0.03; 
    orientedbox(w/m-3*eps/4,h/n-3*eps/4,w+eps,h+eps,arr=Arrow(4,position=0.5)); 
  \end{asy}
\end{minipage}
\[
  \overbrace{N(x+\Delta x,y) \Delta y}^{\text{integral over right
    side}} -   \overbrace{N(x, y) \Delta y}^{\text{integral over left
    side}} \approx (\partial_x N)(x,y) \, \Delta x \, \Delta y. 
\]
Similarly, the contribution of the top and bottom sides is
$-(\partial_y M)(x,y)\, \Delta x \, \Delta y$. So all together,
\textbf{circulation} of $\mathbf{F}$ around $R$ is $(\partial_xN -
\partial_y M)\, \Delta x \, \Delta y$. This is also approximately
equal to the integral of $(\partial_xN -
\partial_y M)$ over $R$, since we may regard $\partial_xN -
\partial_y M$ as constant over $R$. 

The line integrals of $\mathbf{F}$ over the small rectangles sum to
the line integral of $\mathbf{F}$ around the boundary* of $D$, because
each interior segment is integrated along twice (once for each
adjoining rectangle) and in opposite directions. These contributions
sum to zero, leaving only the integrals along the outer edges. Since
these outer edges fit together to form $\partial D$, the line integrals
along them sum to the line integral along $\partial D$. \sidenote{* In
  other words, the circulation around the boundary of a region is
  \textbf{additive}.}[-15mm] 

Since the integral of $\partial_xN - \partial_y M$ over $D$ is also
equal to the sum of the integral of  $\partial_xN - \partial_y M$
over the small rectangles, the (approximate) Green's theorem for the
small rectangles implies (approximate) Green's theorem for
$D$. Letting the sizes of the rectangles tend to 0, this approximation
becomes exact and yields Green's theorem. 
\end{tcolorbox}

\begin{exercise}{}{}
  Use Green's theorem to find the line integral of $\mathbf{F} = \langle \sqrt{x^2 +
    1}, \arctan x \rangle$ along a counterclockwise traversal of the
  triangle with vertices $(0,0)$, $(1,0)$, and $(0,1)$. 
\end{exercise}

\begin{exercise}{}{}
  Use Green's theorem to find the area under each arch of the cycloid
  shown below.
  \begin{center}
    \begin{asy}
      size(8cm);
      import graph;
      
      real a = 4*pi+0.6;
      real b = 2.5;
      draw((0,0)--(0,b),Arrow(3));
      draw((0,0)--(a,0),Arrow(3)); 
      
      draw(Label("$\mathbf{r}(t) = \langle t - \sin t, 1- \cos t\rangle$",Relative(0.4),fontsize(10),align=6*N),graph(new pair(real t) {return (t-sin(t), 1-cos(t));}, 0, 4*pi+1.0, Spline),Arrow(2)); 
    \end{asy}
  \end{center}
\end{exercise}

\section{Surface integrals and flow} \label{sec:surf}

\subsection{Surface integrals}

\milink{surface_integral_scalar_function_introduction}{surface integrals}

What is the average temperature on the surface of the earth? Let's
overlook the scientific challenges of this problem and imagine that
the earth is a sphere $S$ and that we have a reading of the
temperature $T$ at every point on its surface at a particular point in
time. What do we do with this information to find the average*
temperature at that time?

\sidenote{* The average value of a function over a region is the
  integral of the function over that region divided by the region's
  length/area/volume}[-5mm]
We can use the same approach we use throughout calculus when studying
quantities which are additive and continuously varying: split $S$ into
tiny patches over which $T$ may be treated as constant,
multiply the area* of each patch by the value of $T$ somewhere on that patch, and
sum the resulting products. As the size of the patches tends to zero,
we expect this sum to converge to some limiting value, and we can
declare that limit to be the value of the \textbf{surface integral}* of $f$ over
$S$, denoted $\iint_S f \, {\d}A$. \sidenote{* Or scalar surface
  integral, to distinguish from vector surface integral introduced in
  the next subsection}[5mm]

\begin{example}{}{surf}
  Find the surface integral of $f(x,y,z) = 2x^2 +2y^2 + 2z^2$ over the
  unit sphere $S$. 
\end{example}

\begin{solution}
  This function is equal to 2 everywhere on the unit sphere, so the
  integral is
  \[
    \iint_{S} 2 \,\d A =     2 \iint_{S} 1 \,\d A = 2 \times \text{surface
      area}(S) = 8\pi, 
  \]
  since the surface area formula for a sphere of radius $r$ is $4\pi
  r^2$. 
\end{solution}

Example~\ref{exam:surf} was special because the function happened to
be constant over the surface. A general method for computing surface
integrals can be developed in a manner analogous to
change-of-variables technique in
Section~\ref{sec:changeofvariables} (see Exercise~\ref{exer:surfformula}).
However, surface integrals can be
calculated in the same manner as double integrals if the surface is
contained in a plane (or if it consists of several pieces each of
which is planar---see Exercise~\ref{exer:xyz}). 

\begin{exercise}{}{xyz}
  Find the surface integral of $f(x,y,z) = xyz$ over the rectangular
  prism $[0,1] \times [0,2] \times [0,3]$. 
\end{exercise}

\begin{exercise}{$\star$ (Surface integral formula)}{surfformula}
  If $D$ is a planar domain and $\mathbf{r}:D \to \R^3$ is a function
  mapping $D$ bijectively onto a surface $S$ in $\R^3$ (in other words,
  $\mathbf{r}$ is a \textit{parametrization}* of $S$), then
  \milink{parametrized_surface_introduction}{surface parametrization}
  \begin{equation} \label{eq:surfformula} 
    \iint_S f \, {\d}A = \iint_D f(\mathbf{r}(u,v)) |\partial_u
    \mathbf{r} \times \partial_v \mathbf{r}| \, {\d}A. 
  \end{equation}
  Explain why $|\partial_u
    \mathbf{r} \times \partial_v \mathbf{r}|$ is the appropriate
    Jacobian, and use \eqref{eq:surfformula} to calculate the average value of
    $z$ over the top half of the unit sphere.

    It might help to note that
    \[
      \langle \cos \theta \sin \phi, \sin \theta \sin \phi, \cos
      \phi
      \rangle, 
    \]
    where $(\theta,\phi)$ ranges over $[0,2\pi] \times
    [0,\tfrac{\pi}{2}]$, is a parametrization of the upper unit
    hemisphere. 
\end{exercise}

\subsection{Flow}

\milink{surface_integral_vector_field_introduction}{surface integrals of a
  vector field}

\begin{example}{}{}
  \begin{minipage}[t]{0.7\textwidth}
    Consider a constant-velocity river flowing through a net as
    shown. Find the volume of water flowing through the net per unit
    time, in terms of the area $A$ of the net, the velocity $v$, and
    the angle $\theta$ between the direction of the river's flow and a
    vector normal to the plane of the net.
  \end{minipage}
  \begin{minipage}[t]{0.29\textwidth}
    \begin{lrbox}{\asybox}
    \begin{asy}[width=4cm]
      import three;
      
      currentlight.background = softblue;
      currentprojection = perspective(20,12,8); 

      int n = 10;
      pen waterblue = rgb(0.8,0.8,0.99);
      
      triple A = (3,0,0);
      triple B = (3,0,1);
      triple C = (1,1,1);
      triple D = (1,1,0);
      
      draw(A--B--C--D--cycle);
      
      real t; 
      for(int i=1; i<n; ++i){
        t = i/n; 
        draw(interp(A,B,t)--interp(D,C,t),gray);
        draw(interp(B,C,t)--interp(A,D,t),gray); 
      }
      
      pen waterpen = waterblue + opacity(0.4); 
      
      draw(box((0,0,0),(4,1,1)),MidnightBlue);
      
      int n = 2; 
      for(int i=-n; i<=n; ++i){
        for(int j=-n; j<=n; ++j){
          draw((0,1/2+i/5,1/2+j/5)--(4,1/2+i/5,1/2+j/5),waterpen+linewidth(2.0)); 
        }
      }
    \end{asy}
  \end{lrbox} \raisebox{\dimexpr -\height + 1.5ex \relax}{\usebox{\asybox}}
\end{minipage}
\end{example}

\begin{solution}
  \begin{minipage}{0.75\textwidth}
    Imagine letting the water flow for one time
    unit and then taking a snapshot. The locations of all the water
    molecules which have flowed through the net during this period
    occupy a parallelepiped, as shown. The base area of this
    parallelepiped is $A$, while its height is equal to*
    $v \cos\theta$. Therefore, the volume of water passing through the
    net per unit of time is $\boxed{Av \cos
      \theta}$. \sidenote{* This is a right-triangle
      trigonometry exercise}
  \end{minipage}
  \begin{minipage}{0.23\textwidth}
    \begin{asy}[width=3cm]
      import three;
      currentlight.background = softyellow;
      currentprojection = perspective(20,12,8);
      
      int n = 10;
      pen waterblue = rgb(0.8,0.8,0.99);
      pen wirepen = MidnightBlue; 

      triple A = (3,0,0);
      triple B = (3,0,1);
      triple C = (1,1,1);
      triple D = (1,1,0);
      
      path3 a = A--B--C--D--cycle, b = shift((1/2,0,0))*a;
      path3 c = B--C--shift((1/2,0,0))*C--shift((1/2,0,0))*B--cycle;
      path3 d = C--D--shift((1/2,0,0))*D--shift((1/2,0,0))*C--cycle;
      
      draw(a,wirepen);
      draw(b,wirepen);
      draw(c,wirepen); 
      
      draw(surface(a),waterblue+opacity(0.5));
      draw(surface(b),waterblue+opacity(0.5));
      draw(surface(c),waterblue+opacity(0.5));
      draw(surface(d),waterblue+opacity(0.5)); 
      
      draw(box((0,0,0),(4,1,1)),MidnightBlue);
    \end{asy}
  \end{minipage}
\end{solution}

Let's define the vector $\mathbf{A}$ whose length is equal to $A$ and
whose direction is orthogonal to the net. Then the \textbf{flow}
$Av \cos \theta$ can be written as
\[
  \text{flow} = \mathbf{A} \cdot \mathbf{v}, 
\]
where $\mathbf{v}$ is the river's velocity vector. 

\begin{example}{}{waterflow}
  Suppose that the velocity field of a body of water is given by
  $\mathbf{F} = \langle -2y, 4z, x\rangle$ (in meters per second) and
  that a rectangular net is positioned in the water with corners at
  $(1,1,3)$, $(1,4,3)$, $(1,4,5)$, and $(1,1,5)$ (in meters). Find the
  volume of water flowing through the frame of the net per second. 
\end{example}

\begin{solution}
  Since the velocity field isn't constant, we divide the net into
  small patches and treat the velocity as constant on each one. Since
  the rectangle is contained in the plane $x=1$, the vector
  $\langle 1, 0, 0 \rangle$ is normal to the rectangle. Therefore, the flow
  through a small patch of area $\Delta A$ located at $(x,y,z)$ is
  approximately equal to
  \[
    \mathbf{A} \cdot \mathbf{v} = \big\langle \Delta A, 0, 0 \big\rangle \cdot
    \big\langle -2y, 4z, x \big\rangle = -2y \, \Delta A. 
  \]
  If we sum the flow through each patch across the whole rectangular
  region occupied by the net, we get a Riemann sum that converges as
  $\Delta A \to 0$ to
  \[
    \int_1^4 \int_3^5 (-2y) \, \d x \, \d y = -30. 
  \]
  Therefore, the volume of water is \boxed{30} cubic meters per second, and the
  net flow is in the direction \textit{towards} the side facing the $yz$-plane. 
\end{solution}

The ideas in Example~\ref{exam:waterflow} yield the following
definition, which develops the notion of vector field integration over
surfaces in terms of the scalar surface
integral.\sidenote{A surface with a distinguished side is called an
  \textit{oriented surface}, and not every surface is orientable (for
  example, the M\"obius strip)} 

\begin{defn}{}{flow}
  The \textbf{flow} of a vector field $\mathbf{F}$ through a surface
  $S$ from one side $\mathfrak{s}$ to the other side $\mathfrak{t}$ is defined by
  \[
    \iint_S \mathbf{F} \cdot \d\mathbf{A} =  \iint_S \mathbf{F} \cdot \mathbf{n}
    \, {\d}A, 
  \]
  where $\mathbf{n}=\mathbf{n}(x,y,z)$ is a vector which is
  orthogonal to $S$ at each point $(x,y,z)$ and points in the
  direction from $\mathfrak{s}$ to $\mathfrak{t}$. 
\end{defn}

\begin{exercise}{}{}
  Find the flow of the vector field $\mathbf{F} = \langle x^2, y^2, z^2 \rangle$
  through the unit sphere. 
\end{exercise}

\section{Divergence and curl} \label{sec:divcurl} 

One of the classic and famous vector calculus books is called \textit{div,
  grad, curl, and all that} by H.M. Schey.* We have already discussed the gradient,
and in this section we will develop the two other fundamental vector
calculates derivative operators: divergence and curl. We will
emphasize the underlying physical intuition. \sidenote{* Highly recommended}[-5mm]

\subsection{Divergence}

\milink[-6mm]{divergence_idea}{divergence} 

\begin{defn}{Divergence}{divergence}
  The \textbf{divergence} of a vector field $\mathbf{F} = \langle M, N, P \rangle$ is a scalar function defined by
\[
  \nabla \cdot \mathbf{F} = \partial_x M  +\partial _y N + \partial_z
  P. 
\]
\end{defn}

For example, the divergence of $\langle x^2, xy, z \rangle$ is
$2x + x + 1 = 3x + 1$. The divergence of $\mathbf{F}$ can be
interpreted physically as the \textbf{net flow} of $\mathbf{F}$ out of
a small region located at $(x,y,z)$ per volume of that region. 

\sidenote{* We dot the normal vector $\langle 0, 0, 1
  \rangle$ with $\langle M, N, P \rangle$ to account for the cosine
  factor}
To see this, note that if we put a small box of dimensions $\Delta x, \Delta y, \Delta z$
around $(x,y,z)$, then the net flow through the top of the box is
approximately* $P(x,y,z+\Delta z)\Delta x \Delta y$, and the net flow
through the bottom of the box is approximately
$P(x, y, z)\, \Delta x \,\Delta y$. Thus the difference is approximately
$(\partial_z P)(x,y,z) \, \Delta x \,  \Delta y \, \Delta z$, and he
difference per unit volume is just $\partial_z P$. Similarly, the front/back
and left/right sides contribute $\partial_x M$ and $\partial_y N$ to
the net flow density through the small box.


\begin{example}{}{div}
  \begin{minipage}[t]{0.5\textwidth}
    Figure out where $\nabla \cdot \mathbf{F}$ is positive for the
    vector field $\mathbf{F}$ shown.
  \end{minipage}
  \begin{minipage}[t]{0.48\textwidth}
    \begin{lrbox}{\asybox}
      \begin{asy}[width=7cm]
        int foobar; 
        import graph;
        picture vectorfieldmid(path vector(pair), pair a, pair b,
        int nx=nmesh, int ny=nx, bool truesize=false,
        real maxlength=truesize ? 0 : maxlength(a,b,nx,ny),
        bool cond(pair z)=null, pen p=currentpen,
        arrowbar arrow=Arrow, margin margin=PenMargin)
        {
          picture pic;
          real dx=1/nx;
          real dy=1/ny;
          bool all=cond == null;
          real scale;
          
          if(maxlength > 0) {
            real size(pair z) {
              path g=vector(z);
              return abs(point(g,size(g)-1)-point(g,0));
            }
            real max=size(a);
            for(int i=0; i <= nx; ++i) {
              real x=interp(a.x,b.x,i*dx);
              for(int j=0; j <= ny; ++j)
              max=max(max,size((x,interp(a.y,b.y,j*dy))));
            }
            scale=max > 0 ? maxlength/max : 1;
          } else scale=1;
          
          for(int i=0; i <= nx; ++i) {
            real x=interp(a.x,b.x,i*dx);
            for(int j=0; j <= ny; ++j) {
              real y=interp(a.y,b.y,j*dy);
              pair z=(x,y);
              if(all || cond(z)) {
                path g=scale(scale)*vector(z);
                if(truesize)
                draw(z,pic,g,p,arrow);
                else
                draw(pic,shift(-point(g,size(g)-1)/2)*shift(z)*g,p,arrow,margin);
              }
            }
          }
          return pic;
        }
        size(8cm); 
        
        real eps = 0.2; 
        
        pair a=(0,0);
        pair b=(1,1);
        
        path vector(pair z) {return (0,0)--(z.x^2,-z.y^2);}
        
        draw(a--(b.x+eps,0),Arrow(4));
        draw(a--(0,b.y+eps),Arrow(4));
        
        real eps = 0.03; 
        draw(Label("1",Relative(1)),(b.x,0)--(b.x,-eps),align=S);
        draw(Label("1",Relative(1)),(0,b.y)--(-eps,b.y),align=W);
        
        add(vectorfieldmid(vector,(eps,eps),b,Arrow(3)));
      \end{asy}
    \end{lrbox} \raisebox{\dimexpr -\height + 1.5ex \relax}{\usebox{\asybox}}
  \end{minipage}
\end{example}

\begin{solution}
  We can see that in the top left of the diagram that there is more
  flow into each small region than out of it, since the vectors are
  downward-pointing and longer than the vectors below them. Therefore,
  the divergence is negative in the top left.

  Similarly, the divergence is positive in the bottom-right part of
  the figure. The dividing line between regions of positive and
  negative divergence is $y = x$, since points along that line have
  vectors of equal length pointing towards and away from them.
\end{solution}

\begin{exercise}{}{}
  Find the divergence of the gravitational vector field in
  \eqref{eq:gravity}. 
\end{exercise}

\begin{exercise}{}{}
  Look at a vector plot* to figure out where
  $\nabla \cdot \mathbf{F} > 0$, using the approach of
  Example~\ref{exam:div}, for the vector field
  $\mathbf{F} = \langle xy, y^2 \rangle$. Then evaluate
  $\nabla \cdot \mathbf{F}$ and find where
  $\nabla \cdot \mathbf{F} > 0$ algebraically.
  \sidenote{\href{https://cocalc.com/projects/7925f475-a0bd-4621-b78a-466c24c7863c/files/vector_plot.ipynb}{\cocalc}
    for drawing a vector plot}
\end{exercise}

\subsection{Curl}

\milink[-6mm]{curl_idea}{curl} 

\begin{defn}{Curl}{curl}
  The \textbf{curl} of a vector field $\mathbf{F} = \langle M, N, P \rangle$ is
  is a vector field on $\R^3$ defined by \renewcommand\arraystretch{1.4}
  \[
    \nabla \times \mathbf{F} =
    \left|
      \begin{array}{ccc}
        \mathbf{i} & \mathbf{j} & \mathbf{k} \\
        \partial_x  & \partial_y & \partial_z \\
              M         &       N       &      P
      \end{array} 
    \right| = \big\langle \partial_y P - \partial_z N, \:
    -\partial_x P  + \partial_z M, \:
    \partial_x N - \partial_y M \big\rangle. 
  \]
\end{defn}

We've already seen the third component, $\partial_x N - \partial_y M$,
in Green's theorem. In the context, $\partial_x N - \partial_y M$
measures the line integral (per unit of enclosed area) around a small
loop perpendicular to the $z$-axis. This quantity is called
\textbf{circulation density}. The other two components of
$\nabla \times \mathbf{F}$ supplement this information by providing the
circulation density with respect to the $x$ and $y$ directions.

We can visualize this idea physically by interpreting the vector field
as a flow velocity and imagining placing a small paddle wheel (with an
axis of rotation in the $x$, $y$, or $z$ direction) at a particular
point in this field. The corresponding component of the curl measures
how rapidly and in which direction this paddle wheel turns.

\begin{example}{}{}
  \begin{minipage}[t]{0.5\textwidth}
    Consider the vector field $\mathbf{F}$ shown. Find the
    sign of the $z$-component of the curl of $\mathbf{F}$ at any point
    in the $xy$-plane.
  \end{minipage}
  \begin{minipage}[t]{0.49\textwidth}
    \begin{lrbox}{\asybox}
      \begin{asy}[width=7cm]
        import multitools;
        currentprojection=perspective(5,2.8,1.1);
        
        currentlight.background = softblue;
        
        path3 gravity(triple t){
          return length(t) == 0 ? O--O : O--(t.x*t.z,t.x^2+t.y*t.z,t.z);
        }
        
        triple A = (0,0,0);
        triple B = (1,1,1);
        
        add(VectorPlot3D(gravity,A,B,4,4,4,maxlength=0.2,minlength=0.025,p=LightSeaGreen,Arrow3(6)));
        
        real l = 1.2;
        pen axispen = gray+opacity(0.5);
        
        draw(O--O+l*X,axispen,Arrow3());
        draw(O--O+l*Y,axispen,Arrow3());
        draw(O--O+l*Z,axispen,Arrow3());
        
        real t = 0.13; 
        real a = t+0.495, b = t+0.505;
        real c = t+0.4, d = t+0.6; 
        
        draw(shift(0,0,-0.01)*extrude((c,a)--(c,b)--(d,b)--(d,a)--cycle,0.02*Z),LightBlue);
        draw(shift(0,0,-0.01)*extrude((a,c)--(b,c)--(b,d)--(a,d)--cycle,0.02*Z),LightBlue);
        draw(shift(0,0,-0.01)*extrude(circle(((a+b)/2,(a+b)/2),0.015),0.08*Z),LightBlue);      
      \end{asy}
    \end{lrbox} \raisebox{\dimexpr -\height + 1.5ex}{\usebox{\asybox}}
\end{minipage}
\end{example}

\begin{solution}
  We can see that if we place a small paddlewheel at a point of interest (as
  shown in the figure above) that it will rotate in the
  counterclockwise direction, because the vectors on the right
  (meaning the side where $x$ is larger) push harder than the vectors
  on the left. Therefore, the $z$-component of the curl
  is $\boxed{\text{positive}}$. 
\end{solution}

When we studied gradients, we learned that directional
derivatives of a function in the coordinate directions determine its directional
derivatives in all directions: the derivative in the $\langle u_1, u_2
\rangle$ direction is equal a linear combination with weights $u_1$
and $u_2$ of the derivatives in
the coordinate directions. The same idea holds for the curl: 
\begin{theo}{}{curl} \bang{-3mm}
  The circulation density of a vector field $\mathbf{F}$ with respect
  to a unit vector $\mathbf{u} $ is
  equal to $\left(\nabla \times \mathbf{F}\right) \cdot \mathbf{u}$. 
\end{theo}

\begin{example}{}{}
  Find the orientation for which a paddle wheel at the point $(1,1,1)$
  in the velocity field
  $\langle xyz,x^2 - y, z \rangle$ which will maximize how fast it
  spins. 
\end{example}

\begin{solution}
  We calculate the curl:
  $\nabla \times \mathbf{F} = \langle 0, xy, -xz + 2x \rangle$, which
  at the point $(1,1,1)$ is equal to $\langle 0,1,1 \rangle$. Since
  the dot product of a fixed vector $\mathbf{v}$ with a unit vector is
  maximized when the unit vector is aligned with $\mathbf{v}$, we see
  that the paddle wheel should be oriented so that its axis is in the
  direction
  $\boxed{\left\langle 0, \tfrac{1}{\sqrt{2}}, \tfrac{1}{\sqrt{2}} \right\rangle}$.
\end{solution}

\begin{exercise}{}{}
  Calculate $\nabla \times \mathbf{F}$, where $\mathbf{F} = \langle e^{\sin \log x} + y^2,
  -2z, y^3 + \cos z \rangle$. 
\end{exercise}

\begin{exercise}{}{}
  Show that the curl of a conservative vector field is zero. 
\end{exercise}

\section{Divergence theorem} \label{sec:divtheorem}

\milink{divergence_theorem_idea}{the divergence theorem}

\begin{wrapfigure}[8]{R}[1cm]{5cm}
  \centering
  \begin{asy} 
    size(5cm);
    import graph3;
    
    currentprojection = orthographic(6,1,2); 

    void drawcube(real  a, real b, real c){
      draw(shift((a,b,c))*unitcube,LightSeaGreen+opacity(0.3));
      draw(box((a,b,c),(a+1,b+1,c+1)));
      real o = 0.1;
      pen arrowpen = linewidth(2)+MidnightBlue; 
      draw((a+1/2,b+1/2+o,c+1/4)--(a+1/2,b+1/2+o,c-1/4),arrowpen,Arrow3()); //bottom 
      draw((a+1/2,b+1/2-o,c+3/4)--(a+1/2,b+1/2-o,c+5/4),arrowpen,Arrow3()); //top
      draw((a+1/2,b+1/4,c+1/2+o)--(a+1/2,b-1/4,c+1/2+o),arrowpen,Arrow3());
      draw((a+1/2,b+3/4,c+1/2-o)--(a+1/2,b+5/4,c+1/2-o),arrowpen,Arrow3());
      draw((a+1/4,b+1/2,c+1/2)--(a-1/4,b+1/2,c+1/2),arrowpen,Arrow3());
      draw((a+3/4,b+1/2,c+1/2)--(a+5/4,b+1/2,c+1/2),arrowpen,Arrow3()); 
    }
    
    int n = 3; 
    
    for(int i=1; i<= 1; ++i){
      for(int j=1; j<= n; ++j){
        for(int k=1; k<= n; ++k){
          drawcube(i,j,k); 
        }
      }
    }
  \end{asy}
  \caption{The sum of the flows of $\mathbf{F}$ (not shown) out of
    each cell is equal to the flow out of the whole box
    \label{fig:divergencetheorem}}
\end{wrapfigure}


Suppose that $\mathbf{F}: \R^3 \to \R^3$ is a vector field. As
discussed in the previous section, the divergence of $\mathbf{F}$ at
each point measures the net flow of $\mathbf{F}$ out of a small region
around that point, per unit volume. Note that to integrate
$\nabla \cdot \mathbf{F}$ over a region $D$ in $\R^3$, we divide $D$
into many small pieces, multiply the volume of each piece by the value
of $\nabla \cdot \mathbf{F}$ there, and sum the results. The
contribution of each piece is the net flow out of that
piece*, so when we add the contributions of all the pieces we get the
net flow out of $D$. This idea is called the \textit{divergence
  theorem}.  \sidenote{* The contribution of each piece is
  the divergence somewhere in the piece times its volume, and the
  divergence equals flow per unit volume, so the product is the net
  flow out of the piece}[-1cm]


\begin{theo}{Divergence theorem}{divergence}
  If $\mathbf{F} : \R^3 \to \R^3$ is a vector field with continuous
  partial derivatives and $D$ is a region in $\R^3$ bounded by a
  piecewise smooth surface $S = \partial D$, then 
  \[
    \overbrace{\iiint_D \nabla \cdot \mathbf{F} \,
      \d V}^{\text{pointwise net flow density
        integrated over $D$}} = \overbrace{\iint_{\partial D}
    \mathbf{F} \cdot \, {\d}\mathbf{A}}^{\text{total flow through
      $\partial D$}}. 
  \]
\end{theo}

\begin{example}{}{}
  Verify that the divergence theorem holds in the case where
  $\mathbf{F} = \langle x^2, 3z^2, 2z^2 + y^2 \rangle$ and $D = [0,1]^3$. 
\end{example}

\begin{solution}
\begin{minipage}{0.7\textwidth}
    The divergence of $\mathbf{F}$ is $2x + 4z$, so the divergence theorem
    asserts that
    \[
    \iiint_D (2x + 4z) \, {\d}V = \iint_{\partial D} \langle x^2, 3z^2,
    2z^2+y^2 \rangle \cdot \d\mathbf{A}. 
  \]
  The left-hand side equals
  \[
    \int_0^1     \int_0^1     \int_0^1 (2x + 4z)  \, \d x \, \d y \, {\d}z =
    3. 
  \]
    To evaluate the right-hand side directly, we split the boundary of
  the cube into its six square faces. For the
  top face, we get 
\end{minipage}
\begin{minipage}{0.29\textwidth}
  \begin{asy}[width=4.5cm]
    import multitools;
    currentprojection=perspective(5,2.5,2);
    currentlight.background = softyellow; 

    path3 v(triple t){
      real x = t.x, y = t.y, z = t.z; 
      return O--(x^2, 3*z^2, 2*z^2 + y^2); 
    }
    
    triple A = (0,0,0);
    triple B = (1,1,1);
    add(VectorPlot3D(v,A,B,5,5,5,maxlength=0.2,minlength=0.025,p=LightSeaGreen));
    
    draw(box(O,X+Y+Z)); 
    
    real l = 1.1;
    pen axispen = gray+opacity(0.5);
    triple newO = O; 
    draw(newO--newO+l*X,axispen,Arrow3());
    draw(newO--newO+l*Y,axispen,Arrow3());
    draw(newO--newO+l*Z,axispen,Arrow3()); 
  \end{asy}
\end{minipage}
  
  \[
    \iint_{\text{top face}} \mathbf{F} \cdot \d\mathbf{A} =
    \iint_{\text{top face}} \langle x^2, 3z^2, 2z^2 + y^2 \rangle \cdot \langle 0, 0, 1 \rangle \d A =
    \int_{0}^1\int_0^1 (2+y^2) \, \d x \, \d y = \frac{7}{3}, 
  \]
  where we've substituted $1$ for $z$ since $z=1$ for every point in
  the top face. Likewise, the integral over the top face is
  \[
    -\int_{0}^1\int_0^1 (0+y^2) \, \d x \, \d y = -\frac{1}{3}, 
  \]
  where the negative sign comes from the fact that the
  outward-pointing normal on the bottom face is
  $\langle 0, 0, -1 \rangle$.
  
  Similarly, the integral over the $x=1$ face is
  \[
    \int_{0}^1\int_0^1 1 \, \d y \, {\d}z = 1, 
  \]
  while the $x=0$ face contributes 0. The $y=1$ face yields 
  \[
    \int_{0}^1\int_0^1 3z^2 \, \d x \, {\d}z = 1, 
  \]
  and the $y=0$ face gives $-\int_0^1 \int_0^1 3z^2 \, \d x \, {\d}z =
  -1$. Indeed,
  \[
    3 \stackrel{\checkmark}{=} \frac{7}{3} +\left(- \frac{1}{3}\right) + 1 + 0 + 1 + (-1). 
  \]
\end{solution}

\begin{exercise}{}{}
  Verify the divergence theorem in the case where $\mathbf{F} =
  \langle x^2, y^2, z^2 \rangle$ and $S$ is the unit sphere. 
\end{exercise}

\begin{exercise}{}{weirdsurface}
  \begin{minipage}[t]{0.5\textwidth}
    Consider the vector field
    $\mathbf{F} = \langle x^3, xz, 1-3zx^2 \rangle$. Verify that the
    divergence of $\mathbf{F}$ is everywhere zero. Then use the
    divergence theorem to calculate the flow of $\mathbf{F}$ through
    the surface $S$ shown. Note that this is not a closed surface: it
    excludes the square $[0,1]^2 \times \{0\}$ on the bottom.
  \end{minipage}
  \begin{minipage}[t]{0.48\textwidth}
    \begin{lrbox}{\asybox}
    \begin{asy}[width=7cm]
    import graph3;
    currentprojection=perspective(5,2.5,2);
    currentlight.background = softgreen; 
    
    triple f(pair p){
      real x = p.x, y = p.y;
      return (x,y,10*x*y*(1-x)*(1-y)); 
    }
    
    real l = 1.1;
    pen axispen = gray+opacity(0.5);
    triple newO = O; 
    draw(newO--newO+l*X,axispen,Arrow3());
    draw(newO--newO+l*Y,axispen,Arrow3());
    draw(newO--newO+l/2*Z,axispen,Arrow3()); 
    
    draw(surface(f,(0,0),(1,1),20,Spline),LightSeaGreen+opacity(0.7),MidnightBlue+thin()); 
    draw((0,0,0)--(0,1,0)--(1,1,0)--(1,0,0)--cycle);

    real eps = 0.05; 
    draw(Label("1",Relative(1.0),align=S),(1,0,0)--(1,0,-eps));
    draw(Label("1",Relative(1.0),align=S),(0,1,0)--(0,1,-eps)); 
  \end{asy}
\end{lrbox}\raisebox{\dimexpr -\height + 1.5ex \relax}{\usebox{\asybox}}
\end{minipage}
\end{exercise}

\newpage 

\section{Stokes' theorem} \label{sec:stokes} 

\milink{stokes_theorem_idea}{Stokes' theorem}

\begin{wrapfigure}[12]{R}[1cm]{5cm}
  \begin{asy}
    size(5cm);
    import graph3;
    
    currentprojection = orthographic(5,2.45,2);
    real gap = 0.0075; 
    
    material surfacemat = material(opacity=0.4,diffusepen=0.5*black+0.5*LightSeaGreen,
    emissivepen=0.5*black+0.5*LightSeaGreen,shininess=0.7); 
    
    triple f(pair p){
      real x = p.x, y = p.y;
      return (x,y,10*x*y*(1-x)*(1-y)); 
    }
    
    path3 segment(pair p, pair q){
      return graph(new triple (real t) {return f(interp(p,q,t));},0,1,10,Spline); 
    }
    
    path3 segment2(pair p, pair q){
      return graph(new triple (real t) {return (f(interp(p,q,t)).x,
        f(interp(p,q,t)).y,
        f(interp(p,q,t)).z - gap);},0,1,Spline); 
    }
    
    void orientedbox(picture pic=currentpicture,
    path3 segment(pair, pair)=segment, 
    real a, real b, real w, real h,
    pen p=currentpen,
    arrowbar3 arr=Arrow3(3.5,position=5)) {
      draw(pic,segment((a,b),(a+w,b)),     p,arr);
      draw(pic,segment((a+w,b),(a+w,b+h)), p,arr);
      draw(pic,segment((a+w,b+h),(a,b+h)), p,arr);
      draw(pic,segment((a,b+h),(a,b)),     p,arr); 
    }
    
    int m=8, n=8;
    real w = 1;
    real h = 1;
    real k = 0.94;
    
    draw(surface(new triple (pair p) {return (f(p).x,f(p).y,f(p).z-gap);},(0,0),(1,1),Spline),surfacemat); 
    
    for(int i=0;i<m;++i) {
      for(int j=0;j<n;++j) {
        orientedbox(w*i/m,h*j/n,k*w/m,k*h/n, (i+j) % 2 == 1 ? 0.3*black+0.7*SeaGreen : MidnightBlue); 
      }
    }
    
    real eps = 0.03; 
    orientedbox(segment2, 0,0,w,h,MidArrow3(DefaultHead2)); 
  \end{asy}
  \caption{The circulation of $\mathbf{F}$ (not shown) around each patch sums to
    the circulation around the boundary of the surface\label{fig:Stokes}}
\end{wrapfigure}
  
The planar domain $D$ in Green's theorem can be thought of as a
surface $S$, in which case the conclusion of Green's theorem can be
written as
\[
  \iint_S \nabla \times \mathbf{F} \cdot
  \d\mathbf{A} =
  \int_{\partial S} \mathbf{F} \cdot
  \d\mathbf{r}. 
\]
The argument for Green's theorem now applies even if $S$ doesn't lie
in a plane, because Theorem~\ref{th:curl} tells us that
$\nabla \times \mathbf{F} \cdot \d\mathbf{A}$ measures the circulation
around a small patch $\d A$ of the surface $S$. As we discussed for
Green's theorem, if you divide a surface into many small patches and
sum the circulations around all of them, you get the circulation
around the boundary of the surface.  This generalization of Green's
theorem is known as \textit{Stokes' theorem}.* \sidenote{* If you
  take a differential geometry course, you will learn a far more
  general result of the same name which implies Green's theorem and
  the divergence theorem as special cases}
  
\begin{theo}{Stokes' theorem}{stokes}
  If $\mathbf{F} : \R^3 \to \R^3$ is a vector field with continuous
  partial derivatives and $S$ is a surface in $\R^3$, then
  \[
    \overbrace{\iint_S \nabla \times \mathbf{F} \cdot
      \d\mathbf{A}}^{\text{pointwise circulation density integrated
        over $S$}} =
    \overbrace{\int_{\partial S} \mathbf{F} \cdot
      \d\mathbf{r}}^{\text{circulation around $\partial S$}}. 
  \]
\end{theo}

\begin{example}{}{stokes}
  Let $\mathbf{F} = \langle x \sin (\pi y), e^x, -\cos(\pi z)
  \rangle$. Find the flow of $\nabla \times \mathbf{F}$ through the
  surface shown in Exercise~\ref{exer:weirdsurface}.
\end{example}

\begin{solution} 
  Stokes' theorem tells us that the flow of $\nabla \times \mathbf{F}$
  through $S$ is equal to the line integral of $\mathbf{F}$ around
  $\partial S$. We can see from the figure that $\partial S$ consists
  four line segments, so we calculate $\int \mathbf{F} \cdot {\d}\mathbf{r}$
  along each one and sum the results. Integrating from $(0,0,0)$ to
  $(1,0,0)$, we get
  \[
    \int_0^1\mathbf{F}(x,0,0) \cdot \langle 1, 0, 0 \rangle  \, \d x =
    0. 
  \]
  From $(1,0,0)$ to $(1,1,0)$, we get
  \[
    \int_0^1\mathbf{F}(1,y,0) \cdot \langle 0, 1, 0 \rangle  \, \d y =
    e. 
  \]
  From $(1,1,0)$ to $(0,1,0)$ we get
  \[
    \int_0^1\mathbf{F}(x,1,0) \cdot \langle -1, 0, 0 \rangle  \, \d x = 0. 
  \]
  And finally from $(0,1,0)$ back to the origin we get
  \[
    \int_0^1\mathbf{F}(0,y,0) \cdot \langle 0, -1, 0 \rangle  \, \d y = -1. 
  \]
  So altogether the circulation of $\mathbf{F}$ around the boundary of
  $S$ is $\boxed{e-1}$. 
\end{solution}

Example~\ref{exam:stokes} shows that a surface can be deformed without
changing the  flow of $\nabla \times \mathbf{F}$ through it, as
long as it is deformed in such a way that its boundary is preserved: 

\begin{obs}{}{}
  In the context of Stokes' theorem, if $S_1$ and $S_2$ are surfaces
  whose boundaries are the same, then
  \[
    \iint_{S_1} (\nabla \times \mathbf{F}) \cdot \d\mathbf{A} =
  \iint_{S_2}
  (\nabla \times \mathbf{F})
  \cdot \d\mathbf{A}. 
\]
\end{obs}

\begin{exercise}{}{}
  Suppose that $S$ is the surface consisting of the points on the
  sphere $x^2 + y^2 + z^2 = 1$ which are not inside the sphere
  $x^2 + y^2 + (z+1)^2 = 1$. Find $\iint_{S} \nabla \times \mathbf{F}
  \cdot \d\mathbf{A}$, where $\mathbf{F} = \langle yz, x, e^{xyz}
  \rangle$. 
\end{exercise}

\begin{exercise}{}{}
  Suppose $\mathbf{F} = \langle xy, y, xz \rangle$. Find
  $\iint_S \nabla \times \mathbf{F} \cdot \d\mathbf{A}$ where $S$ is the portion
  of the unit sphere $x^2 + y^2 + z^2 = 1$ in the first octant.
\end{exercise}

\chapter*{Colophon}

This text was typeset with pdf\TeX, using \texttt{tcolorbox} and a version of the
\texttt{mathpazo} package's Palatino fonts which was modified to borrow
Greek symbols from Utopia and blackboard bold symbols from Computer
Modern. The cover art is rendered using Ti\textit{k}Z, from code on
this StackExchange thread:

\texttt{http://tex.stackexchange.com/questions/85904/showcase-of-beautiful-title-page-done-in-tex}

The figures are all produced in Asymptote and are included using the
\texttt{asymptote} LaTeX package. All the files necessary to produce
this document are available at github.com/sswatson.

\appendix

\chapter{Appendix} 

\section{Sets and functions}  \label{a:setsandfunctions}

A \textbf{set} is a collection of elements. These elements can be numbers, points, shapes, vectors, other sets, whatever. For example, 
\[
  A = \{1,4,9\} 
\]
is the set consisting of the positive, single-digit perfect squares. The main thing you can do with a set is check whether a particular element is in it. For example, we say that $1 \in A$ (read ``1 is an element of $A$''), while $2 \notin A$ (``2 is not an element of $A$''). 

Some sets with standard and specially typeset names include 
\begin{itemize} 
\item $\R$, the set of real numbers, 
\item $\mathbb{Q}$, the set of rational numbers, 
\item $\mathbb{Z}$, the set of integers, and 
\item $\mathbb{N}$, the set of natural numbers. 
\end{itemize}
  
\begin{tcolorbox}[title = Subsets and set equality, parbox = false, colframe = MidnightBlue, colback=softblue] 
We say that $A \subset B$ (read ``$A$ is a \textbf{subset} of $B$'') if every element of $A$ is an element of $B$. For example, 
\[
\{1,4,9\} \subset \{1,2,3,4,9,10\}. 
\]
We say that two sets $A$ and $B$ are \textbf{equal} if $A\subset B$ and $B\subset A$. Note that 
\[
\{1,1,2\} = \{1,2\} = \{2,1\} . 
\]
since each element of each set is in the others. Thus we can see that
sets ``don't care'' about repeated elements or order. All that matters
is what is in and what is not. It is customary to write sets with
repeats omitted, for clarity.
\end{tcolorbox}


\begin{tcolorbox}[title = Intersections and unions, parbox = false,
  colframe = MidnightBlue, colback=softblue] 
We write $A\cap B$, the \textbf{intersection} of $A$ and $B$, for the set of all the elements that are in both $A$ and $B$. So, for example, 
\[
\{1,4,9\} \cap \{ x \in \R \,: x^2 > 15 \} = \{4,9\}. 
\]
That second set on the left-hand side, which is written in \textit{set-builder} notation, is read as ``the set of all real numbers $x$ such that the square of $x$ is greater than 15''. 

We write $A\cup B$, the \textbf{union} of $A$ and $B$, for the set of all the elements that are in either $A$ or $B$. So, for example, 
\[
\{1,4,9\} \cup \{1,9,25\} = \{1,4,9,25\}. 
\]
\end{tcolorbox}

\begin{tcolorbox}[title = Functions, breakable, parbox = false, colframe = MidnightBlue, colback=softblue] 
If $A$ and $B$ are sets, then a function $f:A \to B$ is a rule that assigns a single element of $B$ to each element of $A$. The set $A$ is called the \textbf{domain} of $f$ and $B$ is called the \textbf{codomain} of $f$. Given a subset $A'$ of  $A$, we define the \textbf{image} $f(A')$ to be 
\begin{equation} \label{eq:image} 
f(A') = \{b \in B \, : \, \text{there exists }a \in A' \text{ so that } f(a) = b\}. 
\end{equation} 
This is the set of all elements of $B$ that get mapped to from some element of $A'$. The \textbf{range} of $f$ is defined to be the set $f(A)$, which contains all the elements of $B$ that get mapped to at least once. 

Similarly, if $B'\subset B$, then the \textbf{preimage} $f^{-1}(B')$ of $B'$ is defined by 
\[
f^{-1}(B') = \{a \in A \, : f(a) \in B'\}. 
\]
This is the subset of $A$ consisting of every element of $A$ that maps to some element of $B'$. 

A function $f$ is \textbf{injective} if no two elements in the domain map to the same element in the codomain; in other words if $f(a) = f(a')$ implies $a=a'$. 

A function $f$ is \textbf{surjective} if the range of $f$ is equal to the codomain of $f$; in other words, if $b \in B$ implies that there exists $a\in A$ with $f(a) = b$. 

A function $f$ is \textbf{bijective} if it is both injective and
surjective. This means that for every $b\in B$, there is exactly one
$a\in A$ such that $f(a) \in b$. If $f$ is bijective, then the
function from $B$ to $A$ that maps $b\in B$ to the element $a \in A$
that satisfies $f(a) = b$ is called the \textbf{inverse} of $f$.

If $f: A \to B$ and $A' \subset A$, then the \textbf{restriction} of
$f$ to $A$ is the function $\left. f \right|_{A'} : A' \to B$ defined
by $\left. f \right|_{A'}(x) = f(x)$ for all $x \in A'$.

If $f: A \to B$ and $g:B \to C$, then the function $g\circ f$ which
maps $x \in A$ to $g(f(x))\in C$ is called the \textbf{composition} of
$g$ and $f$. 

If the rule defining a function is sufficiently simple, we can
describe the function using \textbf{anonymous function notation}. For
example, $x \in \R\mapsto x^2 \in \R$, or $x\mapsto x^2$ for short, is
the squaring function from $\R$ to $\R$. Note that bar on the left
edge of the arrow, which distinguishes the arrow in anonymous function
notation from the arrow between the domain and codomain of a named
function.
\end{tcolorbox}

\begin{tcolorbox}[title = Cartesian product, breakable, parbox = false, colframe = MidnightBlue, colback=softblue] 
The \textbf{Cartesian product} of two sets $A$ and $B$, denoted $A
\times B$, is the set of all pairs $(a,b)$ where $a\in A$ and $b\in
B$. For example, $[0,3] \times [0,2]$ is a rectangle in the plane. We
sometimes use exponents for a Cartesian product of a set with
itself. Thus $[0,1]^2$ is a unit square in $\R^2$, and $[0,1]^3$ is a
unit cube in $\R^3$.
\end{tcolorbox}


\newpage

\section{Trig review}

This appendix provides a streamlined presentation of
trig which is intended to provide enough starting off points to
recover everything else you need. 

\begin{tcolorbox}[title = Trig Review, colback = softblue, breakable, 
  colframe = MidnightBlue] 
  \begin{enumerate}[itemsep = 8pt, parsep = 6pt, leftmargin = 3pt]

  \item \textbf{Cosine and sine}. The basic trig functions are
    $\cos \theta$ and $\sin \theta$.  The most important definition of
    these functions is the following: the cosine of an angle $\theta$
    is equal to the $\mathbf{x}$-\textbf{coordinate} of the point
    obtained by rotating $(1,0)$ an angle of $\theta$ about the
    origin. Sine is the same, but with the $y$-coordinate instead of
    $x$.

    \bang{2mm} This idea bears repeating: {\color{DarkRed} the point
      on the unit circle obtained by rotating $(1,0)$ an angle
      $\theta$ about the origin is equal to
      $(\cos \theta, \sin \theta)$, \textbf{by definition of cosine
        and sine}.}
  
  \item  \textbf{The other ones}. The other four trig functions are simply abbreviations for
  various combinations of sine and cosine: 
\begin{center}  
  \begin{tabular}{cc} 
    $\sin\theta=\sin\theta$ & $\sec\theta=\tfrac{1}{\cos\theta}$ \\
  $\rule{0pt}{20pt}\cos\theta=\cos\theta$ & $\csc\theta=\tfrac{1}{\sin\theta}$ \\
    $\rule{0pt}{20pt}\tan\theta=\tfrac{\sin\theta}{\cos\theta}$ &
                                                                  $\cot\theta=\tfrac{\cos\theta}{\sin\theta}$
    \\
  \end{tabular}
\end{center}

\item \textbf{Special right triangles}. The following two triangles,
  each half of a regular polygon, can be hand for evaluating trig
  functions at special angles.
    \begin{center}
      \begin{asy}[width=0.5\textwidth]
        size(12cm);
        defaultpen(fontsize(8)); 
        settings.outformat="pdf";
        
        real eps=0.1;
        real arceps = 0.2;
        real shift = 1.6;
        
        draw((0,0)--(0,1)--(1,1),dashed);
        draw((0,0)--(1,0)--(1,1)--cycle);
        draw((1-eps,0)--(1-eps,eps)--(1,eps));
        
        draw((shift+1,0)--(shift+2,0)--(shift+1,sqrt(3)),dashed);
        draw((0+shift,0)--(1+shift,0)--(1+shift,sqrt(3))--cycle);
        draw((shift+1-eps,0)--(shift+1-eps,eps)--(shift+1,eps));
        
        label("$\displaystyle{\frac{\sqrt{2}}{2}}$",(0.5,0),S);
        label("$\displaystyle{\frac{\sqrt{2}}{2}}$",(1,0.5),E);
        label("$1$",(0.5,0.5),NW);
        label("$45^\circ$",(arceps,arceps*0.4));
        label("$45^\circ$",(1-arceps/2,1-arceps*1.2));
        
        label("$\displaystyle{\frac{1}{2}}$",(shift+0.5,0),S);
        label("$\displaystyle{\frac{\sqrt{3}}{2}}$",(shift+1,sqrt(3)*0.44),E);
        label("1",(shift+0.5,sqrt(3)/2),NW);
        label("$60^\circ$",(shift+arceps,0.5*arceps));
        label("$30^\circ$",(shift+1-arceps/2,sqrt(3)-2*arceps));
      \end{asy}
    \end{center}
    
  \item  \textbf{Pythagorean identities} The famous identity $\sin^2\theta + \cos^2\theta = 1$ follows from
    the definition of sine and cosine combined with the Pythagorean theorem. Dividing both
    sides of this equation by $\sin^2\theta$ or $\cos^2\theta$, we get 
    \[
      \tan^2 \theta + 1 = \sec^2 \theta \qquad  \text{ and } \qquad 
      1 + \cot^2 \theta = \csc^2 \theta. 
    \]
    
  \item  \textbf{Sum-angle formulas}. The sine sum-angle formula is worth memorizing: for all
    $\alpha$ and $\beta$, 
    \[
      \boxed{\sin(\alpha + \beta) = \sin \alpha \cos \beta + \sin \beta \cos
        \alpha}
    \] 
    The cosine sum-angle formula is worth memorizing too, although it can be derived
    fairly easily from the sine formula by substituting
    $\tfrac{\pi}{2} - \alpha$ for $\alpha$ and $-\beta$ for $\beta$.
    We get 
    \[
      \cos(\alpha + \beta) = \cos \alpha \cos \beta - \sin \alpha \sin
      \beta .
    \]
    
  \end{enumerate}
\end{tcolorbox}
From the above identities, we can derive many others. For example,
   setting $\alpha = \beta$ in the cosine sum-angle formula, we get 
   \[
   \cos 2 \alpha = \cos^2 \alpha - \sin^2 \alpha. 
   \]
   Substituting $\cos^2\alpha = 1 - \sin^2\alpha$, we find that 
   \[
     \cos 2\alpha = 1 - 2\sin^2\alpha. 
   \]
   which can be solved to express $\sin^2\alpha$ in terms of $\cos
   2\alpha$.
 
\newpage

\section{Visualizing functions}

  Graphical visualization is an important conceptual tool for reasoning
  about the behavior of functions. There are a variety of different
  methods for visualizing functions (see the table on the next page
  for pictures):

  \begin{tcolorbox}[title = Function Visualization Methods, colback =
    softblue, colframe = MidnightBlue] 
  \begin{enumerate}[leftmargin = 12pt, itemsep = 6pt, parsep = 6pt]
  \item \textbf{Graphs}. The graph of a function $f : \R^1 \to \R^1$ is
    the set of points of the form $(x,f(x))$, where $x$ is in the
    domain of $f$. The graph of a function $f : \R^2 \to \R^1$ is
    the set of points of the form $(x,y,f(x,y))$, where $(x,y)$ is in the
    domain of $f$. The graph uses one or two dimensions for the input
    and one dimension for the output, so it only works (as a
    visualization tool) for $f : \R^n \to \R^m$ if $m + n \leq 3$.

    The graph involves no loss of information; in principle, you can
    read off anything you want to know about a function from its
    graph. It depicts the domain and the codomain in the same picture. 
    
  \item \textbf{Level sets}. A level set of a function $f:\R^2 \to \R$
    is the solution set of an equation of the form $f(x,y) = c$, where
    $c$ is some constant. For example, the $c=1$ level set of the
    function $x^2 + y^2$ is the unit circle. The level set of a
    function $f:\R^3 \to \R$ is typically a \textit{surface}, For
    example, the level sets of $x^2 + y^2 + z^2$ are spheres.

    We can visualize a function by drawing its level sets. However,
    there are a couple drawbacks: we have to choose a discrete number
    of level sets to draw, and the picture doesn't tell us which $c$
    value corresponds to each level set, unless we draw that
    information in with colors or labels. When we visualize a function
    in this way, we are looking only at the \textit{domain} of the
    function. 

  \item \textbf{Grid lines}. For a function $T$ from $\R^2$ to $\R^2$, we can
    understand $T$ as a transformation which moves points in the plane
    to other points in the plane, and we can visualize this geometric
    action by drawing the images of various grid lines under $T$. This
    visualization is drawn on the codomain side and loses some
    information about which grid lines match to which images. 
    
  \item \textbf{Traces}. We can visualize a function $\mathbf{r}$ from
    $\R^1$ to $\R^2$ or $\R^3$ by highlighting every point in $\R^2$
    or $\R^3$ which is equal to $\mathbf{r}(t)$ for some $t\in
    \R$. This set of points is called the \textit{trace} of
    $\mathbf{r}$.

    The trace is drawn entirely on the codomain side, which means that
    this visualization lacks information about which $t$ value or
    values mapped to each highlighted point.

  \item \textbf{Vector Fields}. For functions from $\R^2$ to $\R^2$ or
    $\R^3$ to $\R^3$, we can visualize them by interpreting the output
    value as a vector and depicting a discrete set of these vectors as small
    arrows drawn in place at the corresponding input values. Doing this
    requires scaling the vectors down so the picture doesn't get
    chaotic. This visualization incorporates inputs and outputs in the
    same picture, and information is lost about the absolute size
    of each vector and about what happens between the discrete set of
    input values shown. 
  \end{enumerate}
\end{tcolorbox}

Table~\ref{table:visualizations} below shows examples of each type of
visualization, with the input (domain) dimension varying by row and
the output (codomain) dimension by column. An example of a common type
of visualization is shown for input-output pair of dimensions.

In a couple cases, the method shown isn't the only one in common use:
a function from $\R^2$ to $\R^2$ can also be drawn as a vector field
(particularly if one is thinking of the outputs as vectors rather than
points in $\R^2$), and the level set method shown for a function from
$\R^3$ to $\R^1$ can also be applied to a function from $\R^2$ to
$\R^1$.

Conventional names are used for each function; note that these vary by
input and output dimension. Functions from $\R^3$ to $\R^2$ don't have
a dedicated visualization method, although one could visualize each
component of such a function separately, or identify the codomain
$\R^2$ with the $xy$-plane in $\R^3$ to make a vector field
representation.

%----------------------------

\newsavebox{\asyboxone}
\begin{lrbox}{\asyboxone}
  \begin{asy}
    defaultpen(fontsize(5));
    size(4cm);
    import graph;
    real f(real x){ return x*x;}
    draw(graph(f,-2,2),MidnightBlue,Arrows());
    draw((-2,0)--(2,0),Arrows());
    draw((0,0)--(0,4),Arrow());
    label("$x$",(2,0),align=E,MidnightBlue);
    label("$f(x)=x^2$",(0,4),align=N,MidnightBlue);  
    label("graph",(0,-0.2));
  \end{asy}
\end{lrbox}
\begin{lrbox}{\asyboxtwo}
  \begin{asy}
    defaultpen(fontsize(5)); 
    size(4cm);
    import graph;
    import x11colors;
    pair f(real t){ return (t*(t-1)*(t+1),t^2-1);}
    draw(graph(f,-1.6,1.6),MidnightBlue,Arrows());
    draw((-2,0)--(2,0),Arrows());
    draw((0,-2)--(0,2),Arrow());
    label("$\mathbf{r}(t) = (t^3 - t, t^2 - 1)$",(0,2.2),align=N,MidnightBlue);  
    label("trace",(0,-2.2)); 
  \end{asy}
\end{lrbox}
\begin{lrbox}{\asyboxthree}
  \begin{asy}
    size(4cm);
    defaultpen(fontsize(5));
    import graph3;
    triple f(real t){return (t*cos(20*t), t*sin(20*t), t);}
    draw(O--2*X,Arrow3());
    draw(O--2*Y,Arrow3());
    draw(O--1.6*Z,Arrow3());
    draw(graph(f,-1.2,1.2,100,Spline),MidnightBlue,Arrows3());
    label("$\mathbf{r}(t) = (t\cos(20t),t\sin(20t), t)$",(0,0,1.8),align=N,MidnightBlue);  
    label("trace",(0,0,-2.0));
  \end{asy}
\end{lrbox}

%----------------------------

\newsavebox{\asyboxfour}
\begin{lrbox}{\asyboxfour}
  \begin{asy}
    defaultpen(fontsize(5));
    size(4cm);
    import graph3;
    real f(pair z) {return exp(-z.x^2-z.y^2);}
    draw(surface(f,(0,0),(1.7,1.7),nx=32,Spline),
    LightSeaGreen+opacity(0.8),MidnightBlue);
    draw(O--2.2*X,Arrow3());
    draw(O--2.2*Y,Arrow3());
    draw(O--1.3*Z,Arrow3());
    label("$x$",(2.2*X),align=W);
    label("$y$",(2.2*Y),align=E);
    label("$f(x,y) = e^{-x^2-y^2}$",(1.3*Z),align=N);
    label("graph",-2Z); 
  \end{asy}
\end{lrbox}
\newsavebox{\asyboxfive}
\begin{lrbox}{\asyboxfive}
  \begin{asy}
    size(4cm,0);
    defaultpen(fontsize(5));
    import graph;
    real f(real x){
      return x*x;
    }
    int n = 4;
    draw((-n-1,0)--(n+1,0),Arrows(4));
    draw((0,-n-1)--(0,n+1),Arrows(4));
    picture sidepic;
    size(sidepic,5cm);
    add(sidepic,currentpicture);
    for(int i=-n;i<=n; ++i){
      draw((i,-n)--(i,n),MidnightBlue+opacity(0.5));
      draw((-n,i)--(n,i),MidnightBlue+opacity(0.5));
    }
    pair f(real s, real t){
      return (s+t+0.01*t^3+1,s-t);
    }
    typedef pair fun(real);
    fun F(real t) {
      return new pair(real s) {return f(s,t);};
    }
    fun G(real s) {
      return new pair(real t) {return f(s,t);};
    }
    for(int i=-n;i<=n;++i){
      draw(sidepic,graph(F(i),-n,n),MidnightBlue+opacity(0.5));
      draw(sidepic,graph(G(i),-n,n),MidnightBlue+opacity(0.5));
    }
    add(shift((15,0))*sidepic);
    label("$T(x,y) = (x+y+\frac{x^3}{100}+1,x-y)$",(-1,8));
    draw((3,5)..(6,5.5)..(9,5),Arrow(4));
    label("grid lines",(8,-12)); 
  \end{asy}
\end{lrbox}
\newsavebox{\asyboxsix}
\begin{lrbox}{\asyboxsix}
  \begin{asy}
    defaultpen(fontsize(5));
    size(4cm);
    import graph3;
    triple f(pair z) {return (z.y,sin(7*z.x),z.x);}
    draw(surface(f,(0,0),(1.7,1.7),Spline),
    LightSeaGreen+opacity(0.8),MidnightBlue);
    draw(O--2.2*X,Arrow3());
    draw(O--2.2*Y,Arrow3());
    draw(O--1.6*Z,Arrow3());
    label("$x$",(2.2*X),align=W);
    label("$y$",(2.2*Y),align=E);
    label("$\mathbf{r}(s,t) = (t,\sin(7s),s)$",(1.8*Z+Y),align=N);
    label("trace",-1.6Z); 
  \end{asy}
\end{lrbox}

%----------------------------

%----------------------------


\newsavebox{\asyboxseven}
\begin{lrbox}{\asyboxseven}
  \begin{asy}
    defaultpen(fontsize(5)); 
    import smoothcontour3;
    import palette; 
    size(4cm); 
    
    real f(real x, real y, real z, real k) {
      return x^2 + y^2 - z^2 + k ;
    }
    
    pen[] colors = Gradient(5 ... new pen[] {LightSeaGreen, MidnightBlue});
    int ctr = 0;
    
    for(real k=-1; k <= 1; k += 0.5) {
      draw(implicitsurface(
      new real (real x, real y, real z) {return f(x,y,z,k);},
      (-3,-3,-2), (3,3,2)
      ),
      surfacepen = material(opacity=0.5,
      diffusepen=colors[ctr]+opacity(0.2),
      emissivepen=0.2*white+opacity(0.2)));
      ctr += 1; 
    }
    label("$f(x,y,z) = x^2 + y^2 - z^2$",(0,0,2.7));
    label("level surfaces",(0,0,-4.5));
  \end{asy} 
\end{lrbox}

\newsavebox{\asyboxeight}
\begin{lrbox}{\asyboxeight}
  \begin{asy} 
    import graph3;
    defaultpen(fontsize(5)); 
    import multitools; 
    size(4cm);
    currentprojection=perspective((45,135,30));
    
    path3 gradient1(triple z){
      return O--(z.y+z.z,z.x+z.z,z.x+z.y);
    }
    
    triple A=(0,0,0);
    triple B=(5,5,5);
    
    add(VectorPlot3D(gradient1,A,B,5,5,5,p=LightSeaGreen,lambda=2));

    label("$\mathbf{F}(x,y,z) = \langle y+z,x+z,x+y \rangle$",5.5*Z+2*X);

    label("vector field", -1.5*Z+2*X); 

    draw(O--5*X,Arrow3(3));
    draw(O--5*Y,Arrow3(3));
    draw(O--5*Z,Arrow3(3));
  \end{asy}
\end{lrbox}

%----------------------------

  \begin{table}[h!]
    \centering
    \begin{tabular}{M{5mm}M{5mm}|M{4.2cm}|M{4.2cm}|M{4.2cm}|N}
      \multicolumn{5}{c}{\hspace{2cm} Output Dimension\vphantom{$\dfrac{1}{2}$}}  \\ 
      \multirow{3}{1cm}{\rotatebox[origin=c]{90}{ \parbox[c]{14cm}{\centering Input
      Dimension} }}
      & & 1 & 2 & 3 &\\ \cline{2-5}
      & 1 & \usebox{\asyboxone} & \usebox{\asyboxtwo} & \usebox{\asyboxthree} &\\ \cline{2-5}
      & 2 & \usebox{\asyboxfour} & \usebox{\asyboxfive} & \usebox{\asyboxsix} &\\ \cline{2-5}
      & 3 & \usebox{\asyboxseven} & (none) & \usebox{\asyboxeight}& \\ \cline{2-5}
    \end{tabular}
    \caption{Different methods of visualizing functions from $\R^n$
      to $\R^m$, arranged by $(n,m)$ pairs \label{table:visualizations}}
  \end{table} 

  \newpage

  \section{Summation notation}

  Some shorthand is essential for writing sums with many
  terms. Perhaps the most common approach is to use ellipses:
  \[
    1 + 2  + 3 + \cdots + 99 + 100 = 5050. 
  \]
  However, this approach is not ideal because the reader is left to
  infer the pattern.

  When more precision is required, we would like to specify a formula
  for the $k$th term as well as a starting and ending value. For
  example, the sum
  \[
    1 + 4 + 9 + 16 + \cdots + 100
  \]
  can be written as ``the sum of $k^2$ as $k$ ranges from 1 to
  10''. The math notation that has been adopted for abbreviating this
  English phrase is the following:
  \[
    \sum_{k=1}^{10} k^2 
  \]
  The variable $k$ is called a \textit{dummy variable}, since it is
  only there as a way to specify a formula for generating the
  terms. We could change each $k$ to a different symbol without
  changing the essential meaning, which is ``sum the first 10 positive
  perfect squares''.

  \begin{exercise}{}{}
    Find $\displaystyle{\sum_{k=1}^5\frac{1}{k(k+1)}}$. 
  \end{exercise}

  
  \begin{exercise}{}{}
    Find $\displaystyle{\sum_{k=1}^5\sum_{j=1}^k j }$. 
  \end{exercise}
  
  \newpage

  \section{Polar, cylindrical, and spherical coordinate reference}
  \label{sec:polarref}

  \newsavebox{\xconstantfig} 
  \begin{lrbox}{\xconstantfig}
    \begin{asy}
      import graph3;

      pen surfacepen = LightSeaGreen+opacity(0.4);
      pen wirepen = MidnightBlue;
      
      draw(O--1.5*X,Arrow3()); draw(O--1.5*Y,Arrow3()); draw(O--1.5*Z,Arrow3());
      size(4cm);
      
      int n = 3;  
      for(int i = -n; i <= n; ++i){
        real x = i/n; 
        draw(surface(new triple(pair p){return (x,p.x,p.y);},
        (-1,-1),
        (1,1),
        20,
        Spline),
        material(opacity=0.4,diffusepen=surfacepen,emissivepen=0.6*black+0.4*surfacepen),  
        wirepen); 
      }

      label("$x=\mathrm{constant}$", -2.1*Z+0.2*X);
    \end{asy}
  \end{lrbox}

  \newsavebox{\yconstantfig} 
   \begin{lrbox}{\yconstantfig}
    \begin{asy}
      import graph3;

      pen surfacepen = LightSeaGreen+opacity(0.4);
      pen wirepen = MidnightBlue;
      
      draw(O--1.5*X,Arrow3()); draw(O--1.5*Y,Arrow3()); draw(O--1.5*Z,Arrow3());
      size(4cm);
      
      int n = 3;
      for(int i = -n; i <= n; ++i){
        real y = i/n; 
        draw(surface(new triple(pair p){return (p.x,y,p.y);},
        (-1,-1),
        (1,1),
        20,
        Spline),
        surfacepen,
        wirepen); 
      }

      label("$y=\mathrm{constant}$", -2.3*Z);
    \end{asy}
  \end{lrbox}

  \newsavebox{\zconstantfig} 
  \begin{lrbox}{\zconstantfig}
    \begin{asy}
      import graph3;

      pen surfacepen = LightSeaGreen+opacity(0.4);
      pen wirepen = MidnightBlue;
      
      draw(O--1.5*X,Arrow3()); draw(O--1.5*Y,Arrow3()); draw(O--1.5*Z,Arrow3());
      size(4cm);
      
      int n = 3;
      for(int i = -n; i <= n; ++i){
        real z = i/n; 
        draw(surface(new triple(pair p){return (p.x,p.y,z);},
        (-1,-1),
        (1,1),
        20,
        Spline),
        surfacepen,
        wirepen); 
      }

      label("$z=\mathrm{constant}$", -2.3*Z);
    \end{asy}
  \end{lrbox}
  
  \newsavebox{\rconstantfig} 
  \begin{lrbox}{\rconstantfig}
    \begin{asy}
      import graph3;

      pen surfacepen = LightSeaGreen+opacity(0.4);
      pen wirepen = MidnightBlue;

      triple xyz_cylindrical(real r, real theta, real z){
        return (r*cos(theta), r*sin(theta), z); 
      }
      
      draw(O--1.5*X,Arrow3()); draw(O--1.5*Y,Arrow3()); draw(O--1.5*Z,Arrow3());
      size(4cm);
      
      int n = 3;  
      for(int i = 1; i <= n; ++i){
        real r= i/n; 
        draw(surface(new triple(pair p){return xyz_cylindrical(r,p.x,p.y);},
        (0,-1),
        (2*pi,1),
        20,
        Spline),
        surfacepen,
        wirepen); 
      }

      label("$r=\mathrm{constant}$", -1.8*Z);
    \end{asy}
  \end{lrbox}

  \newsavebox{\thetaconstantfig} 
  \begin{lrbox}{\thetaconstantfig}
    \begin{asy}
      import graph3;

      pen surfacepen = LightSeaGreen+opacity(0.4);
      pen wirepen = MidnightBlue;

      triple xyz_cylindrical(real r, real theta, real z){
        return (r*cos(theta), r*sin(theta), z); 
      }
      
      draw(O--1.5*X,Arrow3()); draw(O--1.5*Y,Arrow3()); draw(O--1.5*Z,Arrow3());
      size(4cm);
      
      int n = 5;  
      for(int i = 1; i <= n; ++i){
        real theta = 2*pi*i/n; 
        draw(surface(new triple(pair p){return xyz_cylindrical(p.x,theta,p.y);},
        (0,-1),
        (1,1),
        20,
        Spline),
        surfacepen,
        wirepen); 
      }

      label("$\theta=\mathrm{constant}$", -1.7*Z);
    \end{asy}
  \end{lrbox}

  \newsavebox{\rhoconstantfig} 
  \begin{lrbox}{\rhoconstantfig}
    \begin{asy}
      import graph3;

      pen surfacepen = LightSeaGreen+opacity(0.4);
      pen wirepen = MidnightBlue;

      triple xyz(real rho, real theta, real phi){
        return rho*(cos(theta)*sin(phi), sin(theta)*sin(phi), cos(phi)); 
      }
      
      draw(O--1.5*X,Arrow3()); draw(O--1.5*Y,Arrow3()); draw(O--1.5*Z,Arrow3());
      size(4cm);
      
      int n = 3;  
      for(int i = 1; i <= n; ++i){
        real rho = i/n; 
        draw(surface(new triple(pair p){return xyz(rho,p.x,p.y);},
        (0,0),
        (2*pi,pi),
        20,
        Spline),
        surfacepen,
        wirepen); 
      }

      label("$\rho=\mathrm{constant}$", -1.5*Z);
    \end{asy}
  \end{lrbox}

  \newsavebox{\phiconstantfig} 
  \begin{lrbox}{\phiconstantfig}
    \begin{asy}
      import graph3;

      pen surfacepen = LightSeaGreen+opacity(0.4);
      pen wirepen = MidnightBlue;

      triple xyz(real rho, real theta, real phi){
        return rho*(cos(theta)*sin(phi), sin(theta)*sin(phi), cos(phi)); 
      }
      
      draw(O--1.5*X,Arrow3()); draw(O--1.5*Y,Arrow3()); draw(O--1.5*Z,Arrow3());
      size(4cm);
      
      int n = 6;  
      for(int i = 1; i < n; ++i){
        real phi = pi*i/n; 
        draw(surface(new triple(pair p){return xyz(p.x,p.y,phi);},
        (0,0),
        (1, 2*pi),
        20,
        Spline),
        surfacepen, wirepen); 
      }
      
      label("$\phi= \mathrm{constant}$", -1.5*Z);
    \end{asy}
  \end{lrbox}
  
  \begin{tcbraster}[title = Coordinate conversion,
    raster every box/.style={valign=center,halign=center}, 
    raster columns = 3, raster height = 11cm, colback =
    softblue, colframe = MidnightBlue]
    \begin{tcolorbox}[blankest, valign=center]
      \begin{tcbraster}[raster columns = 1, raster height=5.5cm]
        \begin{tcolorbox}[title=Polar to Cartesian, colback =
    softblue, colframe = MidnightBlue, valign=center]
    $
      \left\{
        \begin{array}{r@{\:}c@{\:}l}
          x &=& r \cos \theta \\
          y &=& r \sin \theta 
        \end{array}
      \right.
    $
    \end{tcolorbox}
    \begin{tcolorbox}[title = Cylindrical  to Cartesian, colback =
    softblue, colframe = MidnightBlue, valign=center]
      $
        \left\{
          \begin{array}{r@{\:}c@{\:}l}
            x &=& r \cos \theta \\
            y &=& r \sin \theta \\
            z &=& z
          \end{array}
        \right.
      $
    \end{tcolorbox}
    
  \end{tcbraster}
\end{tcolorbox}
  \begin{tcolorbox}[title = Spherical  to Cartesian]
      $
        \left\{
          \begin{array}{r@{\:}c@{\:}l}
            x &=& \rho \cos \theta \sin \phi \\
            y &=& \rho \sin \theta \sin \phi\\
            z &=& \rho \cos \phi
          \end{array}
        \right.
      $
    \end{tcolorbox}
    \begin{tcolorbox}[title=Area/Volume differentials, colback =
    softseagreen, colframe = SeaGreen, valign=center]
      $
        \left\{
          \begin{array}{r@{\:}c@{\:}l}
            \d A &=& r \, {\d}r \, \d\theta \\
            \d V &=& r \, {\d}r \, \d\theta \, {\d}z \\
            \d V &=& \rho^2 \sin \phi \, {\d}\rho \, {\d}\phi\, \d\theta
          \end{array}
        \right.
      $
    \end{tcolorbox}
  \end{tcbraster}
  
  \begin{table}[h!]
    \centering
    \begin{tabular}{M{4.2cm}|M{4.2cm}|M{4.2cm}}
      \usebox{\xconstantfig} & \usebox{\yconstantfig} & \usebox{\zconstantfig}  \\ \hline
      \usebox{\rconstantfig} & \usebox{\thetaconstantfig}  & \usebox{\zconstantfig}  \\ \hline
      \usebox{\rhoconstantfig} & \usebox{\thetaconstantfig}  & \usebox{\phiconstantfig}  \\ \hline
    \end{tabular}
    \caption{Level surfaces for each coordinate
      in the rectangular, cylindrical, and spherical
      systems. \label{table:coordinateslices}}
  \end{table}

  \newpage 

  \section{SageMath} \label{sec:sagemath} 

  Math is more fun when you learn how to take advantage of
  computational resources to assist your learning. Some problem
  solving tasks are done much better by computers than people, and
  while in some cases it is important to gain facility with performing
  such calculations by hand, at some point you want to delegate
  tedious tasks to the computer and spend your time and attention on
  the more creative aspects of problem solving.

  The open source project which has arguably made the most concerted
  and successful effort to be broadly useful to math students with
  minimal fuss is SageMath (or just Sage). You can use Sage in your
  browser without having to install anything (at
  sagecell.sagemath.org, or at cocalc.com for more extensive
  resources). You can use it essentially as a calculator (i.e., with
  minimal programming) for many things. However, if you do want to do
  something a little more involved, this is quite convenient, since
  Sage uses famously beginner-friendly Python as its language. This is
  where Sage really shines in comparison to a proprietary tool like
  Wolfram|Alpha, which is great for one-liners but not so much if you
  want to do a several-step computation.

  Here are some examples of calculations you can do with Sage. You
  have to tell it that you're going to use \texttt{x} as a symbol
  using the \texttt{var} function; you can declare several symbols in
  this way using spaces. The text following the hashtag is a comment
  and is ignored by Sage. 

\begin{lstlisting}
> var('x y') # tell sage that you want to use x and y as symbols
> integrate(sqrt(x^2+1),x) + integrate(e^y * sin(y), y)
-1/2*(cos(y) - sin(y))*e^y + 1/2*sqrt(x^2 + 1)*x + 1/2*arcsinh(x)

> limit(sin(x^2)/x^2,x=0) # find a limit
1

> cos(3*x).trig_expand() # work with trig functions
cos(x)^3 - 3*cos(x)*sin(x)^2

> diff(x^x,x) # differentiate the function x^x
x^x*(log(x) + 1)

> find_local_maximum(sin(x) + cos(x), 0, 2*pi) # find the maximum of sin+cos over [0,2pi]
(1.414213562373095, 0.78539814681742492)

> plot(sin(x) + cos(x), 0, 2*pi) # plot a function

> [factor(x^n - 1) for n in [1..5]] # factor the first five polynomials of the form x^n - 1
[x - 1,
 (x + 1)*(x - 1),
 (x^2 + x + 1)*(x - 1),
 (x^2 + 1)*(x + 1)*(x - 1),
 (x^4 + x^3 + x^2 + x + 1)*(x - 1)]
\end{lstlisting}

Throughout the text, some computations are performed using Sage*. They
are indicated with the CoCalc icon
\href{http://cocalc.com}{\cocalc} which can be clicked to
open a page at cocalc.com showing the result as well as the
code used to generate it. \sidenote{* Actually, some of these linked
code snippets are in \href{http://julialang.org}{Julia}, which is a
newer language more suited to numerical work}[-5mm]

If you want to learn more about Sage, I recommend the (freely
available) book
\href{http://www.gregorybard.com/Sage.html}{\textit{Sage for
    Undergraduates}} by Gregory Bard.

\newpage

\section{Wait, is the gradient normal or
  tangent?} \label{sec:gradienta}

\begin{wrapfigure}[28]{R}[1cm]{7cm}
  \begin{asy}
    import graph3;
    size(7cm);
    real h = 3;
    currentprojection = perspective(10,3,4); 
    draw(O--1.2*h*X,Arrow3());
    draw(O--1.2*h*Y,Arrow3());
    draw(O--0.8*h*Z,Arrow3());
    triple f(pair p) {
      real th = p.x;
      real ph = p.y; 
      return sqrt(3)*(cos(th)*sin(ph), sin(th)*sin(ph), cos(ph)); 
    }
    draw(surface(f, (0,0), (2*pi, pi),20,Spline), LightSeaGreen+opacity(0.5), MidnightBlue);
    label("$g(x,y,z) = x^2 + y^2 + z^2 = 3$",(2.3,1.7,-0.8),MidnightBlue); 
    real eps = 1.2;  
    draw(Label("$\nabla g(1,1,1)$",Relative(1),align=NE),(1,1,1)--(1,1,1)+eps*(1,1,1),MidnightBlue, Arrow3());
    real t = 0.08;
    triple sideways = sqrt(3/2)*t*eps*(0,-1,1); 
    draw((1,1,1)+sideways--(1,1,1)+sideways+t*eps*(1,1,1)--(1,1,1)+t*eps*(1,1,1)); 
    dot((1,1,1),linewidth(2.0));
    real eps = 0.05; 

    for(int i=1;i<=3;++i){
      draw((i,0,0)--(i,0,-eps));
      draw((0,i,0)--(0,i,-eps));
      if (i < 2) draw((0,0,i)--(0,-eps,i)); 
    }
  \end{asy}
  \caption{The gradient of a function at a point is orthogonal to the level
    set of that function through that point} \label{fig:gradg} 

  \begin{asy}
    import graph3;
    size(7cm);
    real h = 3;
    currentprojection = perspective(10,3,4); 
    draw(O--1.2*h*X,Arrow3());
    draw(O--1.2*h*Y,Arrow3());
    draw(O--0.8*h*Z,Arrow3());
    triple f(pair p) {
      real th = p.x;
      real ph = p.y; 
      return sqrt(3)*(cos(th)*sin(ph), sin(th)*sin(ph), cos(ph)); 
    }
    draw(surface(f, (0,0), (2*pi, pi/2),20,Spline), LightSeaGreen+opacity(0.5), MidnightBlue);
    label("$f(x,y) = \sqrt{3 - x^2 - y^2}$",(2.3,1.7,-0.2),MidnightBlue); 
    real eps = 1.2;  
    draw(Label("$\left.\bigg \langle\partial_x f, \partial_y f, |\nabla f|^2 \bigg\rangle \right|_{(1,1)}$",Relative(0.5),align=ENE),(1,1,1)--(1,1,1)+(-1,-1,2),MidnightBlue, Arrow3());
    dot((1,1,1),linewidth(2.0));
    draw(Label("$\nabla f$",align=1.2*SSW),(1,1,0)--(0,0,0),Arrow3());
    draw((1,1,1)--(1,1,0),dashed);
    draw((0,0,3)--(0,0,0),dashed); 
    real eps = 0.05;
    for(int i=1;i<=3;++i){
      draw((i,0,0)--(i,0,-eps));
      draw((0,i,0)--(0,i,-eps));
      if (i < 2) draw((0,0,i)--(0,-eps,i)); 
    }
    draw(circle((0,0,0),sqrt(3),normal=Z),MidnightBlue);
  \end{asy}
  \caption{The gradient of a function specifies the direction of
    maximum increase, which corresponds to moving along its graph in
    the steepest direction} \label{fig:gradf}
\end{wrapfigure}

You might have a couple different images come to
mind when you think about gradients, one with a vector which is
tangent to a surface, and the other with a vector normal to a
surface. So is the gradient normal or tangent? Both*! To specify the
picture, we have to specify \textit{what function} we're
taking the gradient of, and \textit{what surface} our vector is tangent or
normal to. 
\sidenote{* Sort of: the tangent vector is only closely related to the gradient}[-5mm]

The most important idea is that the gradient of a function at a point
specifies its direction of maximum increase. This has two
implications:

(1) (Figure~\ref{fig:gradg}) If we look at the gradient of a function
$g$ from $\R^3$ to $\R^1$, then the gradient of $g$ is orthogonal to
the \textbf{level surface} of $g$ at each point $(x_0,y_0,z_0)$.

(2) (Figure~\ref{fig:gradf}) If we look at the gradient of a function
$f$ from $\R^2$ to $\R^1$, then the gradient of $f$ at $(x_0,y_0)$
tells us the direction \textit{in the $x$-$y$ plane} in which we
should move so that the point $(x,y,f(x,y))$ on the \textbf{graph} of
$f$ moves in the direction of maximum $z$ increase.* \sidenote{* If the
  graph of $f$ corresponds to a hilly landscape, the gradient at each
 point only specifies the cardinal direction of maximum increase. The
 hiker moving in this direction naturally moves tangent to the hill,
 because gravity keeps them pinned to graph}

In the second case, if we want a vector which is tangent to the graph
of $f$ we need to include the third component (since the gradient only
has two). This component should be chosen so that the vector is
contained within the plane tangent to the graph at that point, which
means we should substitute into the tangent plane equation
\[
  z - z_0 = (\partial_x f)(x_0,y_0)(x-x_0) +  (\partial_y
  f)(x_0,y_0)(y-y_0). 
\]
Substituting $x-x_0 = (\partial_x f)(x_0,y_0)$ and $y-y_0 =
(\partial_y f)(x_0,y_0)$, we find that the third component should be
the squared gradient of $f$ at $(x_0,y_0)$. 



\newpage 

\section{The central idea of integral calculus} \label{sec:centralidea}

Mass is an \textbf{additive} property of an object. This means that if
we subdivide the object into smaller pieces and sum their masses, we
get the same result as if we had measured the mass of the original
object. Many other quantities of interest in calculus are additive,
such as the length of a path, the area of a surface, the volume of a
3D region, the flow of a vector field across a surface, the integral
of a function over an interval or region, the moment of inertia,
torque, etc.

The central idea of integral calculus is that an additive property of
an object can be calculated by \textbf{subdividing the object into
  many tiny pieces}, approximating the quantity of interest for each
piece, summing the results, and letting the number of pieces go to
infinity to obtain an expression which is amenable to formal
integration techniques. There are two pretty amazing things that
happen here: (1) we end up with an \textit{exact} answer even though
our approach is based on approximations, and (2) performing the
integration ends up being far more tractable than working with the
approximating sums.

The force behind (2) is the fundamental theorem of calculus, which
provides a link between limits of approximating sums and the
often-quite-easy task of finding an antiderivative. But (1) typically
remains mysterious until one takes a course or reads a book in
which the hand-waving justifications of an introductory calculus course
are replaced with rigorous proofs. Nevertheless, the basic idea behind
(1) can be appreciated an at elementary level. It is helpful to begin
by considering what can go wrong. \sidenote{We will focus on integrating
  positive functions---the story requires modifications to accommodate
functions which take on negative or zero values}

\begin{exercise}{A cautionary tale}{cautionarytale}
  \begin{minipage}[t]{0.64\textwidth} \parskip = 0.2 in 
  For each positive integer $n$, consider the greatest function $f_n$
  which is piecewise constant over each interval
  $\left[\frac{i}{n}, \frac{i+1}{n}\right)$, where $i$ ranges from 0
  to $n-1$, and which is less than or equal to the function $f(x) = x^2+1$ at
  each point $x\in [0,1]$.

  Find the arc length of the graph of $f_n$ over $[0,1]$ (either (i)
  count only the horizontal steps or (ii) count the horizontal and
  vertical steps). Does this approximate the arc length of $f$ over
  $[0,1]$ increasingly well as $n\to\infty$?
\end{minipage}
\begin{minipage}[t]{0.35\textwidth}
  \begin{lrbox}{\asybox}
  \begin{asy}
    defaultpen(fontsize(10));
    size(5.3cm);
    import graph;
    real f(real x){ return 0.1 + 1/5*x*x;}
    picture gr;
    draw(gr,graph(f,0,2),MidnightBlue);
    draw(gr,(0,0)--(2,0));
    draw(gr,(0,0)--(0,1),Arrow(2));
    real eps = 0.05;
    draw(gr,Label("1",Relative(1),align=S),(2,0)--(2,-eps));
    
    label(gr,"$x$",(2,0),align=E,MidnightBlue);
    label(gr,"$f(x)$",(0,1.05),align=N,MidnightBlue);
    
    int n = 8;
    add(gr);
    for(int i=0;i< 2*n; ++i){
      draw((i/n,f(i/n))--((i+1)/n,f(i/n)),MidnightBlue);
      draw(((i+1)/n,f(i/n))--((i+1)/n,f((i+1)/n)),MidnightBlue+linetype(new real[] \
      {1,2})+linewidth(0.25));
    }
  \end{asy}
\end{lrbox} \raisebox{\dimexpr -\height+1.5ex
  \relax}{\usebox{\asybox}}
\end{minipage}
\end{exercise}

\begin{wrapfigure}[12]{R}[1cm]{5cm}
  \begin{asy}
    defaultpen(fontsize(10));
    size(5.3cm);
    import graph;
    real f(real x){ return 0.1 + 1/5*x*x;}
    picture gr;
    int i = 10, n = 8;
    fill(gr,graph(f,i/n,(i+1)/n)--((i+1)/n,f(i/n))--cycle,0.6*white+0.6*DarkRed);
    filldraw(gr,box((i/n,0), ((i+1)/n,f(i/n))),SeaGreen);
    draw(gr,graph(f,0,2),MidnightBlue);
    draw(gr,(0,0)--(2,0));
    draw(gr,(0,0)--(0,1),Arrow(2));
    real eps = 0.05;
    draw(gr,Label("1",Relative(1),align=S),(2,0)--(2,-eps));
    
    label(gr,"$x$",(2,0),align=E,MidnightBlue);
    label(gr,"$f(x)$",(0,1.05),align=N,MidnightBlue);
    
    add(gr);
    for(int i=0;i< 2*n; ++i){
      draw((i/n,f(i/n))--((i+1)/n,f(i/n)),MidnightBlue);
      draw(((i+1)/n,f(i/n))--((i+1)/n,f((i+1)/n)),MidnightBlue+linetype(new real[] {1,2})+linewidth(0.25));
    }
  \end{asy}
  \caption{The area under the step function approximates the area
    under the curve\label{fig:relerror}}
\end{wrapfigure}

Exercise~\ref{exer:cautionarytale} shows that some properties of $f_n$
can approximate those of $f$ quite poorly even if $f_n$ approximates
$f$ very well. What goes wrong here with this approximation is that
the \textbf{relative error} is large. In other words, the ratio of the
arc length of $f$ over a short interval and the arc length of the
approximating ``stairstep'' over that interval does not converge to 1
as the width of the interval is decreased to zero.

Contrast Exercise~\ref{exer:cautionarytale} with
Figure~\ref{fig:relerror}, which illustrates approximating the area
under the graph of $f$ with the area under $f_n$. In this case the
relative error of the approximation for each slice decreases to
zero. In other words, quotient of the green area and the red+green
area converges to 1 as the number of slices converges to infinity. By
Exercise~\ref{exer:relerror} below, this implies that the relative
error of the approximating sum converges to zero as the number of
steps goes to infinity. Therefore, the limit of the sequence of
approximating sums is exactly equal to the area under the curve.

\begin{exercise}{Small relative error for the part implies small
    relative error for the whole}{relerror}
  Let $\epsilon > 0$. Suppose that $A_i$ and $A_i'$, where $i$ ranges
  from $1$ to $n$, are positive numbers satisfying
  $1-\epsilon \leq \frac{A_{i}'}{A_{i}} \leq 1 + \epsilon$ for all $i$
  from $1$ to $n$. Show that
  $1-\epsilon \leq \frac{A_1'+\cdots+A_n'}{A_1+\cdots+A_n} \leq 1 +
  \epsilon$.  
\end{exercise}

\begin{exercise}{}{intrelerror}
  Verify that if $f:[a,b] \to (0,\infty)$ is a continuous function and
  $x_0 \in [a,b)$, then the quotient of the area under the graph of $f$ over
  $[x_0,x_0+h]$ and the area under the graph of the constant function
  $f(x_0)$ over $[a,b)$ converges to 1 as $h \to 0^+$. 
\end{exercise}

The above discussion does omit some important details. For example,
Exercise~\ref{exer:relerror} stipulates that we achieve same small
relative error $\epsilon$ across the whole interval, while in
Exercise~\ref{exer:intrelerror} we focus on each point $x_0$ one at a
time. Also, what if $f$ isn't assumed to be positive, and the area
under the curve over a particular short interval happens to be zero?

Despite these shortcomings, the relative error perspective can clarify
what's going on when we do our ``split it into many tiny
pieces'' derivations throughout the course. 

\newpage 

\section{Additional exercises} \label{sec:addexer}

\asubsection{sec:ndimspace}

\begin{aexercise}
  Find a formula for the distance from the point $(x,y,z)$ to the
  $z$-axis. 
\end{aexercise}

\begin{aexercise}
  Find a formula for the distance from the point $(x,y,z)$ to the
  plane $x = -1$. 
\end{aexercise}

\begin{aexercise}
  Find an equation whose solution set is the set of points whose distance to $(-3,2,5)$
  is equal to 4. What is the radius of the intersection of this sphere with the $yz$-plane?
\end{aexercise}

\begin{aexercise}
  A rope of length $12\pi$ units is partially wrapped around a tree of radius 12 units, as shown in the figure below. The part of the rope not touching the tree is pulled tight. Find the coordinates of the end of the rope, labeled $E$. 
  \begin{center}
    \includegraphics{exercisefigures/ropetree}
  \end{center}
\end{aexercise}

\newpage 

\asubsection{sec:RntoRn}

\begin{aexercise}
Find linear transformations with the following geometric
descriptions. 

(a) reflect across the $x$-axis

(b) rotate 180 degrees and double the distance from the origin

(c) halve the distance from the origin while preserving the angle
between $(x,y)$, the origin, and the positive $x$-axis

(d) rotate 90 degrees counterclockwise

(e) project a point in $\mathbb{R}^3$ onto the $xy$-plane
\end{aexercise}

\asubsection{sec:det}

\begin{aexercise}
  A $2\times 2$ matrix is said to be \textit{symmetric} if its top
  right and lower left entries are equal. The \textit{diagonal}
  entries are the top left and lower right entries. Show that the
  diagonal entries of a $2\times 2$
  symmetric matrix with positive determinant must have the same sign. 
\end{aexercise}

\begin{aexercise}
  Find all values of $\lambda$ for which 
\[
\det \left[
\begin{array}{ccc}
3 - \lambda & 0 & 0 \\
0 & 2-\lambda & 1 \\
1 & 1 & 1 - \lambda
\end{array}
\right]
= 0. 
\]
\end{aexercise}

\newpage 

\asubsection{sec:vectors}

\begin{aexercise}
Consider the vectors $\mathbf{u}$, $\mathbf{v}$ and $\mathbf{w}$ in the figure below. 

\begin{center}
  \includegraphics{exercisefigures/parallelogram}
\end{center}

(i) Find scalars $a$ and $b$ such that $\mathbf{w} = a\mathbf{u} + b\mathbf{v}$. You may assume that both $a$ and $b$ are integer multiples of $0.5$. Express your answer as an ordered pair $(a,b)$, written in the box. 

(ii) Find scalars $a$ and $b$ such that $\mathbf{w} - \mathbf{u} + \mathbf{v} = a\mathbf{u} + b\mathbf{v}$. You may assume that both $a$ and $b$ are integer multiples of $0.5$. Express your answer as an ordered pair $(a,b)$, written in the box. 
\end{aexercise}

\begin{aexercise}
  \begin{minipage}[t]{0.58\textwidth}
    The point $M$ is the midpoint of $P$ and $Q$. Express $\mathbf{c}$
    in terms of $\mathbf{a}$ and $\mathbf{b}$, and express $\mathbf{d}$
    in terms of $\mathbf{a}$ and $\mathbf{b}$.
  \end{minipage}
  \begin{minipage}[t]{0.4\textwidth}
    \raisebox{\dimexpr -\height + 1.5ex\relax}{\includegraphics[width=5cm]{exercisefigures/midpoint}}
  \end{minipage}
\end{aexercise}

\begin{aexercise}
  Suppose that all three coordinate planes are outfitted with a mirror
  surface, and that laser beam is directed from a point in the first
  octant toward the origin. Suppose that the beam is slightly
  misaimed, so that it strikes one of the mirror planes first. Then
  the beam strikes a second plane, and then a third, before being
  reflected back out into the first octant.

  Show that the reflected beam returns along  path which is parallel
  to its incoming path. The physics idea you need here is that when  a
  light beam hits a reflective plane, its reflection angle is the same
  as its incident angle, and the plane containing the incoming and
  outgoing beams meets the reflective plane at a 90-degree angle. 
\end{aexercise}

\asubsection{sec:dot}

\begin{aexercise}
  Use dot products to find a vector which is orthogonal to both
  $\langle 1, -2, 4 \rangle$ and $\langle -3, 1, 1 \rangle$. Hint:
  represent your vector as $\langle a, b, c \rangle$ and solve for
  $a$, $b$, and $c$. 
\end{aexercise}

\begin{aexercise}
Suppose that a force $\mathbf{F}=(1,-2)$ is acting on an object moving
parallel to the vector $(4,1)$. Decompose $\mathbf{F}$ into a sum of
vectors $\mathbf{F}_1$ and $\mathbf{F}_2$, where $\mathbf{F}_1$ points
along the direction of motion and $\mathbf{F}_2$ is perpendicular to
the direction of motion.
\end{aexercise}

\begin{aexercise}%{$\star$ (1975 USAMO)}{}
  Let $A$, $B$, $C$, and $D$ be four points in $\mathbb{R}^3$. Use vectors to show that 
\[
AB^2 +BC^2 + CD^2 + DA^2 \geq AC^2 + BD^2. 
\]
(This generalizes the fact that the sum of the squares of the sides of
a quadrilateral is at least the sum of the squares of its diagonals.)
Make a statement about when equality holds.
\end{aexercise}

\asubsection{sec:cross}

\begin{aexercise}
  Consider a triangle $T$ with vertices $A=(1,0)$, $B=(0,1)$, and $C$ on the line $y=-x$. Use a cross product to find the area of $T$, and show that it does not depend on the location of $C$. 
\end{aexercise}

\begin{aexercise}
  Is the cross-product associative? Is the dot product associative?
  Prove or give a counterexample for each.
\end{aexercise}

\begin{aexercise}
  Suppose that the four vectors $\vec{a}$, $\vec{b}$, $\vec{c}$, and
  $\vec{d}$ in $\R^3$ are coplanar. Show that \\
  $(\vec{a}\times\vec{b})\times(\vec{c}\times\vec{d})=\vec{0}$.
\end{aexercise}

\begin{aexercise}
  The volume of a parallelepiped is the product of the area of its
  base and its height. Consider the parallelepiped spanned by
  $\mathbf{a}$, $\mathbf{b}$, and $\mathbf{c}$. You may suppose for
  simplicity that $\mathbf{b}$, $\mathbf{c}$, and $\mathbf{a}$ form a
  right-handed triple of vectors, which means that a right-handed
  screw rotated an angle less than 180$^\circ$ from $\mathbf{b}$ to
  $\mathbf{c}$ advances in the direction of $\mathbf{a}$.

  (a) Let us think of the parallelogram spanned by $\mathbf{b}$ and
  $\mathbf{c}$ as the base of the parallelepiped. What is the (signed)
  area of this parallelogram?

  (b) What is the height of the parallelogram, in terms of
  $\mathbf{a}$ and the unit vector pointing in the direction of
  $\mathbf{b} \times\mathbf{c}$?

  (c) Put together parts (a) and (b) to derive the triple scalar
  product formula for the volume of a parallelepiped.
\end{aexercise}

\begin{aexercise}
  Suppose that $\ell_1(t)=t\mathbf{a}+\mathbf{b}_1$ and $\ell_2(t)=t\mathbf{a}+\mathbf{b}_2$ are parallel lines in $\mathbb{R}^2$ or $\R^3$. Show that the distance $D$ between them is given by 
\[
D = \frac{|\mathbf{a}\times(\mathbf{b}_2-\mathbf{b_1})|}{|\mathbf{a}|}. 
\]
\end{aexercise}

\begin{aexercise}
  Find the point $P$ on the line $(3-t,2+2t,-4t)$ which is closest to the point $Q=(3,7,1)$.
\end{aexercise}

\begin{aexercise}[enhanced, breakable]%{$\star$}{}
  It is possible to prove the vector triple product formula
  \[
    \mathbf{a}\times(\mathbf{b}\times\mathbf{c}) =
    (\mathbf{a}\cdot\mathbf{c})\mathbf{b} -
    (\mathbf{a}\cdot\mathbf{b})\mathbf{c}
  \]
  in a tedious way using coordinates. This problem outlines a more
  conceptual proof, taken from a note written by William C.\ Schulz.

  (a) Use the paralellepiped interpretation of the triple scalar
  product to show that
  \[
    \mathbf{b}\cdot(\mathbf{c}\times\mathbf{n}) =
    \mathbf{c}\cdot(\mathbf{n}\times\mathbf{b}) =
    \mathbf{n}\cdot(\mathbf{b}\times\mathbf{c})
  \]
  (b) Use the right-hand rule to observe that if $\mathbf{c}$ is
  perpendicular to $\mathbf{n}$, then
  \[
    \mathbf{n}\times(\mathbf{c}\times\mathbf{n}) =
    |\mathbf{n}|^2\mathbf{c}.
  \]
  (c) Show that it suffices to consider the case where $\mathbf{a}$,
  $\mathbf{b}$, and $\mathbf{c}$ form a basis for $\mathbb{R}^3$.

  (d) Write $\mathbf{a}\times(\mathbf{b}\times\mathbf{c})$ as a linear
  combination of $\mathbf{a}, \mathbf{b},$ and $\mathbf{c},$ so that
  \begin{equation} \label{comp}
    \mathbf{a}\times(\mathbf{b}\times\mathbf{c}) = \kappa
    \mathbf{a}+\lambda \mathbf{b}+\mu\mathbf{c},
  \end{equation} 
  (e) The easiest coefficient to determine is $\kappa$. What is it?

  (f) To find $\lambda$, dot both sides of \eqref{comp} with
  $\mathbf{c}\times \mathbf{n},$ where
  $\mathbf{n}\colonequals \mathbf{b}\times\mathbf{c}$.

  (g) To find $\mu$, dot both sides of \eqref{comp} with
  $\mathbf{b}\times \mathbf{n},$ where
  $\mathbf{n}\colonequals \mathbf{b}\times\mathbf{c}$.
\end{aexercise}

\asubsection{sec:lines_and_planes}

\begin{aexercise}
  Find an equation for the plane containing the points $(3, -1, 2),
  (2, 0, 5)$, and $(1, -2, 4)$.
\end{aexercise}

\begin{aexercise}
  Find an equation for the plane that contains the lines
\[
\left\{
\begin{array}{c@{\,}c@{\,}c}
  x(t)&=&5+t \\
  y(t)&=&1-t \\
  z(t)&=&4-3t. 
\end{array}
\right.
\]
and
\[
\left\{
\begin{array}{c@{\,}c@{\,}c}
  x(t)&=&5-4t \\
  y(t)&=&1+t \\
  z(t)&=&4-3t. 
\end{array}
\right.
\]
\end{aexercise}

\begin{aexercise}
  Let $O=(0,0,0)$ be the origin in $\R^3$. If the vector
  $\overrightarrow{OP}$ has length 3, what is the greatest possible
  distance from $P$ to the line
  $\ell = \{-1-t,1+2t,t) \,:\, t\in \R$\}?
\end{aexercise}

\asubsection{sec:motion_in_space}

\begin{aexercise}
A chickadee starts at the point $(2,-4,1)$ and flies in the direction of the vector $\left(\frac{3}{\sqrt{10}},0,\frac{1}{\sqrt{10}}\right)$ at a rate of $\sqrt{10}$ units per second. A hummingbird starts at the point $(8,20,7)$ and flies in the direction of the vector $\left(\frac{1}{\sqrt{5}},-\frac{2}{\sqrt{5}},0\right)$ at a rate of $3\sqrt{5}$ units per second.

(a) Do the paths of the chickadee and the hummingbird intersect? 

(b) Do the hummingbird and the chickadee collide?
\end{aexercise}

\begin{aexercise}
  Suppose that a particle is revolving clockwise around the point
  $(4,2)$ at a rate of 3 revolutions per second. Write parametric
  equations describing the location of the particle at time $t$,
  assuming that it starts at the point $(6,2)$.
\end{aexercise}

\begin{aexercise}
  Sketch the image of the path $\vec{x}(t)=(\cos t, e^t)$.
\end{aexercise}

\begin{aexercise}
  Derive an integral formula for the arc length of a differentiable
  path $\mathbf{r}(t)$ where $t$ ranges from $a$ to $b$.

  Hint: use the Pythagorean theorem to find the distance from
  $\mathbf{r}(t)$ to $\mathbf{r}(t+\Delta t)$. Divide the curve into
  many such segments and approximate its length as the sum of the
  length of those segments. Then let $\Delta t \to 0$. 
\end{aexercise}

\begin{aexercise}
  Calculate the total length of the curve represented by the
  parametric equation $\mathbf{r}(t) = \langle a\cos^3t, a\sin^3 t \rangle$,
  where $a$ is a positive constant.
\end{aexercise}

\begin{aexercise}
All the figures below were made by graphing $\{(x(t),y(t))\,:\,0\leq t \leq 2\pi\}$ for simple expressions $(x(t),y(t))$. For each graph, find a curve whose image looks (at least roughly) like the figure shown.

\begin{center}
  \includegraphics{exercisefigures/parametric}   \includegraphics{exercisefigures/parametric2}   \includegraphics{exercisefigures/parametric3} 
\end{center}
\end{aexercise}

\asubsection{sec:quadric_surfaces}

\begin{aexercise} Prove that the projection onto the
$xy$-plane of the intersection of the plane $x+y+z = 1$ and the ellipsoid
$x^2 + 4y^2 + 4z^2 = 4$ is an ellipse.
\end{aexercise}

\begin{aexercise}
Find an equation for the set of points whose distance to the plane $z
= -1$ and to the point $(0,0,1)$ are equal. What kind of shape is this surface?
\end{aexercise}

\begin{aexercise}
Show that if the point $(a,b,c)$ lies on the hyperbolic
paraboloid $z = y^2 - x^2$, then the line with parametric equation
$(a+t,b+t,c + 2(b-a)t)$ lies on the hyperbolic paraboloid. (Thus, even
though the paraboloid is curvy, it contains lots of lines!)
\end{aexercise}

\asubsection{sec:coordinates}

\begin{aexercise}
  Write inequalities describing the unit sphere in cylindrical
  coordinates.
\end{aexercise}

\begin{aexercise}
  Consider a sphere of radius 10 centered at the origin. Suppose that
  the portion of the sphere above the plane $z=8$ is
  removed. Furthermore, a sphere of radius $2$ is removed from the
  center of the solid. Write inequalities in spherical coordinates to
  describe the resulting shape.
\end{aexercise}

\asubsection{sec:limits}

\begin{aexercise}
  Evaluate 
  $\displaystyle{\lim_{(x,y)\to(0,0)}\frac{2xy^2+xy}{x^2+y^2}}$, or explain why
  the limit fails to exist.
\end{aexercise}

\begin{aexercise}
  Define $f(x) = (x^2 - 4)/(x-2)$ when $x\neq 2$ and $f(2) =
  c_1$. Determine the value of the constant $c_1$ for which $f$ is
  continuous. Do the same for
  \[
    g(x,y) = \left\{
      \begin{array}{cl}
        \frac{3|x|^3+3|y|^3-x^{10}\arctan(y)}{|x|^3+|y|^3} & \text{ if}(x,y)\neq (0,0) \\
        c_2 & \text{ if}(x,y)=(0,0).
      \end{array} 
    \right.
  \]
\end{aexercise}

\begin{aexercise}
  Suppose that $f,g,$ and $h$ are real-valued functions on $\R^n$, and
  suppose that $f(\mathbf{x}) = g(\mathbf{x})h(\mathbf{x})$. Prove that if $g$
  is bounded and $\lim_{\mathbf{x}\to \mathbf{0}}h(\mathbf{x}) = 0$, then
  $\lim_{\mathbf{x}\to \mathbf{0}}f(\mathbf{x}) = 0$.
\end{aexercise}

\asubsection{sec:partial}

\begin{aexercise}
  Find $\pd{f}{x}$, $\pd{f}{y}$, and $\pd{f}{z}$ for
  $f(x,y,z) = \frac{1}{xy} + \log(xyz) + e^x\sin(y z) $.
\end{aexercise}

\begin{aexercise}
  Find the partial derivatives of
  \[
    F(a,b) = \int_a^b \sqrt{t^3+1}\,dt
  \]
  with respect to $a$ and with respect to $b$.
\end{aexercise}


\begin{aexercise}
  Suppose $f:\R^2 \to \R$. Is it possible for $\pd{f}{x}$ and
  $\pd{f}{y}$ to exist at $(0,0)$ while $f$ is not differentiable at
  $(0,0)$?  Prove that it isn't possible, or provide an example to
  show that it is possible.
\end{aexercise}

\asubsection{sec:linapprox}

\begin{aexercise}
  Verify the linear approximation
  \[
    \frac{2x+3}{4y+1} \approx 3 + 2x - 12y
  \]
  at $(x,y) = (0,0)$.
\end{aexercise}

\begin{aexercise}
  Find the linear approximation of the function
  $f(x,y) = 1-xy\cos(\pi y)$ at $(1,1)$ and use it to approximate
  $f(1.02,0.97)$.
\end{aexercise}



\asubsection{sec:optim}

\begin{aexercise}
  For functions of one variable, it is impossible for a continuous
  function to have two local maxima and no local minimum: between
  every two peaks there must be a trough. Show, however, that the
  function
  \[
    f(x,y) = -(x^2-1)^2-(x^2y-x-1)^2
  \]
  has only two critical points and has a local maximum at each (!!). 
\end{aexercise}


\begin{aexercise}
  The Hessian of a twice-differentiable function $f:\R^2 \to \R$ is
  defined by
  \[
    H(x,y) = \left[
      \begin{array}{cc}
        \partial_x^2 f & \partial_x\partial_y f \\
        \partial_x\partial_y f & \partial_y^2 f
      \end{array}
    \right]. 
  \]
  The \textit{second derivative test} says that if $\det H(x,y)  < 0$ at
  a critical point $(x,y)$ then $f$ has neither local minimum nor a
  local maximum at $(x,y)$, and if $\det H(x,y)  > 0$ then $f$ has a
  local minimum or a local maximum at $(x,y)$. The minimum and maximum
  cases are distinguished by whether the diagonal entries of $H$ are
  positive or negative, respectively.

  Verify that the second derivative test gives the correct results for
  the functions $x^2+y^2$, $-x^2-y^2$, and $x^2 - y^2$ at the origin. 
\end{aexercise}

\asubsection{sec:dd_and_grad}

\begin{aexercise}
  Find the equation for the plane tangent to $z=x^2-6x+y^3$ at
  $(x,y) = (4,3)$.
\end{aexercise}

\begin{aexercise}
  Suppose that the temperature in a room $[0,5]^3$ is given as a
  function of position by $T(x,y,z) = 50 + x^2 + (y-3)^2 + 2z$. You
  are a bug starting at position $(3,2,2)$, and you are cold. You
  decide to move in the direction of greatest temperature increase at
  all times.

  (a) What vector describes the direction in which you initially move?

  (b) Do you first hit the ceiling, the floor, or a wall?
\end{aexercise}

\begin{aexercise}
  Show that the sum of the $x$-, $y$- and $z$-intercepts of any
  tangent plane to the surface
  $\sqrt{x} + \sqrt{y} + \sqrt{z} = \sqrt{c}$ is a constant.
\end{aexercise}

\begin{aexercise}
  The second directional derivative of $f(x,y)$ in the direction
  $\mathbf{u}$ is defined to be
  \[
    D_\mathbf{u}^2 f(x,y) = D_\mathbf{u}[D_{\mathbf{u}} f(x,y)].
  \]
  Find $D_\mathbf{u}^2 f(x,y)$ if $f(x,y) = x^3 + 5x^2y + y^3$ and
  $\mathbf{u} = \frac{1}{5}\langle 3,4\rangle$.
\end{aexercise}



\asubsection{sec:chainrule}

\begin{aexercise}
  Let $f(x,y)=e^{3x+y}$, and suppose that $x=s^2+t^2$ and $y=2+t$. Find $\partial f/\partial s$ and $\partial f/\partial t$ by substitution and by means of the chain rule. Verify that the results are the same for the two methods. 
\end{aexercise}


\begin{aexercise}
  Use the chain rule to find $\partial z/\partial s$, $\partial
  z/\partial t$, and $\partial z/\partial u$ when $s=4, t = 2, u=1$, given 
  \[
    z = x^4 + x^2y, \quad x = s + 2t - u, \quad y = stu^2.
  \]
\end{aexercise}

\begin{aexercise}
  A conical ice sculpture melts in such a way that its height decreases at a rate of 0.001 meters per second and its radius decreases at a rate of 0.002 meters per second. At what rate is the volume of the sculpture decreasing when its height reaches 3 meters, assuming that its radius is 2 meters at that time? Express your answer in terms of $\pi$ and  in units of cubic meters per second. 
\end{aexercise}

\asubsection{sec:lagrange}

\begin{aexercise}
  Some of the level curves of a function $f: \R^2 \to \R$ are
  shown. Find the minimum and maximum values of $f(x,y)$ subject to
  the constraint that $(x,y)$ lies on the red curve.
  \begin{center}
    \includegraphics{exercisefigures/lagrange_exercise} 
  \end{center}
\end{aexercise}

\begin{aexercise}
  Minimize the function $f(x,y) = (x-y)^2$ subject to the constraint
  $xy = 1$ without using Lagrange multipliers. Verify that the method
  of Lagrange multipliers gives the same result.
\end{aexercise}

\begin{aexercise}
  Verify the hypotheses of the extreme value theorem and find the
  absolute maximum and minimum values of $f(x,y) = x^2+y^2-2x$ on the
  closed triangle $D$ with vertices $(2,0)$, $(0,2)$, and $(4,0)$.
\end{aexercise}

\begin{aexercise}
  Use Lagrange multipliers to find the maximum and minimum values of
  $f(x,y) = x^2 + y^2$ subject to $xy = 1$.
\end{aexercise}

\begin{aexercise}
  Use Lagrange multipliers to find the maximum and minimum values of
  $x^2 + y^2 + z^2$ subject to the constraint $x + y + z =
  12$. Confirm your answer by solving the same problem using vector
  methods.
\end{aexercise}

\begin{aexercise}
  Use Lagrange multipliers to show that the rectangle with maximum
  area that has a given perimeter $p$ is a square.
\end{aexercise}

\asubsection{sec:double}

\begin{aexercise}
  Evaluate
  $\displaystyle{\int_{0}^{1}\!\int_{0}^{y^2} x^2 y \,\d x\,\d y}$ and
  sketch the region of integration in $\R^2$ indicated by the limits
  of integration.
\end{aexercise}

\begin{aexercise}
  Evaluate $\displaystyle{\int_{0}^\pi \!\int_{y}^{\pi}\frac{\sin x}{x} \,\d x\,\d y}$.
\end{aexercise}

\begin{aexercise}%{$\star$}{}
  Evaluate $\displaystyle{\int_{-\infty}^{\infty} e^{-x^2}\,\d x}$. (Hints: write down
  the product of this integral with itself, the second time using $y$
  as the dummy variable. Rewrite this as an area integral and switch
  to polar coordinates.)
\end{aexercise}

\begin{aexercise}%{($\star$)}{}
  Define a function $f(x,y)$ on $[0,1] \times [0,2]$ by
  \[
    f(x,y) = \left\{ \begin{array}{cl}
                       1 & \text{ if }x\text{ is rational} \\
                       0 & \text{ if }x\text{ is irrational and }y \leq 1 \\
                       2 & \text{ if }x\text{ is rational and }y > 1.
                     \end{array} 
                   \right.
  \]
  Show that the iterated Riemann integral
  $\int_0^1\int_0^2 f(x,y)\,\d y\,\d x$ exists, and find
  its value. Show that the iterated Riemann integral
  $\int_0^2\int_0^1 f(x,y)\,\d x\,\d y$ does not exist.
\end{aexercise}

\begin{aexercise}
  For $(x,y) \neq (0,0)$, we define
  \[
    f(x,y) = \frac{xy(x^2-y^2)}{(x^2+y^2)^3}.
  \]
  Calculate the iterated integrals of $f$ over $[0,2]\times[0,1]$.
\end{aexercise}

\asubsection{sec:triple}

\begin{aexercise}
  Explain why the integral of the function
  $f(x,y,z) = \frac{1}{x+y+z+1}$ over the cube
  $[0,1] \times [0,1] \times [0,1]$ is equal to
  $\lim_{n\to\infty} S_n$, where
  \[
    S_n = \sum_{i=1}^n \sum_{j=1}^n \sum_{k=1}^n
    \frac{f(i/n,j/n,k/n)}{n^3}.
  \]
  (b) At \texttt{sagecell.sagemath.org}, use the code
\begin{verbatim} 
n = 20
var("x","y","z","i","j","k")
f(x,y,z) = 1/(x+y+z+1)
assume(0<x<1);assume(0<y<1);assume(0<z<1)
I = integrate(integrate(integrate(f(x,y,z),x,0,1),y,0,1),z,0,1)
S = sum(sum(sum(f(i/n,j/n,k/n)/n^3,k,1,n),j,1,n),i,1,n)
(I,S,N(I),N(S))
\end{verbatim}
  which gives the exact values of the integral and $S_{20}$ followed
  by their decimal representations, to show that $S_{20}$ is close to
  the value of the integral. Increase $n$ (just change the first line
  of code to assign a different value to $n$) by multiples of 5 to
  \textbf{find the least value of $n$} such that $n$ is a multiple of
  5 and $S_n$ differs from $I$ by less than 0.01.
\end{aexercise}

\begin{aexercise}
  Find the region $E$ in $\R^3$ for which
  \[
    \iiint_E (1-x^2-2y^2-3z^2) \, {\d}V
  \]
  is as large as possible. (Hint: this problem is not as difficult as
  it may seem.)
\end{aexercise}


\begin{aexercise}
  Sketch the solid whose volume is given by the integral $\displaystyle{\int_0^1
  \int_0^{1-x}\int_0^{2-2z}\,\d y\,dz\,\d x}$.
\end{aexercise}

\begin{aexercise}
  Integrate $f(x,y,z) = 1-z^2$ over the tetrahedron $W$ with vertices
  at the origin, $(1,0,0)$, $(0,2,0)$, and $(0,0,3)$.
\end{aexercise}

\asubsection{sec:polar_int}

\begin{aexercise}
  Find the volume of the solid that is bounded by the paraboloid
  $z = 9 - x^2 - y^2$, the $xy$-plane, and the cylinder
  $x^2 + y^2 = 4$.
\end{aexercise}

\begin{aexercise}
  Express as a triple integral the volume of the wedge in the first
  octant that is cut from the cylinder $y^2+z^2 = 1$ by the planes
  $y=x$ and $x=1$.
\end{aexercise}

\begin{aexercise}
  Sketch the solid whose volume is given by the integral
  $\int_{-\pi/2}^{\pi/2} \int_0^2 \int_0^{r^2} \, r\,\d z\,\d r\,\d\theta$,
  and evaluate the integral.
\end{aexercise}

\begin{aexercise}
  Evaluate
  $\int_0^1 \int_0^{\sqrt{1-x^2}}
  \int_{\sqrt{x^2+y^2}}^{\sqrt{2-x^2-y^2}} \, xy \, {\d}z \, \d y \, \d x$ by
  changing to spherical coordinates.
\end{aexercise}

\begin{aexercise}
  Find the mass of a ball $B$ given by $x^2+y^2+z^2 \leq a^2$ if the
  density at any point is proportional to its distance from the
  $z$-axis. (Hint: use cylindrical coordinates.)
\end{aexercise}


\begin{aexercise}
  Find the volume of the region $W$ that represents the intersection
  of the solid cylinder $x^2+y^2 \leq 1$ and the solid ellipsoid
  $2(x^2+y^2)+z^2\leq 10.$
\end{aexercise}

\begin{aexercise}
Use polar coordinates to combine the sum 
\[
\int_{1/\sqrt{2}}^1\int_{\sqrt{1-x^2}}^x xy \, \d y \, \d x + 
\int_{1}^{\sqrt{2}} \int_0^x xy \,\d y \,\d x + 
\int_{\sqrt{2}}^2 \int_0^{\sqrt{4-x^2}} xy \,\d y \, \d x
\]
into one double integral. Then evaluate the double integral.
\end{aexercise}

\begin{aexercise}
  Find the average distance from a point in a ball of radius $a$ to
  the center. Recall that the average value, denoted $\overline{f}$,
  of a function $f$ over a region $R$ is given by
  \[
    \overline{f} = \frac{\iiint_R f \, {\d}V}{\iiint_R 1 \, {\d}V}.
  \]
\end{aexercise}


\asubsection{sec:changeofvariables}

\begin{aexercise}
  Let $D$ be a parallelogram with vertices $(0,0)$, $(1,0)$, $(1,1)$,
  and $(2,1)$. Calculate $\iint_D 1\,\d A$ in two ways:

  (a) Find $\iint_D 1\,\d A$ without using calculus.

  (b) Find $\iint_D 1\,\d A$ using the change of variables $u = 2x - 2y$
  and $v=2y$.
\end{aexercise}

\begin{aexercise}
Find
$\iint_R \cos \left(\frac{y-x}{y+x}\right) \, {\d}A$, where $R$ is the
trapezoidal region with vertices $(1,0)$, $(2,0)$, $(0,2)$, and
$(0,1)$.
\end{aexercise}

\begin{aexercise}
  Evaluate the integral $\iint_R e^{(x+y)/(x-y)} \, {\d}A$, where $R$ is
  the trapezoidal region with vertices $(1,0), (2,0)$, $(0,-2)$, and
  $(0,-1)$.
\end{aexercise}

\begin{aexercise}
  Find $\iint_R y^2 \, {\d}A$, where $R$ is the region bounded by the
  curves $xy = 1$, $xy = 2$, $xy^2 = 1$, and $xy^2 = 2$.
\end{aexercise}

\begin{aexercise}
  Find $\iint_R e^{x+y} \, {\d}A$ where $R$ is given by the inequality
  $|x| + |y| \leq 1$.
\end{aexercise}

\begin{aexercise}
  Evaluate the integral
  $\displaystyle{\int_{0}^{2}\!\int_{x/2}^{x/2+1} x^5 (2y -
    x)e^{(2y-x)^2} \,\d y\,\d x}$ by making the substitution $u = x$ and
  $v=2y-x$.
\end{aexercise}

\asubsection{sec:vector_fields}

\begin{aexercise}
  Sketch the vector field $\mathbf{F} =\langle xy, x\rangle$ over
  $[-1,1]^2$. 
\end{aexercise}

\begin{aexercise}
  A velocity field $\mathbf{v}$ showing air flow around an airfoil is
  illustrated below. Sketch a continuous path $C$ along which
  $\int_C \mathbf{v} \cdot \d\mathbf{r}$ is as large as possible. 
  \begin{center}
    \includegraphics[width=5cm]{exercisefigures/airfoil}
  \end{center}
\textit{\tiny Figure credit: Crowsnest at English Wikipedia, CC BY-SA 3.0, \url{https://commons.wikimedia.org/w/index.php?curid=6509566}}
\end{aexercise}

\begin{aexercise}
  Denote by $-C$ the reversal of a path $C$. Show that 
  \[
  \int_{-C} \mathbf{F} \cdot \d\mathbf{r} = -\int_{C}\mathbf{F}
  \cdot \d\mathbf{r}. 
  \]
\end{aexercise}

\asubsection{sec:line_integrals}

\begin{aexercise}
  A thin wire is bent into the shape of a semicircle $x^2 + y^2 = 4$,
  $x \geq 0$. If the linear density of the wire at $(x,y)$ is given by
  $x+y+2$, find the mass of the wire. 
\end{aexercise}

\begin{aexercise}
  Let $\vec{F}:\R^3 \to \R^3$ be the vector field given by
  $\vec{F}(x,y,z)=ay^2\vec{i}+2y(x+z)\vec{j}+(by^2 +z^2)\vec{k}$.

  (a) For which values of $a$ and $b$ is the vector field $\vec{F}$
  conservative?

  (b) Find a function $f:\R^3 \to \R$ such that $\vec{F} = \nabla f$
  for these values.

  (c) Find an equation describing a surface $S$ with the property that
  for every smooth oriented curve $C$ lying on $S$,
  \[
    \int_C \vec{F}\cdot d\vec{r} = 0,
  \]
  for these values.
\end{aexercise}

\begin{aexercise}
  The vector field $\mathbf{F}$ plotted
below is not conservative. Pick two points $a$ and $b$ and sketch two
paths from $a$ to $b$ along which the line integrals of $\mathbf{F}$
are clearly different, and explain why the integrals are clearly
different.

\begin{center}
  \includegraphics[width=8cm]{exercisefigures/vecfield} 
\end{center}
\end{aexercise}

\begin{aexercise}
  According to Coulomb's law, the force between a particle of charge
  $q_1$ at the origin and a particle of charge $q_2$ at the point
  $\vec{r} = (x,y,z)\in \R^3$ is given by
  \[
    \mathbf{F} = \frac{q_1q_2}{4\pi
      \varepsilon_0}\frac{\vec{r}}{|\vec{r}|^3},
  \]
  where $\varepsilon_0$ is a physical constant.

  (a) Is $\vec{F}$ a conservative vector field? If so, find a function
  $\phi:\R^3 \to \R$ such that $\nabla \phi = \vec{F}$.

  (b) If the distance between two charges is tripled, by what factor
  is the force between them reduced?

  (c) How much work is required to move the second particle along the
  path
  \[
    \vec{\gamma}(t)=(1+(1-t)\cos(t^2),\sqrt{\sin{\pi t}},4t - t^2)
    \qquad 0\leq t \leq 1?
  \]
  Express your answer in terms of $q_1$, $q_2$, and $\varepsilon_0$.
\end{aexercise}

\begin{aexercise}
  Let $C$ be a level set of the function $f(x,y)$. Show that
  $\int_C \nabla f\cdot d\vec{s} = 0$.
\end{aexercise}



\asubsection{sec:greens}

\begin{aexercise}
  Find the area of the rectangle $D=[0,a]\times[0,b]$ using Green's
  theorem.
\end{aexercise}

\begin{aexercise}
Use Green's theorem to prove the \textit{shoelace
  formula}, which says that the area of a polygon with vertices
$(x_1,y_1),\ldots,(x_n,y_n)$ listed in counterclockwise order has area
given by one-half times
$x_1y_2 + x_2 y_3 + \cdots + x_{n-1}y_n + x_n y_1 - (x_2 y_1 + x_3 y_2
+ \cdots + x_{n}y_{n-1}+ x_1 y_n)$.
\end{aexercise}

\begin{aexercise}[enhanced,breakable]
Consider the polygon $P$ whose vertices are listed below, in counterclockwise order. Describe a simple algorithm for approximating the area of $P$. 
$$ \tiny
\begin{array}{cc}
  x & y \\ \hline 
 1. & 0. \\
 0.990436 & 0.000510364 \\
 0.962169 & 0.00404386 \\
 0.916455 & 0.0134308 \\
 0.855307 & 0.031127 \\
 0.781385 & 0.059054 \\
 0.697852 & 0.0984698 \\
 0.608195 & 0.149879 \\
 0.516034 & 0.212985 \\
 0.42493 & 0.286693 \\
 0.338182 & 0.369151 \\
 0.258659 & 0.457849 \\
 0.188645 & 0.549739 \\
 0.129723 & 0.641405 \\
 0.0827009 & 0.729244 \\
 0.0475787 & 0.809659 \\
 0.0235696 & 0.879262 \\
 0.00916064 & 0.935053 \\
 0.00221855 & 0.974593 \\
 0.000130899 & 0.996135 \\
 -0.0000248959 & 0.998721 \\
 -0.00129456 & 0.982236 \\
 -0.00665282 & 0.947413 \\
 -0.0188134 & 0.895794 \\
 -0.0400542 & 0.829645 \\
 -0.0720676 & 0.751827 \\
 -0.115845 & 0.665643 \\
 -0.171601 & 0.574651 \\
 -0.238745 & 0.482473 \\
 -0.315899 & 0.392595 \\
 -0.400957 & 0.30818 \\
 -0.491197 & 0.231899 \\
 -0.583415 & 0.165791 \\
 -0.674106 & 0.111162 \\
 -0.759645 & 0.0685255 \\
 -0.836489 & 0.0375914 \\
 -0.901371 & 0.0172975 \\
 -0.951479 & 0.00589015 \\
 -0.984611 & 0.00104315 
\end{array}
\begin{array}{cc}
 -0.999301 & 0.0000100651 \\
 -0.994893 & -0.000198912 \\
 -0.971584 & -0.00262612 \\
 -0.930411 & -0.0101725 \\
 -0.873191 & -0.0254117 \\
 -0.802423 & -0.0504237 \\
  -0.721151 & -0.0866563 \\
 -0.632792 & -0.134824 \\
 -0.540951 & -0.194848 \\
 -0.449221 & -0.265846 \\
 -0.360994 & -0.346165 \\
 -0.279269 & -0.433459 \\
 -0.206502 & -0.524811 \\
 -0.144474 & -0.616889 \\
 -0.0942056 & -0.706119 \\
 -0.0559169 & -0.788884 \\
 -0.0290287 & -0.86172 \\
 -0.0122151 & -0.921505 \\
 -0.00349795 & -0.965635 \\
 -0.000378337 & -0.992164 \\
 0.000000440875 & -0.999913 \\
 0.000669899 & -0.988538 \\
 0.00464303 & -0.958545 \\
 0.0147221 & -0.911264 \\
 0.0333185 & -0.848777 \\
 0.0622956 & -0.773799 \\
 0.102843 & -0.689529 \\
 0.15539 & -0.599477 \\
 0.219565 & -0.507267 \\
 0.294196 & -0.41644 \\
 0.377366 & -0.330264 \\
 0.466505 & -0.251558 \\
 0.558525 & -0.182543 \\
 0.649984 & -0.124732 \\
 0.73727 & -0.0788538 \\
 0.816797 & -0.0448348 \\
 0.885207 & -0.0218148 \\
 0.939547 & -0.00821674 \\
 0.977439 & -0.00185515 \\
 0.997206 & -0.0000804643
\end{array}
$$
\end{aexercise}

\begin{aexercise}[enhanced,breakable]
  A planimeter is a device used to calculate the area of a
  two-dimensional region. In this problem, we explore the mathematics
  behind how the planimeter works. (Thanks to Wikipedia for the
  figures and the description below).

  \begin{center}
    \begin{minipage}{5cm}
      \includegraphics[width=5cm]{exercisefigures/planfig}
    \end{minipage}
    \begin{minipage}{5cm}
      \includegraphics[width=5cm]{exercisefigures/planimeter.jpg}
    \end{minipage}
  \end{center}

  The pointer $M$ at one end of the planimeter follows the contour $C$
  of the surface $S$ to be measured. For the linear planimeter the
  movement of the ``elbow'' $E$ is restricted to the
  $y$-axis. Connected to the arm $ME$ is the measuring wheel with its
  axis of rotation parallel to $ME$. A movement of the arm $ME$ can be
  decomposed into a movement perpendicular to $ME$, causing the
  measuring wheel to rotate, and a movement parallel to $ME$, causing
  the measuring wheel to skid, with no contribution to its reading.

  Use Green's theorem to explain why the final reading on the
  measuring wheel is equal to the area of the surface $S$.
\end{aexercise}

\asubsection{sec:surf}

\begin{aexercise}
  Calculate $\int_{\partial D}xy\,dS$, where $D=[0,1]^3$.
\end{aexercise}


\begin{aexercise}[enhanced,breakable]
  Consider the surface
  $S=\left\{(x,y,z)\in \R^3\,:\,x>0\text{ and }r = 1\text{ and
    }-\sqrt{\frac{\pi^2}{4}-\theta^2}\leq z\leq
    \sqrt{\frac{\pi^2}{4}-\theta^2}\right\}$, shown below. (Note that
  $r$ and $\theta$ refer to cylindrical coordinates.)

  (a) Find the surface area of $S$ by splitting it into vertical
  strips as shown and performing an appropriate integral. 

  (b) Check your answer by finding a non-calculus method of
  calculating the area of $S$.

\begin{center}
  \includegraphics{exercisefigures/label}
\end{center}
\end{aexercise}

\asubsection{sec:divcurl}

\begin{aexercise}
  Find the divergence and curl of $\vec{F} = (2x^2, xe^z, -4y)$.
\end{aexercise}

\begin{aexercise}
  Let $f(x,y) = \log(x^2 + y^2)$ for
  $(x,y)\in \R^2 \setminus \{(0,0)\}$. Show that
  $\nabla \cdot (\nabla f) = 0$.
\end{aexercise}

\begin{aexercise}
Find the flow of the vector field $\vec{F} = x^2\vec{i} + xy\vec{j}$ across the surface 
\[
z = 1-x^2-y^2, \quad z \geq 0. 
\]
\end{aexercise}

\asubsection{sec:divtheorem}

\begin{aexercise}
  Verify the divergence theorem for the cube bounded by the planes
  $x=0$, $x=1$, $y=0$, $y=1$, $z=0$, $z=1$ and the vector field
  $(3x,xy,2xz)$.
\end{aexercise}

\begin{aexercise}
  Confirm that the divergence theorem holds in the case $\mathbf{F} =
  \langle y, z, x \rangle$ and $D = \{(x,y,z) \,: \, x^2 + y^2 + z^2
  \leq 16\}$. 
\end{aexercise}

\begin{aexercise}
  Use the divergence theorem to calculate the flow of $\mathbf{F} =
  \langle x^4, -x^3 z^2, 4xy^2 z \rangle$ through the boundary of the
  region where $x^2 + y^2 \leq 1$ and $0 \leq z
  \leq x + 2$. 
\end{aexercise}

\begin{aexercise}
  Show that the volume of a three-dimensional region $D$ with
  piecewise smooth boundary is equal to 
  \[
  \iint_{\partial D} \mathbf{F} \cdot \, {\d}\mathbf{A},
  \]
  where $\mathbf{F} =\langle x, y, z \rangle$. 
\end{aexercise}

\begin{aexercise}
  \begin{minipage}[t]{0.7\linewidth}
    Use the divergence theorem to calculate the flux of the vector
    field $(x^2 \sin y, x \cos y, -xz \sin y)$ across the ``rounded
    cube'' \[x^8 + y^8 + z^8 = 8.\]
  \end{minipage} 
  \begin{minipage}[t]{0.3\linewidth}
    \begin{center}
      \raisebox{\dimexpr -\height +1.5ex
        \relax}{\includegraphics[width=4cm]{exercisefigures/roundedcube}}
    \end{center}
  \end{minipage}
\end{aexercise}

\asubsection{sec:stokes}

\begin{aexercise}
   Use Stokes' theorem to evaluate $\int_C\mathbf{F} \cdot \d\mathbf{r}$, where $C$ is the triangle with vertices $(1,0,0)$, $(0,1,0)$, and $(0,0,1)$, oriented counterclockwise when viewed from above, and $\mathbf{F} = (x+y^2,y+z^2,z+x^2)$. 
\end{aexercise}

\begin{aexercise}
  Verify Stokes' theorem for a closed sphere with the vector field
$\mathbf{F} = (-y^3,x,z)$. 
\end{aexercise}

\begin{aexercise}
  Find the work done on a particle by a force $\mathbf{F}=\langle
  xyz−e^x, −xyz, x^2yz+sin z \rangle$ on a particle that moves along the line segments from $(0, 0, 0)$, then to $(1, 1, 1)$, then to $(0, 0, 2)$, then back to $(0, 0, 0)$.
\end{aexercise}

\begin{aexercise}
Show that if $C$ is a simple smooth curve contained in the plane
$x+y+z = 1$, then the line integral
\[\int_C \langle z, 2x, 3z \rangle \cdot \d\mathbf{r}\]
depends only on the area of the region enclosed by $C$. This means
that if $C_1$ and $C_2$ are any two such curves enclosing the same area,
then the integrals around $C_1$ and $C_2$ are equal. 
\end{aexercise} 


\end{document}
